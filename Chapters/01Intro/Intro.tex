
%% % % %第一段: 提出并介绍这个问题
%The problem in the world I explore in this paper is....... 
%The reason why it is an important problem is......
%
%% % % %第二段:找出解决问题的视角
%The lends I adopt to explore this problem is.... it is a useful lens because..........
%
%% % % %第三段:描述解决问题的场景和发现Analyzing a dataset of xxxxxxxxxxx, 
%I reach the following findings.......
%
%% % % %第四段:介绍你的贡献
%This study contributes to.........

%Inference and Characterization of Planar Trajectories
%Inference and Characterization of Planar Trajectories
%Inference and Characterization of Planar Trajectories


\section{Background}


The Global Positioning System (abbreviated as GPS) is a space-based navigation system consisting of a network of 24 satellites in six different 12-hour orbital paths spaced. So that at least five of them are in view from every point on the globe \cite{kaplan2005understanding} \cite{bajaj2002gps}. With these satellites, a GPS receiver provides geographic information to them and triangulate its location on the ground, such as longitude, latitude and elevation. These receivers are using passive locating technology, by which they can receive signals without transmitting any data. It is perhaps the most widely publicized location-sensing system providing an excellent lateration framework for determining geographic positions \cite{hightower2001location}. Offered free of charge and accessible worldwide, GPS is used in plenty applications in military and general public, including aircraft tracking, vehicle navigation, robot localization, surveying, astronomy and so on. 

For a moving vehicle, it is often mounted with a GPS unit that communicates with the satellites and records position, average speed and direction of traveling information. With this kind of information, a maneuvering target tracking system becomes available and useful. This tracking system can be used to reduce the cost by knowing in real-time the current location of a vehicle, such as a truck or a bus \cite{chadil2008real}, and also be very useful for Intelligent Transportation System (ITS) \cite{mcdonald2006intelligent}. For instance, it can be used in probe cars to measure real-time traffic data to identify the congesting area. In particular, an orchardist is able to follow the trajectory and motion patterns of tractors, which are working on an orchard or vineyard, and monitor their working status with tracking system. 

No doubt that in a conventional target tracking system, the most common method is the standard Kalman filter \cite{kalman1960new}, which is a recursive solution to the discrete data linear filtering problem. The Kalman filter is a set of mathematical equations that provides an efficient computational means to estimate the state of a process in a recursive way, that minimizes the mean of the squared errors \cite{bishop2001introduction}. \cite{tusell2011kalman} gives a review of some \textit{R} packages, which are used to fit data with Kalman filter methods. Limited to its property, Kalman filter is tied up for a dynamic system, where the parameters and noise variances are unknown. In some dynamic systems, the variances are obtained based on the system identification algorithm, correlation method and least squares fusion criterion. To solve this issue, a self-tuning weighted measurement fusion Kalman filter is presented in \cite{ran2010self}. Likewise, a new adaptive Kalman filter will be another choice \cite{oussalah2001adaptive}. 

When the target maneuver occurs, Kalman filtering accuracy will be reduced or even diverged due to the model mismatch and noise characteristics that cannot be known exactly \cite{liu2014filtering}. Additionally, Kalman filter based methods require the state vector contains pre-specified coefficients during the whole approximation procedure and are within the bounded definition range determined at the beginning \cite{jauch2017recursive}. It is suggested that the smoothing spline fitting through the measured values reflecting the movements is an alternative approach. Even though fitting a univariate local linear trend model using a Kalman filter is equivalent to fitting a cubic spline, see \cite{eubank2004simple}, \cite{durbin2012time}, the latter algorithm overcomes the current limitation and can approximate data points throughout the whole process. 

 

\section{Smoothing Spline Based Reconstruction}

Starting from interpolation, the simplest way of connecting a set of sequences $(t_1,y_1),\ldots, (t_n,y_n)$ is by straight line segments. Known as piecewise
linear interpolation. Apparently, sharp turnings occur at joint knots. A smooth path is expected and more common in real life application. A single polynomial function goes through the entire interval, such as B\'ezier curve, is not as flexible as a combination of several polynomials, each of which is defined on subintervals and joint at certain knots. This kind of piecewise polynomial interpolation is called a spline. 

The core idea of splines is to augment the vector of inputs $T$ with additional variables, then use linear models in this space of derived input features. Adding constraints to construct basis functions $h_i(t), i = 1, 2,\ldots, n$, a linear basis expansion in $T$ is represented as
\begin{equation*}
f(t)=\sum_{i=1}^n \theta_i h_i(t).
\end{equation*}
The key step of a spline interpolation is the choice of basis functions. Once the $h_i(t)$ have been determined, the models are linear in the variables space. 

Several kinds of splines were introduced by \cite{esl2009}. One of them is B-splines, short for basis spline, which are constructed from polynomial pieces and joint at knots. B-spline have a closed-form expression of positions and allows continuity between the curve segments and goes through the points smoothly with ignoring the outliers, see \eg \cite{komoriya1989trajectory}, \cite{ben2004geometric}. It is flexible and has minimal support with respect to a given degree, smoothness, and domain partition. Once the knots are give, it is easy to compute B-spline recursively for any desired degree of the polynomial \cite{de1978practical} \cite{cox1982practical}. Basic simplicity of the idea is explained in \cite{dierckx1995curve} and \cite{eilers1996flexible}. The attractive feature of B-spline is its flexibility for univariate regression. For example,  \cite{gasparetto2007new} is using fifth-order B-spline to compose the overall trajectory with a set of paired data. Roughly, every spline can be represented by B-spline basis. 

Another widely used spline is piecewise cubic spline, which is continuous on interval $[a,b]$ and its first and second derivatives are continues as well \cite{wolberg1988cubic}. For example, consider the model in the following form 
\begin{equation*}
f(t)=d_i(t-t_i)^3+c_i(t-t_i)^2+b_i(t-t_i)+a_i,
\end{equation*}
for given coefficients $d_i, c_i, b_i$ and $a_i$, where $t_i\leq t\leq t_{i+1}$, $i=1,2,\ldots,n$. $f$ is a cubic spline on $[a,b]$ if (1) on each intervals $f$ is a polynomial; (2) the polynomial pieces fit together at knots $t_i$ in such a way that $f$ itself and its first and second derivatives are continuous at each $t_i$. If the second and third derivatives of $f$ are zero at 0 and 1, $f$ is said to be a natural cubic spline \cite{green1993nonparametric}. 


To find the best estimation $\hat{f}(t)$ goes through observation $y_i$, $i=1,\ldots,n$, one can use regression methods returning the smallest sum square errors among all the sequences. Consider a model $y_i=f(t_i)+\varepsilon_i$ with random errors $(\varepsilon_i)_{i=1}^n \sim N(0,\sigma^2)$ in space $\mathit{C}^m[a,b]$, classical parametric regression assumes that $f$ has the form $f(t,\beta)$, which is known up to the data estimated parameters $\beta$ \cite{kim2004smoothing}. When $f(t,\beta)$ is linear in $\beta$, we will have a standard linear model. 


However, a parametric approach only captures features contained in the preconceived class of functions and increases model bias \cite{yao2005functional}. To avoid this, nonparametric methods have been developed. Rather than giving specified parameters, it is desired to reconstruct $f$ from the data $y(t_i)\equiv y_i$ itself  \cite{craven1978smoothing}. The estimates of polynomial smoothing splines appear as a solution to the following minimization problem: find $\hat{f} \in \mathit{C}^m[a,b]$ that minimizes the penalized residual sum of squares: 
\begin{equation}\label{introSmoothingOb}
\mbox{RSS}=\sum_{j=1}^{n}\left(  y_j-f(t_j)\right) ^2+\lambda\int_a^b (f^{(m)})^2dt
\end{equation}
for pre-specified value $\lambda>0$ \cite{aydin2012smoothing}. In the above equation, the first term is the residual sum squares controlling the lack of fit. The second term is the roughness penalty weighted by a smoothing parameter $\lambda$, which varies from $0$ to $+\infty$ and establishes a trade-off between interpolation and straight line in the following way: 
\begin{align*}
\begin{cases}
\lambda = 0  & \mbox{$f$ can be any function that interpolates the data}\\
\lambda = +\infty & \mbox{the simple least squares line fit since no second derivative can be tolerated}\\
\end{cases}
\end{align*}\cite{esl2009}. 
So the cost of the equation (\ref{introSmoothingOb}) is determined not only by its goodness-of-fit to the data quantified by the residual sum of squares, but also by its roughness \cite{schwarz2012geodesy}. The motivation of the roughness penalty term is from a formalization of a mechanical device: if a thin piece of flexible wood, called a spline, is bent to the shape of the curve $g$, then the leading term in the strain energy is proportional to $\int f''^2$ \cite{green1993nonparametric}. 




%
%\subsection{Bayesian Estimation of Polynomial Smoothing Spline}
%
%Being bounded on a Hilbert space $C^{(m)}[0,1]$ with an inner product  $\langle f,g\rangle=\int_0^1fgdt$, \cite{wahba1978improper} showed that a Bayesian version of smoothing spline problem is to take a Gaussian process prior $f(t_i) = a_0+a_1t_i+\cdots + a_{m-1}t_i^{m-1} + x_i$, on $f$ with $x_i=X(t_i)$ being a zero mean Gaussian process whose $m$th derivative is scaled white noise, $i=1,\ldots,n$ \cite{speckman2003fully}. The extended Bayes estimates $f_\lambda$ with a "partially diffuse" prior is as exactly the same as spline solution. Some works have been done on discovering the relationship between nonparametric regression and Bayesian estimation. \cite{heckman1991minimax} shows that if $f$ the regression function $\E(y\mid f)$ has unknown prior distribution  $\mathbf{f}=(f(t_1),\ldots,f(t_n))^\top$ lying in a known class of $\Omega$, then the maximum is taken over all priors in $\Omega$ and the minimum is taken over linear estimator of $\mathbf{f}$. \cite{branson2017nonparametric} propose a Gaussian process regression method that acts as a Bayesian analog to local linear regression for sharp regression discontinuity designs. It is no doubt that one of the attractive features of the Bayesian approach is that, in principle, one can solve virtually any statistical decision or inference problem. Particularly, one can provide an accuracy assessment for $\hat{f}=\E (f\mid \mathbf{y})$ using posterior probability regions \cite{cox1993analysis}. 
%
%
%Based on the correspondence, \cite{craven1978smoothing} proposed an generalized cross-validation estimate for the minimizer $f_\lambda$. The estimate $\hat{\lambda}$ is the minimizer of the function where the trace of matrix $A(\lambda)$ in (\ref{crossvalidationmatrixA}) is incorporated. It is also able to establish an optimal convergence property for their estimator when the number of observations in a fixed interval tends to infinity \cite{wecker1983signal}. A high efficient  algorithm to optimize generalized cross-validation and generalized maximum likelihood scores with multiple smoothing parameters via the newton method was proposed by \cite{gu1991minimizing}. This algorithm can also be applied to the maximum likelihood and the restricted maximum  likelihood estimation. The behavior of the optimal regularization parameter in the method of regularization was investigated by \cite{wahba1990optimal}. 
%



\section{Parameter Selection}

As is discussed in the previous section, the determination of an optimum smoothing parameter $\lambda$ in the interval $(0,+\infty)$ was found to be an underlying complication. Various studies for selecting an appropriate smoothing parameter are developed and compared in literatures. 

Craven and Wahba [5], Hardle [8], Hardle, Hall and Marron
[9], Wahba [26], Hurvich, et al. [11], Eubank [6], Lee and Solo [17], Hastie and Tibshirani
[10], Schimek [22], Cantoni and Ronchetti [4], Ruppert, Wand and Carroll [21], Lee [15, 16],
and Kou [12] supplement on the selection of the smoothing parameter



Almost every technique found in the scientific literature on the reconstruction and trajectory planning problem is based on the optimization of some parameters or some objective function, such as equation (\ref{introSmoothingOb}), \cite{gasparetto2007new}. The most significant optimality criteria are: 
\begin{itemize}
\item[1] minimum execution time,
\item[2] minimum energy (or actuator effort),
\item[3] minimum jerk.
\end{itemize}
Besides the aforementioned approaches, some hybrid optimality criteria have also been proposed (e.g. time energy optimal trajectory planning).





Some ways are give to estimate the parameter \cite{wood2000modelling}, \cite{kim2004smoothing}. However, we hope that $\lambda$ could be a piecewise parameter, rather than a constant number, then we will have piecewise penalty functions. \cite{donoho1995wavelet} introduced adaptive splines and a method to calculate piecewise parameters and \cite{liu2010data} give an improved formula for this method . But to get these parameters, we have to compromise to use adaptive smoothing splines. 



\textbf{Temporally put here }Due to the noise generated from observation units, one can use regression methods to find the best reconstruction returning the smallest sum square errors among all the sequences. Consider a regression model $y_i=f(t_i)+\epsilon_i$, where $a \leq t_1 < \cdots < t_n \leq b$ and $f \in \mathit{C}^2[a,b]$ is an unknown smooth function, $(\epsilon_i)_{i=1}^n \sim N(0,\sigma^2)$ are random errors. In a classical parametric regression, $f$ is assumed having the form $f(x,\beta)$, which is known up to the data estimated parameters $\beta$ \cite{kim2004smoothing}. When $f(x,\beta)$ is linear in $\beta$, we will have a standard linear model. 


Similar to the conventional smoothing spline problem, one have to to choose the penalty function $\lambda(t)$. The fundamental idea of nonparametric smoothing is to let the data choose the amount of smoothness, which consequently decides the model complexity \cite{gu1998model}. Most methods focus on data driven criteria, such as cross validation (CV), generalized cross validation (GCV) \cite{craven1978smoothing} and generalized maximum likelihood (GML) \cite{wahba1985comparison}. A new challenge is posed that the smoothing parameter becomes a function and is varying in domains. The structure of this penalty function controls the complexity on each domain and the whole final model. Liu and Guo proposed to approximate the penalty function with an indicator and extended the generalized likelihood to the adaptive smoothing spline \cite{liu2010data}.






\subsection*{Frequentist Algorithm}

Based on the procedure given by \cite{wahba1975completely}, we  follow the improved steps to calculate a K-fold cross validation. 

Step 1. Remove the first data $t_1$ and last date $t_n$ from the dataset.

Step 2. Divide dataset into k groups:
\begin{align*}
& \mbox{Group 1}: t_2, t_{2+k}, \cdots \\
& \mbox{Group 2}: t_3, t_{3+k}, \cdots \\
& \vdots \\
& \mbox{Group k}: t_{k+1}, t_{2k+1}, \cdots
\end{align*}


Step 3. Guess values of $\lambda_{down}, \lambda_{up}$ and $\gamma$.

Step 4. Delete the first group of data. Fit a smoothing spline to the first data, the rest groups of dataset and the last data, with  $\lambda_{down}, \lambda_{up}$ and $\gamma$ in step 3. Compute the sum of squared deviations of this smoothing spline from the deleted data points.

Step 5. Delete instead the second group of data. Fit a smoothing spline to the remaining data with  $\lambda_{down}, \lambda_{up}$ and $\gamma$. Compute the sum of squared deviations of the spline from deleted data points.

Step 6. Repeat Step 5 for the 3rd, 4th, $\cdots$, $k$th group of data.

Step 7. Add the sums of squared deviations from steps 4 to 6 and divide by $k$. This is the cross validation score of three parameters  $\lambda_{down}, \lambda_{up}$ and $\gamma$.

Step 8. Vary  $\lambda_{down}, \lambda_{up}$ and $\gamma$ systematically and repeat steps 4-7 until CV shows a minimum.



\subsection{Bayesian Algorithm}






\section{Filtering Methods}



As the development of technology of science and real life, the "big data" challenge becomes ubiquitous. Classical methods, such as Markov Chain Monte Carlo (MCMC), are normally suitable and good at handling a batch of data forecasting and analyzing. However, for big data and instant updating data stream, more robust and efficient methods are required. 


Alternative approaches, such as Sequential Monte Carlo, for on-line updating and estimating are well studied in scientific literature and quite prevalent in academic research in the last decades. When it embraced with  state space model, which is a very popular class of time series models that have found numerous of applications in fields as diverse as statistics, ecology, econometrics, engineering and environmental sciences \cite{cappe2009inference} \cite{smcmip2011} \cite{elliott1995estimation} \cite{cargnoni1997bayesian}, it allows us to establish complex linear and nonlinear Bayesian estimations in time series patterns \cite{vieira2016online}. 




\section{Sequential Markov Chain Monte Carlo Algorithm}


Data assimilation is a sequential process, by which the observations are incorporated with a numerical model describing the evolution of this system throughout the whole process. It is applied in many fields, particularly in weather forecasting and hydrology. The quality of the numerical model determines the accuracy of this system, which requires sequential combined state and parameters inferences. Enormous literature has been done on discussing pure state estimation, however, fewer research is talking about estimating combined state and parameters, particularly in a sequential updating way. 

Sequential Monte Carlo method is well studied in the scientific literature and quite prevalent in academic research in the last decades. It allows us to specify complex, non-linear time series patterns and enables performing real time Bayesian estimations when it is coupled with Dynamic Generalized Linear Models \cite{vieira2016online}. However, model's parameters are unknown in real world application and it is a limit for standard SMC. Extensions to this algorithm have been done by researchers. Kitagawa \cite{kitagawa1998self} proposed a self-organizing filter and augmenting the state vector with unknown parameters. The state and parameters are estimated simultaneously by either a non-Gaussian filter or a particle filter. Liu and West \cite{liu2001combined} proposed an improved particle filter to kill degeneracy, which is a normal issue in static parameters estimation. They are using a kernel smoothing approximation, with a correction factor to account for over-dispersion. Alternatively, Strovik \cite{storvik2002particle} proposed a new filter algorithm by assuming the posterior depends on a set of sufficient statistics, which can be updated recursively. However, this approach only applies to parameters with conjugate priors \cite{stroud2016bayesian}. Particle learning was first introduced in \cite{carvalho2010particle}. Unlike Strovik filter, it is using sufficient statistics solely to estimate parameters and promises to reduce particle impoverishment. These particle like methods are all using more or less sampling and resampling algorithms to update particles recursively. 

Jonathan proposed in \cite{stroud2016bayesian} a SMC algorithm by using ensemble Kalman filter framework for high dimensional space models with observations. Their approach combines information about the parameters from data at different time points in a formal way using Bayesian updating. In \cite{polson2008practical}, the authors rely on a fixed-lag length of data approximation to filtering and sequential parameter learning in a general dynamic state space model. This approach allows for sequential parameter learning where importance sampling has difficulties and avoids degeneracies in particle filtering. A new adaptive Markov Chain Monte Carlo method yields a quick and flexible way for estimating posterior distribution in parameter estimation \cite{haario1999adaptive}. This new Adaptive Proposal method, depends on history data, is introduced to avoid the difficulties of tunning the proposal distribution in Metropolis-Hastings methods. 



In this paper, I'm proposing an Adaptive Delayed-Acceptance Metropolis-Hastings algorithm to estimate the posterior distribution for combined states and parameters. To keep a higher running efficiency for the algorithm, we suggest cutting off history data but jut using a fixed length of data up to current state, like a sliding window. Once a new observation comes into the data stream, this window shift one step forward with the same length. The efficiency of this algorithm is discussed regarding to tune the best step size in learning process. 

\section{Summary}



In our case, the velocity data set $v_i$ with some independent Gaussian distributed errors $\varepsilon_i \sim N(0, \frac{\sigma_n^2}{\gamma})$ are used to estimate $f(t)$ simultaneously. $f$ is a linear combination of basis functions, as shown in equation (\ref{introfbasis}), in the meantime, $f'$ is a linear combination of the first derivative of these basis functions
\begin{equation}
f'(t) =\sum_{m=1}^{M}\alpha_mh'_m(t).
\end{equation}

The velocity information  is incorporated into MSE  equation (\ref{intromse}) by the addition of velocity term $(f'(t_i)-v_i)^2$. Then it becomes 
\begin{equation}\label{mse2}
\text{MSE}(f,\lambda,\gamma)=\frac{1}{n}\sum_{i=1}^n(f(t_i)-y_i)^2+\frac{\gamma}{n} \sum_{i=1}^n(f'(t_i)-v_i)^2+\lambda \int_{0}^{1}(f''(t))^2dt,
\end{equation}
and $\hat{f}$ is the minimizer of the MSE equation (\ref{mse2}).

In the model $y=f(t)+\varepsilon$, it is reasonable to assume that the observed data $y_i$ is Gaussian distribution with mean $f(t_i)$ and variance $\sigma_n^2$. In a similar way, the velocity is estimated as  $v=f'(t)+\frac{\varepsilon}{\gamma}$, where $v_i$ is Gaussian distribution with mean $f'(t_i)$ and variance $\frac{\sigma_n^2}{\gamma}$. Then the joint distribution of $\mathbf{y},\mathbf{v},f(t)$ and $f'(t)$ is normal with zero mean and a covariance matrix, which can be estimated through Gaussian Process Regression.


Given a series of such fixes, I want to construct the most likely path taken by the tractor with some smart modeling of the movement. I then want to develop higher level characterizations of the trajectories and apply this to more general problems in curve and object recognition.


