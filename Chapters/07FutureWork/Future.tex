

\subsection*{On-line Tractor Spline}

Tractor spline reconstruction is an advanced smoothing spline algorithm returning least true mean squared errors. However, to implement this algorithm on-line needs feasible solutions. One of them probably is combining spline method and gradient boosting algorithm. 

The gradient boosting algorithm is a powerful machine-learning technique that has shown considerable success in a wide range of practical applications, particularly in machine learning competitions on \textit{Kaggle}. The motivation of gradient boosting algorithm is combining weak learners together as a strong leaner. This algorithm has high customizable application to meet particular needs, like being learned with respect to different loss functions. 

The commonly used base-learner models can be classified into a three distinct categories: linear models, smooth models and decision trees. 


\subsection*{Error Analysis}


\subsection*{GCV for Non-trivial Tractor Spline}


\subsection*{Proposing De-correlated Samples}

Propose samples on two direction $\begin{bmatrix}
W & 0 \\ 0 & U
\end{bmatrix}$.


\subsection*{Grid-based MCMC}


Grid-based methods provide the optimal recursion of the filtered density $p(x_t\mid y_{1:t})$ if the state space is discrete and consists of a finite number of states \cite{ristic2004beyond}. 

MCMC grid based methods. 

A Grid-based Filter for Tracking Bats Applying
Field Strength Measurements.


Grid Based Variational Approximations

Ensemble KF. 