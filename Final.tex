\documentclass[12pt,twoside]{report}    

\usepackage[phd]{otagofront}   %% Use Otago page layout
\usepackage{natbib} 		%% author/year format in round brackets
\usepackage{graphicx}           %% jpg, gif, tiff, and pdf graphics
\usepackage{moreverb}           %% Verbatim Code Listings
\usepackage{amsmath}		%% More math commands
\usepackage{amssymb}		%% More math commands
\usepackage{amsthm}		%% More math commands
\usepackage{mathrsfs}		%% More math commands
\usepackage{color}		%% Handling colors
\usepackage{enumerate}          %% Fancy enumeration lists
\usepackage{lscape}
\usepackage{indentfirst}
\usepackage{siunitx}
\usepackage{lscape}
\usepackage{subcaption}
\usepackage[toc,page]{appendix}
\usepackage[ruled,vlined,linesnumbered,resetcount,algochapter]{algorithm2e}
\usepackage{romannum}
\usepackage{multirow}
\usepackage{array}
\newcolumntype{C}[1]{>{\centering\arraybackslash}p{#1}}


% % % % % % % % %  tikz package % % % % % % % %
\usepackage{tikz}
\usetikzlibrary{arrows.meta,calc,backgrounds}
\usepackage{pgfplots}

\tikzset{
    right angle quadrant/.code={
    \pgfmathsetmacro\quadranta{{1,1,-1,-1}[#1-1]}     % Arrays for selecting quadrant
    \pgfmathsetmacro\quadrantb{{1,-1,-1,1}[#1-1]}
    },
    right angle quadrant=1, % Make sure it is set, even if not called explicitly
    right angle length/.code={\def\rightanglelength{#1}},   % Length of symbol
    right angle length=2ex, % Make sure it is set...
    right angle symbol/.style n args={3}{
        insert path={
            let \p0 = ($(#1)!(#3)!(#2)$) in     % Intersection
                let \p1 = ($(\p0)!\quadranta*\rightanglelength!(#3)$), % Point on base line
                \p2 = ($(\p0)!\quadrantb*\rightanglelength!(#2)$) in % Point on perpendicular line
                let \p3 = ($(\p1)+(\p2)-(\p0)$) in  % Corner point of symbol
            (\p1) -- (\p3) -- (\p2)
        }
    }
}
% % % % % % % % %  tikz package % % % % % % % %
%%%%%%%%%%%%%%%%%%%%%%%%%%%%%  New commands 

\newcommand{\BigO}[1]{{\rm O}\left(#1\right)}
\newcommand{\expo}{\,\mathrm{e}}
\newcommand{\cvar}{\,\mathrm{i}}
\newcommand{\besseli}{\mathrm{I}}
\newcommand{\besselk}{\mathrm{K}}
\newcommand{\bessely}{\mathrm{Y}}
\newcommand{\besselj}{\mathrm{J}}
\newcommand{\eg}{e.g.\ }
\newcommand{\ie}{i.e.\ }
\newcommand{\iid}{\textrm{i.i.d.\ }}
\newcommand{\elemvec}[2]{\big[#1\big]_{#2}}
\newcommand{\ud}[1]{\,\mathrm{d}#1}
\newcommand{\hPhi}{\widehat{\Phi}}
\newcommand{\diag}[1]{\text{diag}\left\{#1\right\}}
\renewcommand{\Re}{\operatorname{Re}}
\renewcommand{\Im}{\operatorname{Im}}
\renewcommand{\labelenumi}{(\roman{enumi})}

%\newcommand{\TODO}[1]{\textcolor{red}{\bf{[#1]}}}
%\newcommand{\EDITED}[1]{\textcolor{blue}{\bf{[#1]}}}
\newtheorem{lemma}{Lemma}
\newtheorem{theorem}{Theorem}
\newtheorem{corollary}{Corollary}
\newcommand{\E}{\mathrm{E}}
\newcommand{\Var}{\mathrm{Var}}
\newcommand{\Cov}{\mathrm{Cov}}
\newcommand{\Corr}{\mathrm{Corr}}
\newcommand{\tr}{\mathrm{tr}}




%%%%%%%%%%%%%%%%%%%%%%%%%%% Front page info 

\title{Inference and Characterization of Planar Trajectories}
\author{Zhanglong Cao}
% %\date{30 August 2017}
\date{}
%%%%%%%%%%%%%%%%%%%%%%%%%%%%%% Extra info 

\fullname{Zhanglong Cao}
\department{Department of Mathematics and Statistics}
\dob{4 March 1984} %% date-of-birth
\address{46 St David Street, Dunedin, New Zealand}



%% Uncomment to just print up a few chapters.
%%
%\includeonly{}

\begin{document}


%% Put in titlepage and contents, etc...
\frontstuff

%% Set to one-and-a-half line-spacing
\linespread{1.3} \normalsize


\chapter{Introduction}\label{ChapterIntro}


\section{Background}


The \textit{Global Positioning System} (abbreviated as GPS) is a space-based navigation system consisting of a network of 24 satellites placed in space in six different 12-hour orbital paths \citep{agrawal2015introduction}, so that at least five of them are in view from every point on the globe \citep{kaplan2005understanding, bajaj2002gps}. A GPS device receives signals from these satellites and triangulates its location in terms of longitude, latitude, and elevation. GPS is the most widely known location-sensing system providing an excellent framework for determining geographic positions \citep{hightower2001location}. Offered free of charge and accessible worldwide, GPS has a vast number of applications, including aircraft tracking, vehicle navigation, robot localization, surveying, astronomy, and so on. 

GPS units in vehicles typically record position, speed, and direction of travel. With this information, a target tracking system becomes available and useful. Such a tracking system can be used to reduce costs by knowing in real-time the current location of a vehicle, such as a truck or a bus \citep{chadil2008real}, with applications to  Intelligent Transportation Systems (ITS) \citep{mcdonald2006intelligent}. It can also be used to measure real-time traffic data and to identify congestion areas. In farming applications, a tracking system allows the location and operational status of farm vehicles to be monitored remotely. 


%It is suggested that the smoothing spline fitting through the measured values reflecting the movements is an alternative approach. Even though fitting a univariate local linear trend model using a Kalman filter is equivalent to fitting a cubic spline, see \citep{eubank2004simple, durbin2012time}, the latter algorithm overcomes the current limitation and can approximate data points throughout the whole process. 


Given time series data from a vehicle-mounted GPS unit, an important question is how to infer the trajectory of the vehicle. This is known as trajectory reconstruction and is the motivating question of this thesis. 




\section{The Problem}

Two keys issues for reconstruction are (\romannum{1}) how to handle observations that are inherently noisy measurements of the truth, and (\romannum{2}) how to interpolate appropriately between observation times.

GPS units in vehicles provide $y_t$, noisy measurements of the actual position $x_t$, and $v_t$, noisy measurements of the actual velocity $u_t$, for a sequence of times $t\in T$. These data may also be augmented with information on operating characteristics of the vehicle, $b_t$. The trajectory reconstruction problem is the problem of estimating $x_s$, for an arbitrary time $s$, given a subset of the observations $\{y_t,v_t,b_t\mid t\in T\}$. Note that in this definition of trajectory reconstruction, we are not explicitly interested in estimating $u_s$.


The \textit{TracMap} company, located in New Zealand and USA, produces GPS display units to aid precision farming in agriculture, horticulture and viticulture. Operational data is collected and sent by these units to a remote server for further analysis. An example of position data, which has been subsampled at irregular time points, is given in Figure \ref{tracoverview}. 

%Generally, this data is irregularly sampled every 1 to 30 seconds over the whole day. 
\begin{figure}[h]
	\centering
	\begin{tikzpicture}
	\node[anchor=south west,inner sep=0] (image) at (0,0) {\includegraphics[width=0.45\textwidth]{Chapters/01Intro/plots/tracoverview01.pdf}};
	\begin{scope}[
	x={(image.south east)},
	y={(image.north west)}
	]
	\node [black, font=\bfseries] at (0.5,0.0) {Easting};
	\node [black, font=\bfseries,rotate=90] at (0.05,0.5) {Northing};
	\end{scope}
	\end{tikzpicture}
	\begin{tikzpicture}
	\node[anchor=south west,inner sep=0] (image) at (0,0) {\includegraphics[width=0.45\textwidth]{Chapters/01Intro/plots/tracoverview02.pdf}};
	\begin{scope}[
	x={(image.south east)},
	y={(image.north west)}
	]
	\node [black, font=\bfseries] at (0.5,0.0) {Easting};
	\node [black, font=\bfseries,rotate=90] at (0.05,0.5) {Northing};
	\end{scope}
	\end{tikzpicture}
	\caption{Examples of GPS data. Observed positions $y_t$ are shown. In trajectory reconstruction, the $y_t$ are combined with velocity information $v_t$ and operating characteristics $b_t$ to infer actual positions $x_s$, for times of interest $s$.}
	\label{tracoverview}
\end{figure}





\section{Smoothing Spline Based Reconstruction}

Smoothing spline approaches are natural solutions to trajectory reconstruction, see 
\eg \cite{eubank2004simple} and \cite{durbin2012time} for details. 

Some form of interpolation is an obvious approach to trajectory reconstruction. The simplest method, piecewise linear interpolation, connects successive locations by straight lines. Clearly, this interpolation implies abrupt changes in velocity at the join points. Smooth trajectories are more common in real life applications. A single polynomial function that is defined on the entire interval, such as B\'ezier curve, is not as flexible as a piecewise combination of polynomials, each of which is defined on a subinterval. The polynomials are jodined at the endpoints of their subintervals -- these endpoints are termed \textit{knots}. This kind of piecewise polynomial interpolation is called a \textit{spline}. 

%The core idea of splines is to augment the vector of inputs $T$ with additional variables, then use linear models in this space of derived input features. Adding constraints to construct basis functions $h_i(t), i = 1, 2,\ldots, m$, a linear basis expansion in $T$ is represented as
%\begin{equation*}
%f(t)=\sum_{i=1}^m \theta_i h_i(t).
%\end{equation*}
%The key step of a spline interpolation is the choice of basis functions. Once the $h_i(t)$ have been determined, the models are linear in the variables space. 

A number of splines are commonly in use, see \cite{esl2009} for discussions. The B-spline, short for basis spline, gives a closed-form expression for the trajectory with continuous second derivatives and goes through the points smoothly while ignoring outliers \citep{komoriya1989trajectory, ben2004geometric}. It is flexible and has minimal support with respect to a given degree, smoothness, and domain partition. Once the knots are given, it is easy to compute the B-spline recursively for any desired degree of the polynomial \citep{de1978practical, cox1982practical}. An attractive feature of the B-spline is its flexibility for univariate regression and its appealing simplicity of the method is explained in \citep{dierckx1995curve, eilers1996flexible}. \cite{gasparetto2007new} use a fifth-order B-spline to compose the overall trajectory. Almost every spline can be represented as a B-spline. 

Another widely used spline is the piecewise cubic spline, which is continuous on an interval $[a,b]$ and has continuous first and second derivatives \citep{wolberg1988cubic}. Let $f(t)$ denote a trajectory reconstruction, \ie the location of the vehicle at time $t$. For the piecewise cubic spline,  
\begin{equation*}
f(t)=d_j(t-t_j)^3+c_j(t-t_j)^2+b_j(t-t_j)+a_j,
\end{equation*}
on the subinterval $[t_j,t_{j+1})$ with given coefficients $d_j, c_j, b_j$ and $a_j$, $j=1,2,\ldots,K$. The coefficients are chose in such a way that $f$ and its first and second derivatives are continuous at each knot $t_j$. If the second derivative of $f$ is zero at $a$ and $b$, $f$ is said to be a \textit{natural cubic spline} and these conditions are called the \textit{natural boundary conditions}  \citep{green1993nonparametric}. 


Given observations $\left(t_i,y_i\right)$, $i=1,\ldots,n$ ($n\geq K$), one can use regression methods to estimate $f(t)$. Specifically, let $y_i=f(t_i)+\varepsilon_i$ with random errors $\left\lbrace \varepsilon_i\right\rbrace_{i=1}^n \sim N\left(0,\sigma^2\right)$. In the case of the natural cubic spline, where $f\in \mathit{C}^{(2)}[a,b]$, this leads to a standard linear parametric model \citep{kim2004smoothing}. 


However, a parametric approach only captures features contained in the preconceived class of functions and increases model bias \citep{yao2005functional}. To improve the performances, nonparametric methods have been developed. Rather than giving specified parameters, it is desired to reconstruct $f$ from the data $y(t_i)\equiv y_i$ itself  \citep{craven1978smoothing}. The estimates of polynomial smoothing splines appear as a solution to the following minimization problem: find $\hat{f} \in \mathit{C}^{(m)}[a,b]$ that minimizes the penalized residual sum of squares: 
\begin{equation}\label{introSmoothingOb}
\mbox{RSS}=\sum_{i=1}^{n}\left(y_i-f(t_i)\right)^2+\lambda\int_a^b \left(f^{(m)}\right)^2dt, 
\end{equation}
for a pre-specified value $\lambda(>0)$ \citep{aydin2012smoothing}. In the above equation, the first term is the residual sum squares controlling the lack of fit. The second term is the roughness penalty weighted by a smoothing parameter $\lambda$, which varies from $0$ to $+\infty$ and establishes a trade-off between interpolation and a  straight line fit in the following way: 
\begin{align*}\small 
\begin{cases}
\lambda = 0  & \mbox{$f$ can be any function that interpolates the data}\\
\lambda = +\infty & \mbox{the simple least squares line fit since no second derivative can be tolerated}
\end{cases}
\end{align*}\citep{esl2009}. 

Hence, the cost of the equation \eqref{introSmoothingOb} is determined not only by its goodness-of-fit to the data quantified by the residual sum of squares but also by its roughness \citep{schwarz2012geodesy}. The motivation of the roughness penalty term is from a formalization of a mechanical device: if a thin piece of flexible wood, called a spline, is bent to the shape of the curve $g$, then the leading term in the strain energy is proportional to $\int f''^2$ \citep{green1993nonparametric}. 


%
%\subsection{Bayesian Estimation of Polynomial Smoothing Spline}
%
%Being bounded on a Hilbert space $C^{(m)}[0,1]$ with an inner product  $\langle f,g\rangle=\int_0^1fgdt$, \citep{wahba1978improper} showed that a Bayesian version of smoothing spline problem is to take a Gaussian process prior $f(t_i) = a_0+a_1t_i+\cdots + a_{m-1}t_i^{m-1} + x_i$, on $f$ with $x_i=X(t_i)$ being a zero mean Gaussian process whose $m$th derivative is scaled white noise, $i=1,\ldots,n$ \citep{speckman2003fully}. The extended Bayes estimates $f_\lambda$ with a ``partially diffuse'' prior is as exactly the same as spline solution. Some works have been done on discovering the relationship between nonparametric regression and Bayesian estimation. \citep{heckman1991minimax} shows that if $f$ the regression function $\E(y\mid f)$ has unknown prior distribution  $\mathbf{f}=(f(t_1),\ldots,f(t_n))^\top$ lying in a known class of $\Omega$, then the maximum is taken over all priors in $\Omega$ and the minimum is taken over linear estimator of $\mathbf{f}$. \citep{branson2017nonparametric} propose a Gaussian process regression method that acts as a Bayesian analog to local linear regression for sharp regression discontinuity designs. It is no doubt that one of the attractive features of the Bayesian approach is that, in principle, one can solve virtually any statistical decision or inference problem. Particularly, one can provide an accuracy assessment for $\hat{f}=\E (f\mid \mathbf{y})$ using posterior probability regions \citep{cox1993analysis}. 
%
%
%Based on the correspondence, \citep{craven1978smoothing} proposed an generalized cross-validation estimate for the minimizer $f_\lambda$. The estimate $\hat{\lambda}$ is the minimizer of the function where the trace of matrix $A(\lambda)$ in \eqref{crossvalidationmatrixA} is incorporated. It is also able to establish an optimal convergence property for their estimator when the number of observations in a fixed interval tends to infinity \citep{wecker1983signal}. A high efficient algorithm to optimize generalized cross-validation and generalized maximum likelihood scores with multiple smoothing parameters via the Newton method was proposed by \citep{gu1991minimizing}. This algorithm can also be applied to the maximum likelihood and the restricted maximum likelihood estimation. The behavior of the optimal regularization parameter in the method of regularization was investigated by \citep{wahba1990optimal}. 
%



\section{Parameter Selection}

As discussed in the previous section, the determination of an optimal smoothing parameter $\lambda$ in the interval $(0,+\infty)$ was found to be an underlying complication and the fundamental idea of nonparametric smoothing is to let the data choose the amount of smoothness, which consequently decides the model complexity \citep{gu1998model}. Various studies for selecting an appropriate smoothing parameter are developed and compared in literatures. Most of these methods are focusing on data driven criteria, such as cross-validation (CV), generalized cross-validation (GCV) \citep{craven1978smoothing} and generalized maximum likelihood (GML) \citep{wahba1985comparison} and recently developed methods, such as improved Akaike information criterion (AIC) \citep{hurvich1998smoothing}, exact risk approaches \citep{wand1997exact} and so on. See \eg \cite{craven1978smoothing, hardle1988far, hardle1990applied, wahba1990spline, green1993nonparametric, cantoni2001resistant, aydin2013smoothing} for details.  


%\subsection*{Cross-Validation}

A classical parameter selection method is \textit{cross-validation} (CV). The idea behind this method can be traced back to 1930s \citep{larson1931shrinkage}. Because in most applications, only a limited amount of data is available. Thus, an idea is to split this dataset into two subgroups, one of which is used for training the model and the other one is used to evaluate its statistical performance. The sample used in evaluation is considered as ``new data'' as long as data is \iid. 

A single data split yields a validation estimate of the risk and averaging over several splits yields a cross-validation estimate \citep{arlot2010survey}. Because of the assumption that data are identically distributed, and training and validation samples are independent, CV methods are widely used in parameter selection and model evaluation. 


For example, a $k$-fold CV splits the data into $k$ roughly equal-sized parts. For the $k$th part, we fit the model to the other $k-1$ parts of the data, and calculate the prediction error of the fitted model when predicting the $k$th part of the data. A detailed procedure is given by \cite{wahba1975completely}: suppose we have $n$ paired data $(t_1,y_1), \ldots, (t_n,y_n)$. Run a $k$-fold CV according to the following Algorithm \ref{kfoldCV}. 
\begin{algorithm}[h]
\SetAlgoLined 
%\KwResutt{Write here the resutt }
Initialization: Remove the first data $t_1$ and last date $t_n$ from the dataset. \\
Split the rest data $t_2,\ldots,t_{n-1}$ into $k$ groups by: for $i=1,\ldots,k$, the $i$th Group $G_i=\{t_{i+1}, t_{i+1+k}, t_{i+1+2k}, \cdots\}$.\\
Guess a value $\lambda^*$. \\
\While{CV score is not optimized}{
\For{$i=1,\ldots,k$}{ \label{kfoldCV01}
Delete the $i$th group of data. Fit a smoothing spline to the first data $(t_1,y_1)$, the rest $k-1$ groups of dataset and the last data $(t_n,y_n)$ with $\lambda^*$.\\ 
Compute the sum of squared deviations $s_i$ of this smoothing spline from the deleted $i$th group data points. \\ 
}\label{kfoldCV02}
Add the sums of squared deviations from steps \ref{kfoldCV01} to \ref{kfoldCV02} and divide it by $k$ to achieve a cross-validation score of $\lambda^*$, that is $s=\frac{1}{k}\sum_{i=1}^{k}s_i$. \\ \label{kfoldCV03}
Vary  $\lambda$ systematically and repeat steps \ref{kfoldCV01} to \ref{kfoldCV03} until CV shows a minimum.
}
\caption{$k$-fold Cross-Validation}\label{kfoldCV}
\end{algorithm}
Mathematically, we denote the CV score as 
\begin{equation*}
\mbox{CV}(\hat{f},\lambda) = \frac{1}{n}\sum_i^n \left( y_i -\hat{f}^{-k(i)}(t_i,\lambda) \right)^2,
\end{equation*}
where $\hat{f}^{(-k)}(t)$ denotes the fitted function computed with the $k$th part of the data removed. Typical choices for $k$ are 5 and 10 \citep{esl2009}.  The function $\mbox{CV}(\hat{f},\lambda)$ provides an estimate of the test error curve, and the tuning parameter $\lambda$ that minimizes it will be the optimal solution. 

A special case of $k$-fold CV is setting $k=n$, which is known as \textit{leave-one-out cross-validation}. In this scenario, the CV function takes each of the data out and calculate the errors of $\hat{f}^{(-i)}$ from the remaining $n-1$ points. In fact, with the property that taking one point out does not affect the estimation, the fitting of a smoothing spline allows us to implement CV methods without hesitation. 

As an improvement of CV, the GCV algorithm was proposed to calculate the trace of the estimation matrix $A(\lambda)$ instead of calculating individual elements for linear fitting under squared error loss, in which way it provides further computational savings. Suppose we have a solution $\hat{f}=A(\lambda)y$ with a given $\lambda$, for many linear fittings, the CV score is 
\begin{equation*}
\mbox{CV} = \frac{1}{n}\sum_{i=1}^{n} \left( y_i - \hat{f}^{(-i)}(t_i)\right)^2 = \frac{1}{n} \sum_{i=1}^{n}\left( \frac{y_i-\hat{f}(t_i)}{1-A_{ii}}  \right)^2, 
\end{equation*}
where $A_{ii}$ is the $i$th diagonal element of $A(\lambda)$. Then, the GCV approximation score is 
\begin{equation*}
\mbox{GCV} =\frac{1}{n} \sum_{i=1}^{n}\left( \frac{y_i-\hat{f}(t_i)}{1-\tr(A)/n}  \right)^2.
\end{equation*}
In smoothing problems, GCV can also alleviate the tendency of cross-validation to under-smooth \citep{esl2009}. 


Rather than $\lambda$ being constant, a new challenge is posed that the smoothing parameter becomes a function $\lambda(t)$ and is varying in domains. The structure of this penalty function controls the complexity of each domain and the whole final model.  \cite{donoho1995wavelet} introduce adaptive splines and a method to calculate piecewise parameters and \cite{liu2010data} give an improved formula for this method. They proposed an approximation to the penalty function with an indicator and extended the generalized likelihood to the adaptive smoothing spline. This will be another interesting research topic. 


%Craven and Wahba [5], Hardle [8], Hardle, Hall and Marron
%[9], Wahba [26], Hurvich, et al. [11], Eubank [6], Lee and Solo [17], Hastie and Tibshirani
%[10], Schimek [22], Cantoni and Ronchetti [4], Ruppert, Wand and Carroll [21], Lee [15, 16],
%and Kou [12] supplement on the selection of the smoothing parameter


%\textbf{Temporally put here }Due to the noise generated from observation units, one can use regression methods to find the best reconstruction returning the least sum square errors among all the sequences. Consider a regression model $y_i=f(t_i)+\epsilon_i$, where $a \leq t_1 < \cdots < t_n \leq b$ and $f \in \mathit{C}^2[a,b]$ is an unknown smooth function, $(\epsilon_i)_{i=1}^n \sim N(0,\sigma^2)$ are random errors. In a classical parametric regression, $f$ is assumed having the form $f(x,\beta)$, which is known up to the data estimated parameters $\beta$ \citep{kim2004smoothing}. When $f(x,\beta)$ is linear in $\beta$, we will have a standard linear model. 


%\subsection*{Improved AIC}
%
%The AIC, abbreviated for Akaike Information Criterion, is originally for parametric problems of model selection. 

Overall, almost every technique found in the scientific literatures on the reconstruction and trajectory planning problem is based on the optimization of objective functions or parameter selections, such as the objective function \eqref{introSmoothingOb} and cross-validation approaches \citep{gasparetto2007new}. 
% Besides the aforementioned approaches, the most significant optimality criteria are: (\romannum{1}) minimum execution time, (\romannum{2}) minimum energy (or actuator effort), (\romannum{3}) minimum jerk. 



\section{Bayesian Filtering}

Smoothing spline algorithms have several advantages in inferring and characterizing planar trajectories, particularly in reconstruction. However, subject to the property that smoothing splines require the solution of a global problem that involves the entire set of points to be interpolated, it might not be suitable for on-line estimation or instant updating \citep{biagiotti2013online}. It is the time to use Bayesian filtering to implement on-line/instant estimation and prediction. 

The word \textit{filtering} refers to the methods for estimating the state of a time-varying system, which is indirectly observed through noisy measurements. A Bayes filter is a general probabilistic approach to infer an unknown probability density function recursively over time using incoming measurements and a mathematical process model. The concept \textit{optimal estimation} refers to some criteria that measure the optimality in specific sense \citep{anderson1979optimal}. For example, a posterior estimation of $\hat{x}_t\sim p(x_t\mid y_{1:t})$ that minimizes the loss function $J_t=\E\lbrack x_t-\hat{x}_t\rbrack^2$ for incorrect estimates, or least mean squared errors, maximum likelihood approximation and so on. See \eg \cite{chen2003bayesian, sarkka2013bayesian} for discussions. Hence, an optimal \textit{Bayesian filtering} uses the Bayesian way of formulating optimal filtering by meeting some statistical criteria. 

It is no doubt that in a conventional target tracking system, the most common method is the standard Kalman filter, which is a recursive solution to the discrete data linear filtering problem. 



\subsection*{Kalman Filter}

In a discrete-time linear system, the optimal Bayesian solution coincides with the least squares solution. The successful optimal one was given by \cite{kalman1960new}, the famous \textit{Kalman filter}. It is a set of mathematical equations that provides an efficient computational means to estimate the state of a process in a recursive way by minimizing the mean of the squared errors \citep{bishop2001introduction}. 

The Kalman filter recursively updates the estimated state by computing from the previous estimation and a new observation. Without the need for storing the entire past observed data, the Kalman filter computes more efficient than computing the estimate directly from the entire past observed data at each step of the filtering process \citep{haykin2001kalman}.  

A detailed Kalman filter and its variants can be found in \citep{chen2003bayesian, rhodes1971tutorial, kailath1981lectures, sorenson1985kalman}. \cite{tusell2011kalman} gives a review of some \textit{R} packages, which are used to fit data with Kalman filter methods. Besides, it is also shown that the Kalman filter can be derived within a Bayesian framework and reduces to a maximum posterior probability (MAP) solution, and can be easily extended to ML solution \citep{haykin2001kalman, guzzi2016data}. 

Consider the following model 
\begin{align}\label{introKFmodel}
\begin{array}{cccccccccc}
\cdots &\to &x_{t-1}&\to &x_{t}&\to &x_{t+1}&\to &\cdots &{\text{truth}}\\
\cdots &&\downarrow &&\downarrow &&\downarrow && \cdots &\\ \cdots&&y_{t-1}&&y_{t}&&y_{t+1}&&\cdots &{\text{observation}}\end{array}
 \end{align}
in which $x_t=F(x_{t-1})+\varepsilon_x$ is the true hidden state propagating through the transition matrix $F$ and $y_t=G(x_t)+\varepsilon_y$ is the observation measured by the measurement matrix $G$ of the system, where $\varepsilon_x$ and $\varepsilon_y$ can be viewed as white noise random sequences with unknown statistics in the discrete-time domain. 

To estimate the filtering state $x_t$ from $y_{1:t}=\left\lbrace y_1,\ldots,y_t\right\rbrace$, Bayesian wants to maximize the posterior $p(x_t\mid y_{1:t})$ by marginalizing out all the previous measurements. Given the joint distribution of $p(x_t,x_{t-1},y_{1:t})$, Kalman filter supposes the expectation $\hat{x}_{t-1}$ and its variance $S_{t-1}$ are known and passing through the system by $\hat{x}_t=F\hat{x}_{t-1}$ and $S_t=FS_{t-1}F^\top + Q_t$, here $Q_t$ is the covariance of $\varepsilon_x$. Because of the log-likelihood function is written in such way: 
$\ln p(x_t\mid y_{1:t}) \propto -\frac{1}{2}(y_t-Gx_t)^\top R_t^{-1}(y_t-Gx_t)-\frac{1}{2}(x_t-\hat{x}_{t-1})^\top S_{t-1}^{-1}(x_t-\hat{x}_{t-1})$, and $R_t$ is the covariance of $\varepsilon_y$. As a result, the solution is 
\begin{equation*}
\hat{x}_t = \left(G^\top R_t^{-1}G+S_{t-1}^{-1}\right)^{-1}\left( G^\top R_t^{-1}y_t+S_{t-1}^{-1}\hat{x}_{t-1} \right).
\end{equation*}
Additionally, by setting $S_t^{-1} = G^\top R_t^{-1}G+S_{t-1}^{-1}$, the recursive estimation of covariance matrix is $S_t = S_{t-1} - K_t GS_{t-1}$, 
and $K_t = S_{t-1} G^\top (R_t +GS_{t-1}G^\top)^{-1}$ is named Kalman gain matrix. Consequently, the recursive estimation is 
\begin{equation}\label{KalmanEstimation}
\hat{x}_t = \hat{x}_{t-1}+K_t(y_t-G\hat{x}_{t-1}).
\end{equation}

Further, compared with the filtering distribution $p(x_t\mid y_{1:t})$, the prediction distribution is trying to find an $n$-steps later distribution $p(x_{t+n}\mid y_{1:t})$ from the current state, and the smoothing distribution is to find the distribution $p(x_k\mid y_{1:t})$ of a specific state $x_k$ in the past for any $k$, where $1<k<t$. 

However, Kalman filter has a limitation that it does not apply to general non-linear model and non-Gaussian distributions. For a non-linear system, one can use the extended Kalman filter (EKF), which is widely used for solving nonlinear state estimation applications \citep{gelb1974applied, bar1993estimation}. The EKF uses Taylor expansion to construct linear approximations of nonlinear functions, therefore the state transition $f$ and observation $g$ do not have to be linear but to be differentiable. However, in the EKF process, these approximations can incur large errors in the true posterior mean and covariance of the transformed random variable \citep{wan2000unscented}. 

Alternatively, the unscented Kalman filter (UKF) is a derivative-free method \citep{julier1997new, wan2000unscented, gyorgy2014unscented}. It uses the Kalman filter to create a normal distribution that approximates the result of a non-linear transformation numerically by seeing what happens to a few deliberately chosen points. The unscented transform is used to recursively estimate the equation \eqref{KalmanEstimation}, where the state random variable is re-defined as the concatenation of the original state and noise variables. By contrast, Kalman filter does not require numerical approximations. 

The performances of EKF and UKF are compared in a few references regarding to different kinds of aspects, such as \citep{chandrasekar2007comparison, laviola2003comparison, st2004comparison}. There is not an overall conclusion that which one performs better. 


Limited to its property, Kalman filter is tied up for a dynamic system, where the parameters and noise variances are unknown. In some dynamic systems, the variances are obtained based on the system identification algorithm, correlation method, and least squares fusion criterion. To solve this issue, a self-tuning weighted measurement fusion Kalman filter is proposed by \cite{ran2010self}. Likewise, a new adaptive Kalman filter will be another choice \citep{oussalah2001adaptive}. 


However, when the target maneuver occurs, Kalman filtering accuracy will be reduced or even diverged due to the model mismatch and noise characteristics that cannot be known exactly \citep{liu2014filtering}. Additionally, Kalman filter based methods require the state vector contains pre-specified coefficients during the whole approximation procedure and are within the bounded definition range determined at the beginning \citep{jauch2017recursive}. 

A more generic algorithm is investigated in the following section. 

\subsection*{Monte Carlo Filter}

\textit{Monte Carlo filter} is a class of Monte Carlo approaches \citep{chen2003bayesian}. The power of these approaches is that they can numerically and efficiently handle integration and optimization problems. 

The important advantage of Monte Carlo is that a large number of posterior moments can be estimated at a reasonable computational effort and that estimates of the numerical accuracy of these results are obtained in a simple way \citep{kloek1978bayesian}. \textit{Sequential Monte Carlo} method uses Monte Carlo approaches to estimate and to compute recursively. One of the attractive merits is in the fact that they allow on-line estimation by combining the powerful Monte Carlo sampling methods with Bayesian inference at an expense of reasonable computational cost \citep{chen2003bayesian}. 

For example, consider the model \eqref{introKFmodel} with parameter $\theta$. The likelihood approximation is $p(y_t\mid y_{1:t-1},\theta)$ and can be written by
\begin{equation*}
p(y_t\mid y_{1:t-1},\theta) = \int p(y_t\mid x_t,\theta)p(x_t\mid y_{1:t-1},\theta) dx_t = \E \lbrack y_t\mid x_t,\theta \rbrack.
\end{equation*}
The standard Monte Carlo algorithm is trying to compute the integration by drawing $N$ independent samples $x_t^{(i)}$ from $p(x_t\mid y_{1:t-1},\theta)$ first and then, by adding them up, to approximate the integration for large $N$ in the following way 
\begin{equation*}
\E\left[ y_t\mid x_t,\theta \right]\approx \frac{1}{N}\sum_i p(y_t\mid x_t^{(i)},\theta), 
\end{equation*}
\citep{kalos2008monte}.

In terms of getting good samples of $x_t^{(i)}$, which can be used for representing $p(y_t\mid x_t^{(i)},\theta)$, an importance sampling method was devised. The idea of this method is by assigning weights $W_t^{(i)}$ to samples, the most important ones are evaluated for computing the integral. Further, sequential importance sampling (SIS) allows a sequential update of the importance weights by 
\begin{equation*}
W_t^{(i)} \propto W_{t-1}^{(i)} \frac{ p\left(y_t \mid x_t^{(i)}\right) p\left(x_{t}^{(i)}\mid x_{t-1}^{(i)}\right) }{q\left(x_{t}^{(i)}\mid x_{t-1}^{(i)},y_{t}\right)}
\end{equation*}
with an appropriate chosen \textit{proposal distribution} $q(x_{t}\mid  x_{t-1},y_{t})$. It is also called \textit{importance density} or \textit{important function}  \citep{chen2003bayesian}. 


Nevertheless, the SIS makes samples skewed that only a few samples have proper weights as time increases and most of them have small but positive weights. This phenomenon is often called \textit{weight degeneracy} or \textit{sample impoverishment} \citep{green1995reversible, berzuini1997dynamic}. 


Besides the SIS processes, a resampling step, also known as a selection step, is trying to eliminate the samples with small weights and duplicate the samples with large weights in a principled way \citep{rubin2004multiple, tanner1987calculation}. This method is named \textit{sampling and importance resampling} (SIR). Suppose samples with associated weights are $\left\lbrace x_t^{(i)},W_t^{(i)}\right\rbrace$, a resampling step is executed by generating new samples $\tilde{x}_t^{(i)}$ according to normalized weights $\tilde{W}_t^{(i)}$. It is pointed out that the resampling step does not prevent weights degeneracy but improve further calculation. 

The common feature of SIS and SIR is that both of these methods are based on importance sampling and updating samples weights recursively. The difference is that in SIR, the resampling step is always performed. Whereas, in SIS, the resampling is only taken when needed. 


\textit{Particle filter} (PF) is the most successful application of importance sampling with resampling algorithm. It randomly generates a cloud of points and push these points through the computation process. It is a recursive implementation of the Monte Carlo approaches \citep{doucet2009tutorial}. 


A generic PF generates $N$ uniformly weighted random measurements $\left\lbrace x_{t-1}^{(i)},\frac{1}{N} \right\rbrace$ first at time $t-1$. Once a new observation $y_t$ comes into the system, the weights will be updated recursively by involving the likelihood function $p(y_t\mid x_t^{(i)})$ and propagation function $p(x_t^{(i)}\mid x_{t-1}^{(i)})$. In fact, it is the SIS step. To monitor how bad is the weight degeneracy, a suggested measurement \textit{effective sample size} is introduced in \citep{kong1994sequential}. It is the reciprocal of the sum of squared weights in the form of 
\begin{equation*}
N_{\mbox{\scriptsize eff}} = \frac{1}{\sum_{i=1}^{N}\left(W_t^{(i)} \right)^2}. 
\end{equation*}
If the $N_{\mbox{\scriptsize eff}}$ is less than a predefined threshold, the resampling procedure is executed and the set of particles remains the same size $N$. 



However, the PF sampling and resampling methods may cause practical problems. Such as high weighted particles have been selected many times and lead to the loss of diversity. This problem is known as sample impoverishment, in which way the particles are not representative. The improvements of particle filter's performance have been devoted by \citep{carpenter1999improved, godsill2001maximum, stavropoulos2001improved, smcmip2011}. 


Apparently, Bayesian filtering has become a broad topic involving many scientific areas that a comprehensive survey and detailed treatment seems crucial to cater the ever growing demands of understanding this important field for many novices, though it is noticed by the author that in the literature there exist a number of excellent tutorial papers on particle filters and Monte Carlo filters \citep{chen2003bayesian, doucet2000sequential, chen2012monte, doucet2000rao}.
 



\section{Markov Chain Monte Carlo Methods}


Schemes exist to counteract sample impoverishment, which incurs in particle filter \citep{ristic2004beyond}. One approach is to consider the states for the particles to be predetermined by the forward filter and then to obtain the smoothed estimates by recalculating the particles' weights via a recursion from the final to the first time step \citep{godsill2000methodology}. Another approach is to use a \textit{Markov chain Monte Carlo} (MCMC) move step \citep{carlin1992monte}. MCMC refers to constructing Markov chains that move in the unobserved quantity space and produce a sequence samples from the posterior distribution. After the chain has been run long enough, the sequence is considered as an approximation to the posterior distribution \citep{kokkala2016particle}. 


MCMC methods are a set of powerful stochastic algorithms that allow us to solve most of these Bayesian computational problems when the data are available in batches \citep{andrieu1999sequential, green1995reversible, andrieu2001model}. They are based on sampling from probability distributions based on a Markov chain. If samples are unable to be drawn directly from a distribution $\pi(x)$, we can construct a Markov chain of samples from another distribution $\hat{\pi}(x)$ that is equilibrium to $\pi(x)$. If the chain is long enough, these samples of the chain can be used as a basis for summarizing features of $\pi(x)$ of interest \citep{smith1993bayesian}. This is a crucial property. See \eg \cite{cappe2009inference, liu2008monte} for details. 




\subsection*{Metropolis-Hastings Algorithm}

The \textit{Metropolis-Hastings} (MH) algorithm is an important class of MCMC algorithms \citep{smith1993bayesian, tierney1994markov, gilks1995markov}. Given essentially a probability distribution $\pi(\cdot)$ (the target distribution), MH algorithm provides a way to generate a Markov chain $x_1, x_2,\ldots, x_t$, who has the target distribution as a stationary distribution, for the uncertain parameters $x$ requiring only that this density can be calculated at $x$. Suppose that we can evaluate $\pi(x)$ for any $x$. The transition probabilities should satisfy the detailed balance condition
\begin{equation*}
\pi\left(x^{(t)}\right)q\left(x', x^{(t)}\right) = \pi\left(x'\right)q\left(x^{(t)}, x'\right),
\end{equation*}
which means that the transition from the current state $\pi\left(x^{\left(t\right)}\right)$ to the new state $\pi\left(x'\right)$ has the same probability as that 
from $\pi\left(x'\right)$ to $\pi\left(x^{\left(t\right)}\right)$. In sampling method, drawing $x_i$ first and then drawing $x_j$ should have the same probability as drawing $x_j$ and then drawing $x_i$. However, in most situations, the details balance condition is not satisfied. Therefore, a function $\alpha\left(x,y\right)$ is introduced for satisfying 
\begin{equation*}
\pi\left(x'\right)q\left(x', x^{\left(t\right)}\right)\alpha\left(x',x^{\left(t\right)}\right) = \pi\left(x^{\left(t\right)}\right)q\left(x^{\left(t\right)}, x'\right)\alpha\left(x^{\left(t\right)},x'\right).
\end{equation*}
In this way, a tentative new state $x'$ is generated from the proposal density $q\left(x';x^{\left(t\right)}\right)$ and it is then accepted or rejected according to acceptance probability 
\begin{equation}\label{IntroAccp}
\alpha=\frac{\pi\left(x'\right)}{\pi\left(x^{\left(t\right)}\right)}\frac{q\left(x^{\left(t\right)}, x'\right)}{q\left(x', x^{\left(t\right)}\right)}.
\end{equation}
If $\alpha \geq 1$, then the new state is accepted. Otherwise, the new state is accepted with probability $\alpha$.

A simple mechanic proposing algorithm is \textit{random walk Metropolis-Hastings} (RM MH). It is easy to implement and symmetric under the exchange of the initial and proposed points. 

Besides, modified Metropolis-Hastings algorithms, such as the delayed-rejection MH, multiple-try MH and reversible-jump MH algorithms have been studied by \cite{tierney1999some, liu2000multiple, green1995reversible}. 
 

%\subsection*{Gibbs Sampling Algorithm}
%
%\textit{Gibbs sampling}, also known as the heat bath method or ``Glauber dynamics'', was proposed by \citep{geman1984stochastic}. It has been proved particularly convenient for a range of applications in Bayesian statistics \citep{smith1993bayesian, mackay2003information}. Gibbs sampler is generating posterior samples conditionally on remaining variables fixed to their current values from a distribution that has at least two dimensions. For example, consider a general case that contains $N$ variables $x_t^{(1)},\ldots,x_t^{(N)}$ at time $t$. The samples for the next state $t+1$ will be 
%\begin{align*}
%x_{t+1}^{(1)}  &\sim \pi(x_1 \mid x_t^{(2)},x_t^{(3)},\ldots,x_t^{(N)}) \\
%x_{t+1}^{(2)}  &\sim \pi(x_2 \mid x_{t+1}^{(1)},x_t^{(3)},\ldots,x_t^{(N)}) \\
%&\vdots \\
%x_{t+1}^{(N)}  &\sim \pi(x_N \mid x_{t+1}^{(1)},x_{t+1}^{(2)},\ldots,x_{t+1}^{(N-1)})
%\end{align*}
%using the newly updated sample $x_{t+1}^{(i)}$. After one iteration, the new sequence $x_{t+1}^{(1)},\ldots,x_{t+1}^{(N)}$ is obtained. 
%
%Gibbs sampling can be viewed as one of Metropolis methods and meets the convergence property that the probability distribution of $x_t$ tends to $\pi(x)$ as $t\to\infty$, as long as $\pi(x)$ does not have pathological properties. Similarly to MH algorithms, some modified Gibbs sampler, such as single-site Gibbs sampling algorithm \citep{sorensen2007likelihood}, a blocked Gibbs sampling algorithm \citep{garcia1996multivariate} and hybrid Gibbs sampling algorithm \citep{mathew2012bayesian}, are well developed and applied in different applications \citep{gelfand1990illustration}. 



\subsection*{Adaptive MCMC Algorithm}


Metropolis-Hastings algorithm is widely used in statistical inference, to sample from complicated high-dimensional distributions. Typically, this algorithm has parameters that must be tuned in each new situation to obtain reasonable mixing times, such as the step size in a random walk Metropolis \citep{mahendran2012adaptive}. Tuning of associated parameters such as proposal variances is crucial to achieving efficient mixing, but can also be difficult. 

%It is introduced that Metropolis-Hastings algorithm draws samples from a target distribution $\pi(\cdot)$ by proposing $x'$ from $x^{(t)}$. Then calculate a probability to decide whether accept or reject it. The proposal distribution 

\textit{Adaptive MCMC} methods have been developed to automatically adjust these parameters, such as \citep{andrieu2008tutorial, girolami2011riemann,  atchade2009adaptive, roberts2009examples}. One of the most successful adaptive MCMC algorithms is introduced by \cite{haario2001adaptive}, where, based on the  random walk Metropolis algorithm, the covariance of the proposal distribution is calculated using all of the previous states. For instance, with an adaptive MCMC chain $x_0,x_1,\ldots,x_t$, the proposal $x'$ is from $N(\cdot\mid x_t,R_t)$, where $R_t$ is the covariance matrix determined by the spatial distribution of the state $x_0,x_1,\ldots,x_t$. 

Even though the adaptive Metropolis algorithm is non-Markovian, the establishment was verified that the adaptive MCMC algorithm indeed has the correct ergodic properties. 

A Bayesian optimization for adaptive MCMC was proposed by \cite{mahendran2012adaptive}. The author proposed adaptive strategy consists of two phases: adaptation and sampling. In the first phase, Bayesian optimization is used to construct a randomized policy. After that, in the second phase,  a mixture of MCMC kernels selected according to the learned randomized policy is used to explore the target distribution. 

Further investigation in the use of adaptive MCMC algorithms to automatically
tune the Markov chain parameters can be found at \citep{roberts2009examples}. 


%
%\subsection*{Hamiltonian Monte Carlo}
%
%The \textit{Hamiltonian Monte Carlo} (HMC), devised by \citep{duane1987hybrid} as hybrid Monte Carlo, is using Hamiltonian dynamics to produce distant proposals for the Metropolis algorithm in order to avoid slow exploration of the state-space that results from the diffusive behavior of simple random walk proposals \citep{neal2011mcmc}. In practice, the HMC sampler is more efficient for sampling in high-dimensional distributions than MH.
%
%
%The key feature of HMC is the Hamiltonian system equation as follows: 
%\begin{equation*}
%H(x,v) = U(x)+K(v),
%\end{equation*}
%which is consisting of potential energy $U(x)$ with a d-dimensional momentum vector (position) $x$ and kinetic energy function $K(v)=\frac{v^\top M^{-1}v}{2}$ with a d-dimensional momentum vector (velocity) $v$. Here, $M$ is a symmetric positive-definite ``mass matrix''. The momentum parameter $v$ is responsible for keeping proposals inside the typical set, instead than having them drifting towards the tails or towards the mode. 
%
%
%To propose $\left\lbrace x',v'\right\rbrace$, HMC is using leapfrog method, which is based on Euler's method and modified Euler's method, to increase the proposing accuracy \citep{betancourt2017conceptual}. It is accepted with the probability
%\begin{equation*}
%\alpha = \min\lbrace  \exp\left(H(x,v)-H(x',v')\right), 1\rbrace.
%\end{equation*}
%
%
%Compared with MH and Gibbs sampling, the HMC has a higher efficiency in most of the high-dimensional cases. It is incorporating not only with energy $U(x)$ but also with a gradient. In this way, HMC explores a larger area and converges to balance faster. 
%
%However, HMC has some problems with sampling from distributions with isolated local minimums and has little energy jumping out of a local minimal cave. 
%
%
%
%\subsection*{Zig-Zag Monte Carlo}
%
%The zig-zag process is a continuous-time piecewise deterministic-stochastic process introduced in \citep{bierkens2016piecewise}. It is an application of the Curie-Weiss model, \citep{turitsyn2011irreversible}, in high dimension and provides a practically efficient sampling scheme for sampling in the big data regime with some remarkable properties \citep{bierkens2017limit}. 
%
%
%Given a target density $\pi(\cdot)$, the Zig-Zag process $f(x,\theta)$ is defined in a $d \otimes 2$ space $E$. $x$ is in the $d$-dimensional topological subspace and $\theta$ is in a binary discrete $\left\lbrace -1,1\right\rbrace_d$ subspace denoting the flipping statues. The switching rate $\lambda(x,\theta)$ agrees with the target distribution $\pi(\cdot)$ in a certain way and is defined as $\lambda(x,\theta) = \max \lbrace 0,\theta U'(x) \rbrace +\gamma(x)$, where $U'(x)  = \lambda(x,\theta) - \lambda(x,-\theta)$. Then, the Zig-Zag operator $L$ is 
%\begin{equation*}
%Lf(x,\theta)=\theta\partial f_x+\lambda(x,\theta)\left(f(x,-\theta)-f(x,\theta )\right),
%\end{equation*}
%for all $(x,\theta)\in E$. 
%
%Thereafter, the obtained sequence of the Zig-Zag process is used to approximate expectations with respect to $\pi(\cdot)$ according to the law of large numbers. 
%
%
%To check the performance of Zig-Zag sampler and comparing with other MCMC sampler, the effective sample size (ESS) is given as $ESS=\frac{\Var_\pi f}{\Var(\pi_t(f))}=\frac{\Var_\pi f}{\sigma^2N_s}N(t)$, $N(t)$ is the number of switches up to time $t$. It is numerically tested that the Zig-Zag sampler has a higher ESS than a tuned Random-Walk Metropolis-Hastings \citep{bierkens2017limit}. 
%
%The application of Zig-Zag process in big data scheme and some properties are given in \citep{bierkens2017limit} and \citep{bierkens2016zig}.
%
%
%
%\subsection*{The T-Walk}
%
%A general purpose sampling algorithm for continuous distributions, named as \textit{the t-walk}, is given by \citep{christen2010general}. It is a self-adjusting MCMC
%algorithm that requires no tuning and has been shown to provide good results in many cases of up to 400 dimensions. Further, because of the t-walk is not adaptive, then it does not require new restricting conditions but only the log of the posterior and two initial points \citep{blaauw2011flexible}. 
%
%
%Given a posterior distribution $\pi(\cdot)$, the new objective function $f(x,x')$ is the product of $\pi(x)\pi(x')$ from $X \otimes X$. The new proposal $(y,y')$ is given by
%\begin{equation*} (y,y') =
%\begin{cases}
%(x,h(x',x)) & \mbox{with probability 0.5}\\
%(h(x',x),x') & \mbox{with probability 0.5}
%\end{cases}
%\end{equation*}
%where $h(x,x')$ is a preselected proposing strategy. In each iteration, only one of the two chains $x$ and $x'$ moves according to a random walk. For example, suppose in the first step the $x$ stays the same but $y'$ is proposed from $q(\cdot\mid x,x')$, then the acceptance ratio is 
%\begin{equation*}
%\frac{\pi(y')}{\pi(x')}\frac{q(x'\mid y',x) }{q(y'\mid x',x)}. 
%\end{equation*}
%After a few iterations, there are two dual and coupled chains obtained. Hence, the t-walk is a kind of multiple chain approach. 
%
%Four recommendations for the choices of $h(x,x')$ including a scaled random walk, referred to \textit{the walk move}, \textit{traverse move}, \textit{hop moves} and \textit{blow moves} are given in \citep{christen2010general}. 
%
%The t-walk is now available in a complete set of computer packages, including \textit{R}, \textit{Python}. It is convenient for researchers to go a deeper implementation. 


% % % % % % % % % % % % % %

\subsection*{Other Monte Carlo Algorithms}

The \textit{Hamiltonian Monte Carlo} (HMC), devised by \cite{duane1987hybrid} as hybrid Monte Carlo, uses Hamiltonian dynamics to produce distant proposals for the Metropolis algorithm in order to avoid slow exploration of the state-space that results from the diffusive behavior of simple random walk proposals \citep{neal2011mcmc}. In practice, the HMC sampler is more efficient for sampling in high-dimensional distributions than MH.

The key feature of HMC is the Hamiltonian system equation as follows: 
\begin{equation*}
H(x,v) = U(x)+K(v),
\end{equation*}
which is consisting of potential energy $U(x)$ with a $d$-dimensional momentum vector (position) $x$ and kinetic energy function $K(v)=\frac{v^\top M^{-1}v}{2}$ with a $d$-dimensional momentum vector (velocity) $v$. To propose $\left\lbrace x',v'\right\rbrace$, HMC uses the leapfrog method, which is based on Euler's method and modified Euler's method, to increase the proposing accuracy \citep{betancourt2017conceptual}. It is accepted with the probability
\begin{equation*}
\alpha = \min\left\lbrace\exp\lbrack H(x,v)-H(x',v') \rbrack, 1\right\rbrace.
\end{equation*}

Compared with MH sampler, the HMC has a higher efficiency in most of the high-dimensional cases. It is incorporating not only with energy $U(x)$ but also with a gradient. In this way, HMC explores a larger area and converges to balance faster. 


The \textit{Zig-Zag Monte Carlo} uses a continuous-time piecewise zig-zag process to increase the sampling efficiency \citep{bierkens2016piecewise}. It is an application of the Curie-Weiss model in high dimension and provides a practically efficient sampling scheme for sampling in the big data regime with some remarkable properties \citep{turitsyn2011irreversible, bierkens2017limit}. 

Given a target density $\pi(\cdot)$, the Zig-Zag process $f(x,\theta)$ is defined in a $d \otimes 2$ space $\mathcal{E}$. $x$ is in the $d$-dimensional topological subspace and $\theta$ is in a binary discrete $\left\lbrace -1,1\right\rbrace_d$ subspace denoting the flipping statues. The switching rate $\lambda(x,\theta)$ agrees with the target distribution $\pi(\cdot)$ in a certain way and is defined as $\lambda(x,\theta) = \max \left\lbrace 0,\theta U'(x) \right\rbrace +\gamma(x)$, where $U'(x)  = \lambda(x,\theta) - \lambda(x,-\theta)$. Then, the Zig-Zag operator $L$ is 
\begin{equation*}
Lf(x,\theta)=\theta\partial f_x+\lambda(x,\theta)\lbrack f(x,-\theta)-f(x,\theta )\rbrack,
\end{equation*}
for all $(x,\theta)\in \mathcal{E}$. 

Thereafter, the obtained sequence of the Zig-Zag process is used to approximate expectations with respect to $\pi(\cdot)$ according to the law of large numbers. 
%To check the performance of Zig-Zag sampler and comparing with other MCMC sampler, the effective sample size (ESS) is given as $ESS=\frac{\Var_\pi f}{\Var(\pi_t(f))}=\frac{\Var_\pi f}{\sigma^2N_s}N(t)$, $N(t)$ is the number of switches up to time $t$. It is numerically tested that the Zig-Zag sampler has a higher ESS than a tuned Random-Walk Metropolis-Hastings \citep{bierkens2017limit}. 
The application of Zig-Zag process in big data scheme and some properties are given in \citep{bierkens2017limit}. %\citep{bierkens2016zig}.




\textit{The t-walk} given by \cite{christen2010general} is a self-adjusting MCMC algorithm that requires no tuning and has been shown to provide good results in many cases of up to 400 dimensions. Because of the t-walk is not adaptive, it does not require new restricting conditions but only the log of the posterior and two initial points \citep{blaauw2011flexible}. 

Given a posterior distribution $\pi(\cdot)$, the new objective function $f(x,x')$ is the product of $\pi(x)\pi(x')$ from $X \otimes X$. The new proposal $(y,y')$ is given by
\begin{equation*} (y,y') =
\begin{cases}
\left(x,h(x',x)\right) & \mbox{with probability 0.5}\\
\left(h(x',x),x'\right) & \mbox{with probability 0.5}
\end{cases}
\end{equation*}
where $h(x,x')$ is a preselected proposing strategy. In each iteration, only one of the two chains $x$ and $x'$ moves according to a random walk. For example, suppose in the first step the $x$ stays the same but $y'$ is proposed from $q(\cdot\mid x,x')$, then the acceptance ratio is 
\begin{equation*}
\frac{\pi\left(y'\right)}{\pi\left(x'\right)}\frac{q\left(x'\mid y',x\right) }{q\left(y'\mid x',x\right)}. 
\end{equation*}
After a few iterations, there are two dual and coupled chains obtained. Hence, the t-walk is a kind of multiple chain approach. 

Four recommendations for the choices of $h\left(x,x'\right)$ including a scaled random walk, referred to \textit{the walk move}, \textit{traverse move}, \textit{hop moves} and \textit{blow moves} are given in \citep{christen2010general}. The t-walk is now available in a complete set of computer packages, including \textit{R}, \textit{Python}. It is convenient for researchers to go a deeper implementation. 



\section{Thesis Outline}

In Chapter \ref{ChapterTS}, an adaptive smoothing V-spline method, which is based on \textit{Hermite spline} basis functions, is proposed to obtain a reconstruction of $f$ and $f'$ from noisy data $y_{1:n}$ and $v_{1:n}$. Instead of minimizing the residuals of $f(t_i)-y_i$ only, the residuals of $f'(t_i)-v_i$ with a new parameter $\gamma$ are consisted in the new objective function. A modified leave-one-out cross-validation algorithm is used for find the optimal parameters. Numerical simulation and real data implementation are given after theoretical methodology. 

In Chapter \ref{ChapterGPR}, the Bayesian estimation form of the V-spline is given. It is proved that the Bayesian estimate is corresponding to a trivial V-spline  in the reproducing kernel Hilbert space $\mathcal{C}_{\mbox{\scriptsize p.w.}}^{(2)}[0,1]$, where the second-derivative is piecewise-continuous.  An extended GCV is used to find the optimal parameters for the Bayesian estimate. 

In Chapter \ref{ChapterFR}, a comprehensive overview of existing methods for sequential state and parameter inference is given. Basic concepts and popular algorithms of sequential state estimation are discussed in the second section. Furthermore, the algorithms for combined state and parameter estimation are brought into a separate section. A numerical comparison of different methods is given at the end of this chapter. 

In Chapter \ref{ChapterMCMC}, a random walk Metropolis-Hastings algorithm in the learning phase is utilized to learn the mean and the covariance of the parameters space. After that, the information is implemented in the estimation phase, where an adaptive Delayed-Acceptance Metropolis-Hastings algorithm is proposed for estimating the posterior distribution of combined state and parameter. To remain a high running efficiency, a sliding window approach, in which way historical data is cut off when new observations come into the data stream, is used to improve the sampling speed. This algorithm is applied to irregularly sampled time series data and implemented in real GPS data set. 

The proof of theorems, details of equation calculations, and results of simulation studies are all presented in appendices. For details, the Appendix \ref{appendTS} includes V-spline related theorems, lemmas, calculations and figures. In Appendix \ref{appendMCMC}, the proposed adaptive sequential MCMC related works and outcomes are illustrated, including details of the recursive form calculation, tables and figures of parameters comparison and so on. 

A spin-off outcome is presented in Appendix \ref{appendSimp}. It is a data simplification method used for reducing the size of a dataset and saving storage costs without losing important information.







\clearemptydoublepage

\chapter{Adaptive Smoothing V-Spline for Trajectory Reconstruction}\label{ChapterTS}


\section{Introduction}

GPS devices are widely used for tracking individuals and vehicles. The position and velocity of moving objects are determined by GPS units and can be used in batch and on-line estimation of trajectories. 

The use of GPS receivers for obtaining trajectory information is carried out for a wide variety of reasons. The \textit{TracMap} company, located in New Zealand and USA, produces GPS display units to aid precision farming in agriculture, horticulture and viticulture. With these units, operational data is collected and sent to a remote server for further analysis. GPS units also guide drivers of farm vehicles to locations on the farm that require specific attention and can indicate the location of potential hazards. 

Given a sequence of position vectors in a tracking system, the simplest way of constructing the complete trajectory of a moving object is by connecting positions with a sequence of lines, known as line-based trajectory representation \citep{agarwal2003indexing}. Vehicles with an omnidirectional drive or a differential drive can actually follow such a path in a drive-and-turn fashion, though it is highly inefficient \citep{gloderer2010spline} and this kind of non-smooth motion can cause slippage and over-actuation \citep{magid2006spline}. By contrast, most vehicles typically follow smooth trajectories without sharp turns. 

Several methods have been investigated to solve this issue. One of them uses the minimal length path that is continuously differentiable and consists of line segments or arcs of circles, with no more than three segments or arcs between successive positions \citep{dubins1957curves}. This method is called Dubins curve and has been extended to other more complex vehicle models but is still limited to line segments and arcs of circles \citep{yang2010analytical}. However, there are still curvature discontinuities at the junctions between lines and arcs, leading to yaw angle errors \citep{wang2017curvature}. 

Spline methods have been developed to overcome these issues and to construct smooth trajectories.  \cite{magid2006spline} propose a path-planning algorithm based on splines. The main objective of the method is the smoothness of the path, not a shortest or minimum-time path. A curve-based method uses a parametric cubic function $P(t)=a_0+a_1t+a_2t^2+a_3t^3$ to obtain a spline that passes through any given sequence of joint position-velocity paired points $(y_1, v_1), (y_2, v_2), \ldots, (y_n,v_n)$ \citep{yu2004curve}. More generally, a B-spline gives a closed-form expression for the trajectory with continuous second derivatives and goes through the points smoothly while ignoring outliers \citep{komoriya1989trajectory, ben2004geometric}. It is flexible and has minimal support with respect to a given degree, smoothness, and domain partition. \cite{gasparetto2007new} use fifth-order B-splines to compose the overall trajectory, allowing one to set kinematic constraints on the motion, expressed as the velocity, acceleration, and jerk. In computer (or computerized) numerical control (CNC), \cite{erkorkmaz2001high} presented a quintic spline trajectory generation algorithm connecting a series of reference knots that produces continuous position, velocity, and acceleration profiles. \cite{yang2010analytical} proposed an efficient and analytical continuous curvature path-smoothing algorithm based on parametric cubic B\'{e}zier curves. Their method can fit ordered sequential points smoothly. 


However, a parametric approach only captures features contained in the preconceived class of functions \citep{yao2005functional} and increases model bias. To avoid this, nonparametric methods have been developed. Rather than giving specified parameters, it is desired to reconstruct the trajectory $f(t)$ from the data $y(t_i)\equiv y_i$, $i=1, \ldots, n$, \citep{craven1978smoothing}. Smoothing spline estimates of $f(t)$ appear as a solution to the following minimization problem: find $\hat{f} \in \mathcal{C}^{(2)}[a,b]$ that minimizes the penalized residual sum of squares:
\begin{equation}\label{smoothingob}
\mbox{RSS}=\sum_{i=1}^{n}\left(  y_i-f(t_i)\right)^2+\lambda\int_{a}^{b} \left(f''(t)\right)^2dt
\end{equation}
for a pre-specified value $\lambda>0$ \citep{aydin2012smoothing}. In equation  \eqref{smoothingob}, the first term is a residual sum of squares and penalizes lack of fit. The second term is a roughness penalty term weighted by a smoothing parameter $\lambda$, which varies from 0 to $+\infty$ and establishes a trade-off between interpolation and a straight line. The roughness penalty term is a formalization of a mechanical device: if a thin piece of flexible wood, called a spline, is bent to the shape of the curve $f$, then the leading term in the strain energy is proportional to $\int f''^2dt$ \citep{green1993nonparametric}. The reconstruction cost, equation \eqref{smoothingob}, is determined not only by its goodness-of-fit to the data quantified by the residual sum of squares but also by its roughness \citep{schwarz2012geodesy}. For a given $\lambda$, minimizing equation \eqref{smoothingob} will give the best compromise between smoothness and goodness-of-fit. Notice that the first term in equation \eqref{smoothingob} depends only on the values of $f$ at $t_i, i=1, \ldots, n$. \cite{green1993nonparametric} show that the function that minimizes the objective function for fixed values of $f(t_i)$ is a cubic spline: an interpolation of points via a continuous piecewise cubic function, with continuous first and second derivatives. The continuity requirements uniquely determine the interpolating spline, except at the boundaries \citep{sealfon2005smoothing}.


\cite{zhang2013cubic} propose Hermite interpolation on each interval to fit position, velocity and acceleration with kinematic constraints. Their trajectory formulation is a combination of several cubic splines on every interval or, alternatively, is a single function found by minimizing 
\begin{equation}
\lambda\sum_{i=1}^{n}\lvert y_i-f(t_i) \rvert^2+(1-\lambda)\int \lvert f''(t) \rvert^2dt,
\end{equation}
where $p$ is a smoothing parameter \citep{castro2006geometric}. 


A conventional smoothing spline is controlled by one single parameter, which controls the smoothness of the spline on the whole domain. A natural extension is to allow the smoothing parameter to vary as a function of the independent variable, adapting to the change of roughness in different domains \citep{silverman1985some, donoho1995wavelet}. The objective function is now of the form 
\begin{equation}\label{objective}
\sum_{i=1}^{n}\left(y_i-f(t_i) \right)^2+\int\lambda(t) \left( f''(t)\right)^2 dt.
\end{equation}


Similar to the conventional smoothing spline problem, one has to choose the penalty function $\lambda(t)$. The fundamental idea of nonparametric smoothing is to let the data choose the amount of smoothness, which consequently decides the model complexity \citep{gu1998model}. When $\lambda$ is constant, most methods focus on data-driven criteria, such as cross-validation (CV), generalized cross-validation (GCV) \citep{craven1978smoothing} and generalized maximum likelihood (GML) \citep{wahba1985comparison}. Allowing the smoothing parameter to be a function poses additional challenges, though \cite{liu2010data} were able to extend GML to adaptive smoothing splines.


In this chapter, an adaptive smoothing spline named the V-spline is proposed that uses modified Hermite spline basis functions to obtain a reconstruction of $f$ and $f'$ from noisy paired position data $\mathbf{y}=\left\lbrace y_1,\ldots,y_n\right\rbrace$ and velocity data $\mathbf{v}=\left\lbrace v_1,\ldots,v_n\right\rbrace$. Rather than just using residuals of $f(t_i)-y_i$, the extra residuals of $f'(t_i)-v_i$ and a new parameter $\gamma$ are included in the objective function. The parameter $\gamma$ controls the degree to which the velocity information is used in the reconstruction. In this way, the V-spline keeps a balance between position and velocity. An advanced cross-validation formula is given for the V-spline parameters. It is shown that the new spline performs well on simulated signal data \textit{Blocks}, \textit{Bumps}, \textit{HeaviSine} and \textit{Doppler} \citep{donoho1994ideal}. Finally, an application of the V-spline to a set of 2-dimensional real data is given. 


\section{V-Spline}\label{SectionTractorSpline}

In the nonparametric regression, consider $n$ paired time series points $\left\lbrace t_1,y_1,v_1\right\rbrace$, $\ldots$, $\left\lbrace t_n,y_n,v_n\right\rbrace$, such that $a \leq t_1<t_2< \cdots < t_n \leq b$, $y$ is the position information and $v$ indicates its velocity. As in \citep{silverman1985some} and \citep{donoho1995wavelet}, we use a positive penalty function $\lambda(t)$ in the following objective function. 

For a function $f:[a,b]\mapsto \mathbb{R}$ and $\gamma>0$, define the objective function 
\begin{equation}\label{tractorsplineObjective}
J[f]= \frac{1}{n} \sum_{i=1}^{n} \left( f(t_i)-y_i \right)^2 + \frac{\gamma}{n} \sum_{i=1}^{n} \left( f'(t_i)-v_i \right)^2 +\sum_{i=1}^{n-1} \int_{t_i}^{t_{i+1}}\lambda(t)  f''^2(t)dt,
\end{equation}
where $\gamma$ is the parameter that weights the residuals between $\mathbf{f}'$ and $\mathbf{v}$. We make the simplifying assumption that $\lambda(t)$ is a piecewise constant and adopts a constant value $\lambda_i$ on interval $(t_i,t_{i+1})$ for $i=1,\ldots, n-1$. 

\begin{theorem}\label{TractorSplineTheorem}
For $n\geq2$, the objective function $J[f]$ is minimized by a cubic spline that is unique and linear outside the knots.
\end{theorem}
A further minimizer of \eqref{tractorsplineObjective} is called a \textit{V-spline}, coming from the incorporation with velocity information and applications on vehicle and vessel tracking. The proof of Theorem \ref{TractorSplineTheorem} is in Appendix \ref{AppendixTractorSplineProof}. 



\subsection{Constructing Basis Functions}

To construct basis functions cooperating with position and velocity, it is recommended to use the cubic Hermite spline \citep{Hermite1863Remarque}, by which the combination of heading and speeding at both registration points are treated at two points \citep{hintzen2010improved}. The following equation describes the cubic Hermite spline on interval $[0,1]$
\begin{equation}
f(s)=\left(2s^3-3s^2+1\right)y_0+\left(s^3-2s^2+s\right)v_0+\left(-2s^3+3s^2\right)y_1+\left(s^3-s^2\right)v_1.
\end{equation}
For an arbitrary interval $[t_i, t_{i+1})$, one can replace $s$ with $(t-t_i)/(t_{i+1}-t_i)$ and the cubic spline basis functions are 
\begin{align}\label{hermitebasis1}
& h_{00}^{\left(i\right)}\left(t\right)=
\begin{cases}
2\left(\frac{t-t_{i}}{t_{i+1}-t_{i}}\right)^3-3\left(\frac{t-t_{i}}{t_{i+1}-t_{i}}\right)^2+1 & t_i\leq t<t_{i+1} \\ 
0 & \mbox{otherwise}
\end{cases}, \\
& h_{10}^{\left(i\right)}\left(t\right)=\begin{cases}
\frac{\left(t-t_{i}\right)^3}{\left(t_{i+1}-t_{i}\right)^2}-2\frac{\left(t-t_{i}\right)^2}{t_{i+1}-t_{i}}+\left(t-t_{i}\right)  \hphantom{.}  & t_i\leq t<t_{i+1} \\ 
0 &   \mbox{otherwise}
\end{cases},\\
& h_{01}^{\left(i\right)}\left(t\right)=
\begin{cases}
-2\left(\frac{t-t_i}{t_{i+1}-t_i}\right)^3+3\left(\frac{t-t_i}{t_{i+1}-t_i}\right)^2 \hphantom{+} & t_i\leq t<t_{i+1} \\ 
0 &   \mbox{otherwise}
\end{cases},\\
& h_{11}^{\left(i\right)}\left(t\right)=\begin{cases}
\frac{\left(t-t_i\right)^3}{\left(t_{i+1}-t_i\right)^2}-\frac{\left(t-t_i\right)^2}{t_{i+1}-t_i}  \hphantom{+t-+123-}  & t_i\leq t<t_{i+1} \\ 
0 &   \mbox{otherwise}
\end{cases}.
\end{align}
Consequently, the cubic Hermite spline $f^{(i)}(t)$ on an arbitrary interval $[t_i,t_{i+1})$ with two successive points $P_i=\left\lbrace t_i, y_i,v_i\right\rbrace$ and $P_{i+1}=\left\lbrace t_{i+1}, y_{i+1},v_{i+1} \right\rbrace$ is expressed as
\begin{equation}\label{cubicHermitesplineform}
f^{(i)}(t)=h_{00}^{(i)}(t)y_i+h_{10}^{(i)}(t) \left(t_{i+1}-t_i\right)  v_i+h_{01}^{(i)}(t)y_{i+1} +h_{11}^{(i)}(t)\left(t_{i+1}-t_i\right) v_{i+1}.
\end{equation}

To construct basis function for a V-spline on the entire interval $[a,b]$, the new basis functions are defined in such way, that $N_1 = h^{(1)}_{00}$, $N_2 = h^{(1)}_{10}$, and for all $i=1,2,\ldots,n-2$, 
\begin{align}
N_{2i+1}(t)&=h_{01}^{(i)}(t)+h_{00}^{(i+1)}(t), \\
N_{2i+2}(t)&= h_{11}^{(i)}(t)+h_{10}^{(i+1)}(t),
\end{align}
and
\begin{align}
N_{2n-1}(t) &= 
\begin{cases}
h_{01}^{(n-1)}(t) & \mbox{if $t<t_n$}\\ 
1 & \mbox{if $t=t_n$}
\end{cases},\\
N_{2n}(t) &= h_{11}^{(n-1)}(t).
\end{align}
Any $f$ in the function space can be represented in the form of
\begin{equation}
f=\sum_{k=1}^{2n}N_k(t)\theta_k,
\end{equation}
where $\left\lbrace \theta_k\right\rbrace_{k=1}^{2n}$ are parameters.





\subsection{Computing V-Spline}

Basis functions have been defined in the previous subsection, therefore the V-spline $f(t)$ on $[a,b]$, where $a \leq t_1 < t_2< \cdots < t_{n-1}<t_n \leq b$, can be found by minimizing the objective function \eqref{tractorsplineObjective}, which is corresponding to
\begin{equation}\label{tractormse}
nJ[f](\theta, \lambda,\gamma) = \left(\mathbf{y}-\mathbf{B}\theta\right)^\top \left(\mathbf{y}-\mathbf{B}\theta\right) +\gamma \left(\mathbf{v}-\mathbf{C}\theta\right)^\top \left(\mathbf{v}-\mathbf{C}\theta\right)+n \theta^\top\mathbf{\Omega}_{\lambda}\theta,
\end{equation}
where $\left\lbrace \mathbf{B}\right\rbrace_{ij} = N_j(t_i)$ , $\left\lbrace \mathbf{C}\right\rbrace_{ij} = N'_j(t_i)$ and $\left\lbrace \Omega_{2n}^{(i)} \right\rbrace_{jk}=\int_{a}^{b}\lambda(t) N''_j(t)N''_k(t)dt$. After substituting the series observation $t_1, \ldots, t_n$ into the basis function, one can get $N_1(t_1)=1, N_1(t_2)=0, \ldots, N_{2i-1}(t_{i})=1, N_{2i}(t_{i})=0, \ldots, N_{2n-1}(t_n)=1, N_{2n}(t_n)=0$; and into its first derivative, one will get $N'_1(t_1)=0, N_1'(t_2)=1, \ldots, N_{2i-1}'(t_{i})=0, N_{2i}'(t_{i})=1, \ldots, N_{2n-1}'(t_n)=0, N_{2n}'(t_n)=1$. That means the matrices $\mathbf{B}$ and $\mathbf{C}$ in the equation \eqref{tractormse} are $n \times 2n$ dimensional and the elements are
\begin{align}
\mathbf{B}&=\left\lbrace B\right\rbrace_{ij}=\begin{cases}
1, & j=2i-1\\
0, & \mbox{otherwise}
\end{cases}\\
\mathbf{C}&=\left\lbrace C\right\rbrace_{ij}=\begin{cases}
1, & j=2i\\
0, & \mbox{otherwise}
\end{cases}
\end{align}
where $i=1, \ldots, n$, $j=1,\ldots,2n$ and $k=1,\ldots,2n$. The $i$-th $\Omega^{(i)}$ on the interval $[t_i,t_{i+1}]$ is a $2n \times 2n$ matrix and $\Omega^{(n)}$ does not exist. The detailed structure of $\Omega$ is in Appendix \ref{PenaltyTermDetails}. As a result, the penalty term is a bandwidth four matrix written in such a way:
\begin{equation}
\mathbf{\Omega}_\lambda=\sum_{i=1}^{n-1}\lambda_i\Omega^{(i)}.
\end{equation}



By taking derivatives of equation \eqref{tractormse} with respect to $\theta$, one can achieve 
\begin{equation}
\left(\mathbf{B}^\top\mathbf{B}+\gamma\mathbf{C}^\top\mathbf{C}+n\mathbf{\Omega}_{\lambda}\right)\hat{\theta}=\left(\mathbf{B}^\top\mathbf{y}+\gamma\mathbf{C}^\top\mathbf{v}\right).
\end{equation}
Therefore, the solution is 
\begin{equation}
\hat{\theta}=\left(\mathbf{B}^\top\mathbf{B}+\gamma\mathbf{C}^\top\mathbf{C}+n\mathbf{\Omega}_{\lambda}\right)^{-1}\left(\mathbf{B}^\top\mathbf{y}+\gamma\mathbf{C}^\top\mathbf{v}\right)\label{thetahat}
\end{equation}
a generalized ridge regression. Consequently, the fitted smoothing spline is given by
$\hat{f}(t)=\sum_{k=1}^{2n}N_k(t)\hat{\theta}_k$. 

A smoothing spline with parameters $\lambda(t)$ and $\gamma$ is an example of a linear smoother \citep{esl2009}. This is because the estimated parameters in equation \eqref{thetahat} are a linear combination of $y_i$ and $v_i$. Denote by $\hat{\mathbf{f}}$ and $\hat{\mathbf{f}'}$ the $2n$ vector of fitted values $\hat{f}(t_i)$ and $\hat{f'}(t_i)$ at the training points $t_i$. Then
\begin{equation}
\begin{split}
\hat{\mathbf{f}} =&\mathbf{B}\left(\mathbf{B}^\top\mathbf{B}+\gamma\mathbf{C}^\top\mathbf{C}+n\mathbf{\Omega}_{\lambda}\right)^{-1}\left(\mathbf{B}^\top\mathbf{y}+\gamma\mathbf{C}^\top\mathbf{v}\right)\\
= & \mathbf{S}_{\lambda,\gamma}\mathbf{y}+\gamma\mathbf{T}_{\lambda,\gamma}\mathbf{v} 
\end{split}
\end{equation}
\begin{equation}
\begin{split}
\hat{\mathbf{f}'}
=&\mathbf{C}\left(\mathbf{B}^\top\mathbf{B}+\gamma\mathbf{C}^\top\mathbf{C}+n\mathbf{\Omega}_{\lambda}\right)^{-1}\left(\mathbf{B}^\top\mathbf{y}+\gamma\mathbf{C}^\top\mathbf{v}\right)\\
=&\mathbf{U}_{\lambda,\gamma}\mathbf{y}+\gamma\mathbf{V}_{\lambda,\gamma}\mathbf{v}
\end{split}
\end{equation}
The fitted $\hat{\mathbf{f}}$ and $\hat{\mathbf{f}'}$ are linear in $\mathbf{y}$ and $\mathbf{v}$, and the finite linear operators $\mathbf{S}_{\lambda,\gamma}, \mathbf{T}_{\lambda,\gamma}, \mathbf{U}_{\lambda,\gamma}$ and $\mathbf{V}_{\lambda,\gamma}$ are known as the smoother matrices. One consequence of this linearity is that the recipe for producing $\hat{\mathbf{f}}$ and $\hat{\mathbf{f}'}$ from $\mathbf{y}$ and $\mathbf{v}$, do not depend on $\mathbf{y}$ and $\mathbf{v}$ themselves; $\mathbf{S}_{\lambda,\gamma}, \mathbf{T}_{\lambda,\gamma}, \mathbf{U}_{\lambda,\gamma}$ and $\mathbf{V}_{\lambda,\gamma}$ depend only on $t_i,\lambda(t)$ and $\gamma$.

Suppose in a traditional least squares fitting, $\mathbf{B}_\xi$ is $N \times M$ matrix of $M$ cubic-spline basis functions evaluated at the $N$ training points $x_i$, with knot sequence $\xi$ and $M \ll N$. Thus the vector of fitted spline values is given by
\begin{align}\label{fhy}
\hat{\mathbf{f}}=\mathbf{B}_\xi\left(\mathbf{B}^\top_\xi\mathbf{B}_\xi\right)^{-1}\mathbf{B}_\xi\mathbf{y}=\mathbf{H}_\xi\mathbf{y}
\end{align}
Here the linear operator $\mathbf{H}_\xi$ is a symmetric, positive semidefinite matrices, and $\mathbf{H}_\xi\mathbf{H}_\xi=\mathbf{H}_\xi$ (idempotent) \citep{esl2009}. In our case, it is easily seen that $\mathbf{S}_{\lambda,\gamma}, \mathbf{T}_{\lambda,\gamma}, \mathbf{U}_{\lambda,\gamma}$ and $\mathbf{V}_{\lambda,\gamma}$ are symmetric, positive semidefinite matrices as well. Additionally, by Cholesky decomposition
\begin{equation}
\left(\mathbf{B}^\top\mathbf{B}+\gamma\mathbf{C}^\top\mathbf{C}+n\mathbf{\Omega}_{\lambda}\right)^{-1}=\mathbf{R}\mathbf{R}^\top,
\end{equation}
it is easy to prove that $\mathbf{T}_{\lambda,\gamma}=\mathbf{B}\mathbf{R}\mathbf{R}^\top\mathbf{C}^\top$ and $\mathbf{U}_{\lambda,\gamma}=\mathbf{C}\mathbf{R}\mathbf{R}^\top\mathbf{B}^\top$, then we will have 
 $\mathbf{T}_{\lambda,\gamma}= \mathbf{U}_{\lambda,\gamma}^\top$. When $\lambda=\gamma=0$, the matrix $\mathbf{S}_{\lambda_0,\gamma_0}=\mathbf{B}\left(\mathbf{B}^\top\mathbf{B}\right)^{-1}\mathbf{B}^\top$ is idempotent.  


\begin{corollary}\label{TractorsplineCorollary}
If $f(t)$ is the V-spline on the entire interval $[t_1,t_n]$, for sufficient cases of a piecewise constant $\lambda(t)$ and parameter $\gamma$, $f(t)$ has the following property:
\begin{enumerate}\itemsep0em 
\item if $\gamma \neq 0$, then $f$ and $f'$ are continuous, $f''$ is piecewise linear but not continuous at knots;
\item if $\gamma = 0$, the same as above;
\item if $\lambda(t)$ is not only piecewise constant but constant on the entire interval and $\gamma \neq 0$, the same as above;
\item if $\lambda(t)$ is not only piecewise constant but constant and $\gamma = 0$, then $f$, $f'$ are continuous, $f''$ is piecewise linear and continuous at knots.
\end{enumerate}
\end{corollary}

The proof of the Corollary \ref{TractorsplineCorollary} is in Appendix \ref{proofofCorollary}. 


\subsection{Adjusted Penalty Term and Parameter Function}


The polynomial in cubic Hermite spline form consists of two points with two positions and two velocities. Suppose there are two points $P_1=\{t_1,y_1,v_1\}$ and $P_2=\{t_2,y_2,v_2\}$ on the an arbitrary interval $[t_1,t_2]$. For sake of simplicity, by assuming $t_1=0$ and $y_1=0$, one can achieve a simple cubic Hermite spline from equation \eqref{cubicHermitesplineform} 
\begin{equation}
f(t) =\left\{ \left(s^3-2s^2+s\right) v_1+\left(-2s^3+3s^2\right)\frac{\Delta d_1}{\Delta T_1}+\left(s^3-s^2\right)v_2 \right\}\Delta T_1, 
\end{equation}
where $s=\frac{t}{\Delta T_1}$. Thus, the second derivative of $f$ is  
\begin{equation}
f''(t)=\frac{1}{\Delta T_1}\left\{ 6\left(\varepsilon_1+\varepsilon_2\right)s-2\left(2\varepsilon_1+\varepsilon_2\right)\right\},
\end{equation}
where $\varepsilon_1=v_1-\bar{v}$, $\varepsilon_2=v_2-\bar{v}$ and $\bar{v}$ is the average velocity $\Delta d_1/\Delta T_1$. Hence, the penalty term $\lambda \int_{t_1}^{t_2}\left(f''(t)\right)^2dt = \lambda \int_0^1 \left(f''(s)\right)^2\Delta T_1 ds$. Being more explicit, the penalty term becomes 
\begin{equation}\label{VSplinePenaltyLambda}
\lambda \int_0^1\left( \frac{f''(s)}{\Delta T_1}\right)^2\Delta T_1 ds = \lambda \frac{\left(2\varepsilon_1+\varepsilon_2\right)^2+3\varepsilon_2^2}{\Delta T_1}. 
\end{equation}
Given a constant $\lambda = \frac{\left(\Delta T_1\right)^3}{\left(\Delta d_1\right)^2}\eta$, the penalty term \eqref{VSplinePenaltyLambda} becomes
\begin{equation}\label{APTintroequation}
\begin{split}
\eta \frac{\left(\Delta T_1\right)^2}{\left(\Delta d_1\right)^2} \left(\left(2\varepsilon_1+\varepsilon_2\right)^2+3\varepsilon_2^2\right) &= \eta \frac{\left(2\varepsilon_1+\varepsilon_2\right)^2+3\varepsilon_2^2}{\bar{v}^2} \\ &\sim \left(\frac{\mbox{discrepancy in velocity}}{\mbox{average velocity}}\right)^2, 
\end{split}
\end{equation}
which will be enormous with large measured errors in velocity $v_1$ or $v_2$ comparing to average velocity $\bar{v}$. 

With noise-free observations, the cubic Hermite spline effectively reconstructs the trajectory between two successive points. However, in real life application, the measurements are coming with errors. Imagine the situation that a vehicle stays unchanged in its positions between a long time gap $\Delta T$. Due to the noise, the velocities $v_1$ and $v_2$ are non-zeros and heading to different directions. Cooperating with velocity, the Hermite spline basis function reconstructs the trajectory that starts from the first point along the direction of $v_1$ and ends at the second point at direction of $v_2$. It returns a wiggle between the two points, however, there should be a straight line. Or after a long break, where $\Delta T$ is extremely large, the velocity becomes worthless. The vehicle can be anywhere during such a long time. In this scenario, we are expecting that the vehicle is following a straight line path. See Figure \ref{figureAPT}. 
\begin{figure}[h]
\centering
\begin{subfigure}[b]{0.45\textwidth}
\centering
\resizebox{\linewidth}{!}{
	  \begin{tikzpicture}
	    \begin{axis}[ 
	        axis x line=bottom,
	        axis y line=left,
	        xlabel={$t$},
	        ylabel={$y$},
	        domain=-0.1:1.11,
	        xtick={0, ..., 1},
	        samples=100,
	        xticklabels=\empty,
	        yticklabels=\empty,
	    ]
       \addplot[line width=0.5mm, red!80!black][domain=0:1]{1.25*pow(x,3)-2.25*pow(x,2)+x};
       %\addplot [black] [domain=-0.1:0]{1.25*pow(x,3)-2.25*pow(x,2)+x};
       \addplot [white] [domain=-0.1:0]{x};
	   \addplot[->] coordinates {(0,0) (0.1,0.1)};
	   \addplot[->] coordinates {(1,0) (1.24,0.06)};
	   \node[circle,fill=blue!30,inner sep=2pt] at (axis cs:0,0) {$y_1$};
	   \node[circle,fill=blue!30,inner sep=2pt] at (axis cs:1,0) {$y_2$};
	   \node[] at (axis cs:0.11,0.11) {$v_1$};
	   \node[] at (axis cs:1.18,0.06) {$v_2$};
	   \end{axis}
	  \end{tikzpicture}
      }
      \caption{cubic Hermite spline reconstruction}
\end{subfigure}
\begin{subfigure}[b]{0.45\textwidth}
       \centering
       \resizebox{\linewidth}{!}{
       \begin{tikzpicture}
       	    \begin{axis}[
       	        axis x line=bottom,
       	        axis y line=left,
       	        xlabel={$t$},
       	        ylabel={$y$},
       	        domain=-0.1:1.11,
       	        xtick={0, ..., 1},
       	        samples=100,
       	        xticklabels=\empty,
       	        yticklabels=\empty,
       	      ]
       	    \addplot [line width=0.1mm, white][domain=0:1]{1.25*pow(x,3)-2.25*pow(x,2)+x};
       	    \addplot [white] [domain=-0.1:0]{x};
			\addplot [line width=0.5mm, red!80!black][domain=0:1] {0};
	   		\addplot[->] coordinates {(0,0) (0.1,0.1)};
	   		\addplot[->] coordinates {(1,0) (1.24,0.06)};
	   		\node[circle,fill=blue!30,inner sep=2pt] at (axis cs:0,0) {$y_1$};
	   		\node[circle,fill=blue!30,inner sep=2pt] at (axis cs:1,0) {$y_2$};
	   		\node[] at (axis cs:0.11,0.11) {$v_1$};
	   		\node[] at (axis cs:1.18,0.06) {$v_2$};
       	  \end{axis}
       \end{tikzpicture}
       }
	\caption{straight line reconstruction}
\end{subfigure}
\caption{Comparing reconstructions of cubic Hermite spline and straight line. On the left side, a genuine cubic Hermite spline is cooperating with noisy velocities. Even though the vehicle is not moving, the reconstruction is following the directions of $P_1$ and $P_2$ and gives a wiggle between the two points. On the right side, it is an expected reconstruction between two not-moving points after a long time gap. }\label{figureAPT}
\end{figure}

We address this issue by introducing an adjusted penalty term $\frac{\left(\Delta T_i\right)^3}{\left(\Delta d_i\right)^2}$ found in equation \eqref{APTintroequation} to the penalty function $\lambda(t)$, in which the V-spline is penalized by its real differences of $\Delta d_i$ and $\Delta T_i$ for each interval $[t_i, t_{i+1}]$. With this term, either the measurement in velocity becomes unreliable comparing to average speed or after a long time gap, the adjusted penalty term will works on the penalty function and forces it to return a straight line rather than a curve on this particular domain. From the physical point of view, the term is the reciprocal of the product of velocity and acceleration. Either velocity or acceleration goes to zero, the vehicle should either stop, which returns a straight line through time on $y$ axis, or keep moving with the same speed, which returns a linear interpolation instead of a curved path. 

Therefore, the final form of the penalty function $\lambda(t,\eta)$ is piecewise constant having the following form on each interval 
\begin{equation}\label{adjustedpenalty}
\lambda_i=\frac{\left(\Delta T_i\right)^3}{\left(\Delta d_i\right)^2}\eta,
\end{equation}
where $\eta$ is an unknown parameter and $t_i\leq t < t_{i+1}$, $i=1,\ldots,n-1$. Eventually, in the objective function, there are two unknown parameters: $\eta$ controlling the curvature of V-spline on different domains and $\gamma$ controlling the residuals of velocity. The piecewise constant function $\lambda(t,\eta)$ is using a data driven method to model the penalty function in the adaptive V-splines. 




\section{Parameter Selection and Cross-Validation}

The problem of choosing the smoothing parameter is ubiquitous in curve estimation, and there are two different philosophical approaches to this question. The first one is to regard the free choice of smoothing parameter as an advantageous feature of the procedure. The other one is to find the parameter automatically by the data \citep{green1993nonparametric}. We prefer the latter one and use data to train our model and find the best parameters. The most well-known method for this is cross-validation.


Assuming that mean of the random errors is zero, the true regression curve $f(t)$ has the property that, if an observation $y$ is taken away at a point $t$, the value $f(t)$ is the best predictor of $y$ in terms of returning a least value of $\left(y-f(t)\right)^2$. 

Now, focus on an observation $y_i$ at point $t_i$ as being a new observation by omitting it from the set of data, which are used to estimate $\hat{f}$. Denote by $\hat{f}^{(-i)}(t,\lambda)$ the estimated function from the remaining data, where $\lambda$ is the smoothing parameter. Then $\hat{f}^{(-i)}\left(t,\lambda\right)$ is the minimizer of  
\begin{equation}\label{originalcv}
\frac{1}{n}\sum_{j \neq i}\left(y_j-f(t_j) \right)^2+\lambda\int (f'')^2dt,
\end{equation}
and can be quantified by the cross-validation score function
\begin{equation}\label{cvform}
\mbox{CV}(\lambda)=\frac{1}{n}\sum_{i=1}^{n}\left(  y_i-\hat{f}^{(-i)}(t_i,\lambda)\right) ^2.
\end{equation}
The basis idea of the cross-validation is to choose the value of $\lambda$ that minimizes $\mbox{CV}(\lambda)$ \citep{green1993nonparametric}. 

%An efficient way to calculate the cross-validation score is introduced by \citep{green1993nonparametric}. 
Through the equation \eqref{fhy}, it is known that the value of the smoothing spline $\hat{f}$ depends linearly on the data $y_1,\ldots,y_n$. Define the matrix $A(\lambda)$, which is a map vector of observed values $y_i$ to predicted values $\hat{f}(t_i)$. Then we have
\begin{equation}\label{crossvalidationmatrixA}
\hat{\mathbf{f}}=A(\lambda)\mathbf{y}
\end{equation}
and the following lemma.
\begin{lemma}\citep{green1993nonparametric}\label{cvlema}
The cross-validation score satisfies
\begin{equation}
\mbox{CV}(\lambda)=\frac{1}{n} \sum_{i=1}^n \left(\frac{y_i-\hat{f}(t_i)}{1-A_{ii}(\lambda)}\right)^2
\end{equation}
where $\hat{f}$ is the spline smoother calculated from the full data set $\left\lbrace (t_i,y_i)\right\rbrace$ with smoothing parameter $\lambda$.
\end{lemma}

For a V-spline and its objective function, there are two parameters $\eta$, as is shown in \eqref{adjustedpenalty}, and $\gamma$ to be estimated for. Therefore, $\hat{f}^{(-i)}(t,\eta,\gamma)$ is the minimizer of  
\begin{align}
\frac{1}{n}\sum_{j \neq i}\left( y_j-f(t_j) \right)^2+\frac{\gamma}{n}\sum_{j \neq i} \left( v_j-f'(t_j) \right)^2+ \int \lambda(t,\eta) \left( f'' \right)^2dt,
\end{align}
and the cross-validation score function is
\begin{align}
\mbox{CV}\left(\lambda(t,\eta),\gamma\right)=\frac{1}{n}\sum_{i=1}^{n}\left( y_i-\hat{f}^{(-i)}\left(t_i,\eta,\gamma\right) \right) ^2.
\end{align}
Additionally, it is known that the parameter $\hat{\theta}=\left(B^\top B+\gamma C^\top C+n\Omega_\lambda\right)^{-1}\left(B^\top\mathbf{y}+\gamma C^\top\mathbf{v}\right)$ and will give us the following form \small
\begin{equation}
\begin{split}
 \hat{\mathbf{f}}&=B\hat{\theta}=B\left(B^\top B+\gamma C^\top C+n\Omega_\lambda\right)^{-1}B^\top\mathbf{y}+B\left(B^\top B+\gamma C^\top C+n\Omega_\lambda\right)^{-1} C^\top\mathbf{v}\\&=S\mathbf{y}+\gamma T\mathbf{v},
 \end{split}
 \end{equation}
 \begin{equation}
 \begin{split}
\hat{\mathbf{f}}'&=C\hat{\theta}=C\left(B^\top B+\gamma C^\top C+n\Omega_\lambda\right)^{-1}B^\top\mathbf{y}+C\left(B^\top B+\gamma C^\top C+n\Omega_\lambda\right)^{-1}C^\top \mathbf{v}\\&=U\mathbf{y}+\gamma V\mathbf{v}.
 \end{split}
\end{equation}\normalsize
From Lemma \ref{cvlema}, we can prove the following theorem: 
\begin{theorem}\label{tractorsplinecvscore}
The cross-validation score of a V-spline satisfies
\begin{equation}\label{tractorcv}
\mbox{CV}\left(\eta,\gamma\right)=\frac{1}{n}\sum_{i=1}^{n} \left( \frac{\hat{f}(t_i)-y_i+\gamma \frac{T_{ii}}{1-\gamma V_{ii}}(\hat{f}'(t_i)-v_i)}{1-S_{ii}-\gamma\frac{T_{ii}}{1-\gamma V_{ii}}U_{ii}} \right)^2
\end{equation}
where $\hat{f}$ is the V-spline smoother calculated from the full data set $\left\lbrace (t_i,y_i,v_i)\right\rbrace$ with smoothing parameter $\eta$ and $\gamma$.
\end{theorem}

The proof of Theorem \ref{tractorsplinecvscore} follows immediately from a lemma, and gives an expression for the deleted residuals $y_i-\hat{f}^{(-i)}(t_i)$ and $v_i-\hat{f}'^{(-i)}(t_i)$ in terms of $y_i-\hat{f}(t_i)$ and $v_i-\hat{f}'(t_i)$ respectively. 

\begin{lemma} \label{cvlemma}
Given fixed $\eta$ to $\lambda(t,\eta)$, $\gamma$ and $i$, denote $\mathbf{f}^{(-i)}$ by the vector with components $f_j^{(-i)}=\hat{f}^{(-i)}\left(t_j,\eta,\gamma\right)$,  $\mathbf{f}'^{(-i)}$ by the vector with components $f_j'^{(-i)}=\hat{f}'^{(-i)}\left(t_j,\eta,\gamma\right)$, and define vectors $\mathbf{y}^*$ and $\mathbf{v}^*$ by 
\begin{align}
\begin{cases}
y_j^*=y_j &j \neq i\\
y_i^*=\hat{f}^{(-i)}(t_i) &\mbox{otherwise}
\end{cases}\\
\begin{cases}
v_j^*=v_j &j \neq i\\
v_i^*=\hat{f}'^{(-i)}(t_i) &\mbox{otherwise}
\end{cases}
\end{align}
Then
\begin{align}
\mathbf{\hat{f}}^{(-i)}&=S\mathbf{y}^*+\gamma T\mathbf{v}^*\\
\mathbf{\hat{f}}'^{(-i)}&=U\mathbf{y}^*+\gamma V\mathbf{v}^*
\end{align}
\end{lemma}


\section{Simulation Study} %and Error Analysis}


In this section, we give extensive comparison of methods for regularly sampled time series data followed by simulation of irregularly sampled data. The examination is based on the ability of reconstructing four functions derived from \textit{Blocks}, \textit{Bumps}, \textit{HeaviSine} and \textit{Doppler}, which were used in \citep{donoho1994ideal, donoho1995adapting, abramovich1998wavelet} because of their caricature features in imaging, spectroscopy and other scientific signal processing. Notice that the Blocks and Bumps functions have infinite first derivatives, and cannot be inferred by V-splines. Hence, we use these functions, denoted by $g(t)$, as models of velocity rather than position. 

If the original function $g(t)$ is treated as the velocity function of some function $f(t)$, then $f'(t)=g(t)$. By setting initial position $f(t_0)=y_0=0$ and acceleration $a_0=0$, one can calculate the position data with the following formula: 
\begin{equation}\label{generateVelocity}
f(t_{i+1})=f(t_i)+\left(g(t_i)+g(t_{i+1}) \right)\frac{t_{i+1}-t_i}{2}. 
\end{equation}
Further, we add some \iid  zero-mean $\varepsilon$ noise to them 
\begin{align}\label{tractorsplinegeneratefunctions}
\begin{split}
y_i &= f(t_i) + \varepsilon_f, \\
v_i &= g(t_i) + \varepsilon_g,
\end{split}
\end{align}
where $\varepsilon_f\sim N(0,\sigma_f/SNR)$, $\varepsilon_g\sim N(0,\sigma_g/SNR)$ and $i=0,\ldots,n$. A random seed was fixed to ensure repeatability. The noise is \iid zero-mean Gaussian distributed with standard deviation regarding to signal-to-noise ratio (SNR), which specifies the ratio of the standard deviation of the function to the standard deviation of the simulated errors. Explicitly, if the standard deviation of the true signal $f$ is $\sigma_f$, the simulated data will be $f+\varepsilon$, where the simulated error  $\varepsilon \sim N(0,\sigma_f/SNR)$. The value of SNR can be chosen 7 or 3. 



\subsection{Regularly Sampled Time Series}

A set of regularly sampled time series data has equal time difference between each pair of successive points. For example, denoted by $\Delta T_i = t_{i+1}-t_i$ for $i=1,\ldots,n-1$, then $\Delta T_1=\cdots = \Delta T_{n-1}$.

Following \cite{nason2010wavelet}, we fix $n=1024$ in the simulation. To compare the performance of the proposed method, a few competitors are attending this competition. For wavelet transform reconstructions, we use the threshold policy of \textit{sure} and \textit{BayesThresh} with levels $l=4, \ldots, 9$  \citep{donoho1995adapting, abramovich1998wavelet}. A semi-parametric regression model with spatially adaptive penalized splines (\textit{P-spline}) is added in comparison  \citep{krivobokova2008fast, ruppert2003semiparametric}.

In the V-spline, there are two parameters $\eta$ and $\gamma$ to optimize. To evaluate the performance of the velocity term in objective function \eqref{tractorsplineObjective} and the adjusted penalty term in \eqref{adjustedpenalty}, the parameter $\gamma$ is set as 0 in one reconstruction of V-spline, whose objective function and solution become
\begin{equation}\label{ofgamma0}
J[f]_{\gamma=0}= \frac{1}{n} \sum_{i=1}^{n} \left(f(t_i)-y_i\right)^2 +\sum_{i=1}^{n-1} \int_{t_i}^{t_{i+1}}\lambda(t) f''^2 dt,
\end{equation}
and
\begin{equation}\label{thetahat0}
\hat{\theta}_{\gamma=0}=\left(\mathbf{B}^\top\mathbf{B}+n\Omega_{\lambda}\right)^{-1}\mathbf{B}^\top\mathbf{y}.
\end{equation}
In another reconstruction, the adjusted penalty term in \eqref{adjustedpenalty} is removed and the model is denoted by ``V-spline without APT''. 




\subsubsection{Numerical Examples}


Figure \ref{num1} to \ref{num4} display the original (velocity), generated position, wavelet with two different threshold methods, P-spline and three kinds of V-spline fitted functions. The parameters $\lambda$ and $\gamma$ of a V-spline are automatically selected with the formula \eqref{tractorcv} by $\textsf{optim}$ function in $R$ \citep{nelder1965simplex}.


\begin{figure}
    \centering
    \begin{subfigure}{0.45\textwidth}
    \centering
    \includegraphics[width=\linewidth,height=0.45\textwidth]{Chapters/02TractorSplineTheory/plot/ggplot/ggBlocks.pdf}
    \caption{The \textit{Blocks} function}
    \end{subfigure}%
    \begin{subfigure}{0.45\textwidth}
    \centering
    \includegraphics[width=\linewidth,height=0.45\textwidth]{Chapters/02TractorSplineTheory/plot/ggplot/ggBlocksNoise.pdf}
    \caption{Noisy \textit{Blocks} at \textit{SNR}=7}
    \end{subfigure}
    \begin{subfigure}{0.45\textwidth}
    \centering
    \includegraphics[width=\linewidth,height=0.45\textwidth]{Chapters/02TractorSplineTheory/plot/ggplot/ggBlocksPosition.pdf}
    \caption{Generated positions}
    \end{subfigure}
    \begin{subfigure}{0.45\textwidth}
    \centering
    \includegraphics[width=\linewidth,height=0.45\textwidth]{Chapters/02TractorSplineTheory/plot/ggplot/ggBlocksPositionNoise.pdf}
    \caption{Noisy position at \textit{SNR}=7}
    \end{subfigure}
    \begin{subfigure}{0.45\textwidth}
    \centering
    \includegraphics[width=\linewidth,height=0.45\textwidth]{Chapters/02TractorSplineTheory/plot/ggplot/ggBlocksSure.pdf}
    \caption{Reconstruction from Wavelet by sure threshold}
    \end{subfigure}
    \begin{subfigure}{0.45\textwidth}
    \centering
    \includegraphics[width=\linewidth,height=0.45\textwidth]{Chapters/02TractorSplineTheory/plot/ggplot/ggBlocksBayes.pdf}
    \caption{Reconstruction from Wavelet by BayesThresh approach}
    \end{subfigure}
    \begin{subfigure}{0.45\textwidth}
    \centering
    \includegraphics[width=\linewidth,height=0.45\textwidth]{Chapters/02TractorSplineTheory/plot/ggplot/ggBlocksPSpline.pdf}
    \caption{Reconstruction by P-spline  }
    \end{subfigure}
    \begin{subfigure}{0.45\textwidth}
    \centering
    \includegraphics[width=\linewidth,height=0.45\textwidth]{Chapters/02TractorSplineTheory/plot/ggplot/ggBlocksGamma.pdf}
    \caption{Reconstruction by V-spline setting $\gamma=0$}
    \end{subfigure}
  \begin{subfigure}{0.45\textwidth}
    \centering
    \includegraphics[width=\linewidth,height=0.45\textwidth]{Chapters/02TractorSplineTheory/plot/ggplot/ggBlocksTractorAPT.pdf}
    \caption{Reconstruction by V-spline with conventional penalty term}
    \end{subfigure}
    \begin{subfigure}{0.45\textwidth}
    \centering
    \includegraphics[width=\linewidth,height=0.45\textwidth]{Chapters/02TractorSplineTheory/plot/ggplot/ggBlocksTractor.pdf}
    \caption{Reconstruction by the proposed V-spline\\\mbox{  } }
    \end{subfigure}
\caption{Numerical example: $\textit{Blocks}$. Comparison of different reconstruction methods with simulated data.}\label{num1}
 \end{figure}

\begin{figure}
    \centering
    \begin{subfigure}{0.45\textwidth}
    \centering
    \includegraphics[width=\linewidth,height=0.45\textwidth]{Chapters/02TractorSplineTheory/plot/ggplot/ggBumps.pdf}
    \caption{True \textit{Bumps} function}
    \end{subfigure}%
    \begin{subfigure}{0.45\textwidth}
    \centering
    \includegraphics[width=\linewidth,,height=0.45\textwidth]{Chapters/02TractorSplineTheory/plot/ggplot/ggBumpsNoise.pdf}
    \caption{Noisy \textit{Bumps} at \textit{SNR}=7}
    \end{subfigure}
    \begin{subfigure}{0.45\textwidth}
    \centering
    \includegraphics[width=\linewidth,height=0.45\textwidth]{Chapters/02TractorSplineTheory/plot/ggplot/ggBumpsPosition.pdf}
    \caption{Generated positions}
    \end{subfigure}
    \begin{subfigure}{0.45\textwidth}
    \centering
    \includegraphics[width=\linewidth,height=0.45\textwidth]{Chapters/02TractorSplineTheory/plot/ggplot/ggBumpsPositionNoise.pdf}
    \caption{Noisy position at \textit{SNR}=7}
    \end{subfigure}
    \begin{subfigure}{0.45\textwidth}
    \centering
    \includegraphics[width=\linewidth,height=0.45\textwidth]{Chapters/02TractorSplineTheory/plot/ggplot/ggBumpsSure.pdf}
    \caption{Reconstruction from Wavelet by sure threshold}
    \end{subfigure}
    \begin{subfigure}{0.45\textwidth}
    \centering
    \includegraphics[width=\linewidth,height=0.45\textwidth]{Chapters/02TractorSplineTheory/plot/ggplot/ggBumpsBayes.pdf}
    \caption{Reconstruction from Wavelet by BayesThresh approach}
    \end{subfigure}
    \begin{subfigure}{0.45\textwidth}
    \centering
    \includegraphics[width=\linewidth,height=0.45\textwidth]{Chapters/02TractorSplineTheory/plot/ggplot/ggBumpsPSpline.pdf}
    \caption{Reconstruction by P-spline }
    \end{subfigure}
    \begin{subfigure}{0.45\textwidth}
    \centering
    \includegraphics[width=\linewidth,height=0.45\textwidth]{Chapters/02TractorSplineTheory/plot/ggplot/ggBumpsGamma.pdf}
    \caption{Reconstruction by V-spline setting $\gamma=0$}
    \end{subfigure}
  \begin{subfigure}{0.45\textwidth}
    \centering
    \includegraphics[width=\linewidth,height=0.45\textwidth]{Chapters/02TractorSplineTheory/plot/ggplot/ggBumpsTractorAPT.pdf}
    \caption{Reconstruction by V-spline with conventional penalty term}
    \end{subfigure}
    \begin{subfigure}{0.45\textwidth}
    \centering
    \includegraphics[width=\linewidth,height=0.45\textwidth]{Chapters/02TractorSplineTheory/plot/ggplot/ggBumpsTractor.pdf}
    \caption{Reconstruction by the proposed V-spline \\\mbox{  } }
    \end{subfigure}
\caption{Numerical example: $\textit{Bumps}$. Comparison of different reconstruction methods with simulated data.}\label{num2}
 \end{figure}


\begin{figure}
    \centering
    \begin{subfigure}{0.45\textwidth}
    \centering
    \includegraphics[width=\linewidth,height=0.45\textwidth]{Chapters/02TractorSplineTheory/plot/ggplot/ggHeaviSine.pdf}
    \caption{True \textit{HeaviSine} function}
    \end{subfigure}%
    \begin{subfigure}{0.45\textwidth}
    \centering
    \includegraphics[width=\linewidth,height=0.45\textwidth]{Chapters/02TractorSplineTheory/plot/ggplot/ggHeaviSineNoise.pdf}
    \caption{Noisy \textit{HeaviSine} at \textit{SNR}=7}
    \end{subfigure}
    \begin{subfigure}{0.45\textwidth}
    \centering
    \includegraphics[width=\linewidth,height=0.45\textwidth]{Chapters/02TractorSplineTheory/plot/ggplot/ggHeaviSinePosition.pdf}
    \caption{Generated positions}
    \end{subfigure}
    \begin{subfigure}{0.45\textwidth}
    \centering
    \includegraphics[width=\linewidth,height=0.45\textwidth]{Chapters/02TractorSplineTheory/plot/ggplot/ggHeaviSinePositionNoise.pdf}
    \caption{Noisy position at \textit{SNR}=7}
    \end{subfigure}
    \begin{subfigure}{0.45\textwidth}
    \centering
    \includegraphics[width=\linewidth,height=0.45\textwidth]{Chapters/02TractorSplineTheory/plot/ggplot/ggHeaviSineSure.pdf}
    \caption{Reconstruction from Wavelet by sure threshold}
    \end{subfigure}
    \begin{subfigure}{0.45\textwidth}
    \centering
    \includegraphics[width=\linewidth,height=0.45\textwidth]{Chapters/02TractorSplineTheory/plot/ggplot/ggHeaviSineBayes.pdf}
    \caption{Reconstruction from Wavelet by BayesThresh approach}
    \end{subfigure}
    \begin{subfigure}{0.45\textwidth}
    \centering
    \includegraphics[width=\linewidth,height=0.45\textwidth]{Chapters/02TractorSplineTheory/plot/ggplot/ggHeaviSinePSpline.pdf}
    \caption{Reconstruction by P-spline}
    \end{subfigure}
    \begin{subfigure}{0.45\textwidth}
    \centering
    \includegraphics[width=\linewidth,height=0.45\textwidth]{Chapters/02TractorSplineTheory/plot/ggplot/ggHeaviSineGamma.pdf}
    \caption{Reconstruction by V-spline setting $\gamma=0$}
    \end{subfigure}
  \begin{subfigure}{0.45\textwidth}
    \centering
    \includegraphics[width=\linewidth,height=0.45\textwidth]{Chapters/02TractorSplineTheory/plot/ggplot/ggHeaviSineTractorAPT.pdf}
    \caption{Reconstruction by V-spline with conventional penalty term }
    \end{subfigure}
    \begin{subfigure}{0.45\textwidth}
    \centering
    \includegraphics[width=\linewidth,height=0.45\textwidth]{Chapters/02TractorSplineTheory/plot/ggplot/ggHeaviSineTractor.pdf}
    \caption{Reconstruction by the proposed V-spline \\ \mbox{  }}
    \end{subfigure}
\caption{Numerical example: $\textit{HeaviSine}$. Comparison of different reconstruction methods with simulated data.}\label{num3}
 \end{figure}

\begin{figure}
    \centering
    \begin{subfigure}{0.45\textwidth}
    \centering
    \includegraphics[width=\linewidth,height=0.45\textwidth]{Chapters/02TractorSplineTheory/plot/ggplot/ggDoppler.pdf}
    \caption{True \textit{Doppler} function}
    \end{subfigure}%
    \begin{subfigure}{0.45\textwidth}
    \centering
    \includegraphics[width=\linewidth,,height=0.45\textwidth]{Chapters/02TractorSplineTheory/plot/ggplot/ggDopplerNoise.pdf}
    \caption{Noisy \textit{Doppler} at \textit{SNR}=7}
    \end{subfigure}
    \begin{subfigure}{0.45\textwidth}
    \centering
    \includegraphics[width=\linewidth,height=0.45\textwidth]{Chapters/02TractorSplineTheory/plot/ggplot/ggDopplerPosition.pdf}
    \caption{Generated positions}
    \end{subfigure}
    \begin{subfigure}{0.45\textwidth}
    \centering
    \includegraphics[width=\linewidth,height=0.45\textwidth]{Chapters/02TractorSplineTheory/plot/ggplot/ggDopplerPositionNoise.pdf}
    \caption{Noisy position at \textit{SNR}=7}
    \end{subfigure}
    \begin{subfigure}{0.45\textwidth}
    \centering
    \includegraphics[width=\linewidth,height=0.45\textwidth]{Chapters/02TractorSplineTheory/plot/ggplot/ggDopplerSure.pdf}
    \caption{Reconstruction from Wavelet by sure threshold}
    \end{subfigure}
    \begin{subfigure}{0.45\textwidth}
    \centering
    \includegraphics[width=\linewidth,height=0.45\textwidth]{Chapters/02TractorSplineTheory/plot/ggplot/ggDopplerBayes.pdf}
    \caption{Reconstruction from Wavelet by BayesThresh approach}
    \end{subfigure}
    \begin{subfigure}{0.45\textwidth}
    \centering
    \includegraphics[width=\linewidth,height=0.45\textwidth]{Chapters/02TractorSplineTheory/plot/ggplot/ggDopplerPSpline.pdf}
    \caption{Reconstruction by P-spline}
    \end{subfigure}
    \begin{subfigure}{0.45\textwidth}
    \centering
    \includegraphics[width=\linewidth,height=0.45\textwidth]{Chapters/02TractorSplineTheory/plot/ggplot/ggDopplerGamma.pdf}
    \caption{Reconstruction by V-spline setting $\gamma=0$}
    \end{subfigure}
  \begin{subfigure}{0.45\textwidth}
    \centering
    \includegraphics[width=\linewidth,height=0.45\textwidth]{Chapters/02TractorSplineTheory/plot/ggplot/ggDopplerTractorAPT.pdf}
    \caption{Reconstruction by V-spline with conventional penalty term}
    \end{subfigure}
    \begin{subfigure}{0.45\textwidth}
    \centering
    \includegraphics[width=\linewidth,height=0.45\textwidth]{Chapters/02TractorSplineTheory/plot/ggplot/ggDopplerTractor.pdf}
    \caption{Reconstruction by the proposed V-spline\\\mbox{  } }
    \end{subfigure}
\caption{Numerical example: $\textit{Doppler}$. Comparison of different reconstruction methods with simulated data}\label{num4}
 \end{figure}

By comparing the results, we can see that all these methods can rebuild up the skeleton of generated trajectory. \textit{Wavelet(sure)} method has more wiggles in interior interval than \textit{Wavelet(BayesThresh)} does, and the latter one becomes fluctuation near boundary knots. \textit{P-spline} gives a smoother fit than wavelets, but the drawback is lack of specific details. V-spline without velocity loses some information, as can be seen from \textit{Blocks} and \textit{Bumps} where there should be a straight line. V-spline without adjusted penalty term gets over-fitting when the direction changes more frequently than normal, although it catches specific feature in \textit{HeaviSine}. The proposed V-spline performs much better than other methods and returns the near-true trajectory reconstructions.  


\begin{figure}
    \centering
    \begin{subfigure}{\textwidth}
    \centering
    \includegraphics[width=0.45\textwidth]{Chapters/02TractorSplineTheory/plot/ggplot/ggBlocksPenaltyBar.pdf}
    \includegraphics[width=0.45\textwidth]{Chapters/02TractorSplineTheory/plot/ggplot/ggBlocksPenaltyLine.pdf}
    \caption{Distribution of the penalty values in reconstructed \textit{Blocks}}
    \end{subfigure}
    \begin{subfigure}{\textwidth}
    \centering
    \includegraphics[width=0.45\textwidth]{Chapters/02TractorSplineTheory/plot/ggplot/ggBumpsPenaltyBar.pdf}
    \includegraphics[width=0.45\textwidth]{Chapters/02TractorSplineTheory/plot/ggplot/ggBumpsPenaltyLine.pdf}
    \caption{Distribution of the penalty values in reconstructed \textit{Bumps}}
    \end{subfigure}
    \begin{subfigure}{\textwidth}
    \centering
    \includegraphics[width=0.45\textwidth]{Chapters/02TractorSplineTheory/plot/ggplot/ggHeaviSinePenaltyBar.pdf}
    \includegraphics[width=0.45\textwidth]{Chapters/02TractorSplineTheory/plot/ggplot/ggHeaviSinePenaltyLine.pdf}
    \caption{Distribution of the penalty values in reconstructed \textit{HeaviSine}}
    \end{subfigure}
    \begin{subfigure}{\textwidth}
    \centering
    \includegraphics[width=0.45\textwidth]{Chapters/02TractorSplineTheory/plot/ggplot/ggDopplerPenaltyBar.pdf}
    \includegraphics[width=0.45\textwidth]{Chapters/02TractorSplineTheory/plot/ggplot/ggDopplerPenaltyLine.pdf}
    \caption{Distribution of the penalty values in reconstructed \textit{Doppler}}
    \end{subfigure}
\caption{Distribution of the penalty values $\lambda(t,\eta)$ in V-spline. Figures on the left side indicate the values varying in intervals. On the right side, these values are projected into reconstructions. The bigger the blacks dots present, the larger the penalty values are.}\label{numpenalty}
\end{figure}


Figure \ref{numpenalty} shows the estimated penalty values $\lambda(t,\eta)=\frac{\left(\Delta T\right)^3}{\left(\Delta d\right)^2}\eta$ at SNR=7. The figures in the left column illustrate the values of the penalty term at different intervals, the figures in the right column are the observations and reconstructed trajectory. The bigger black dots present larger penalty values. It can be seen that $\lambda(t,\eta)$ adapts to the smoothness pattern of position and will be large where a long time gap may occur. The details of how this penalty function works will be explained in next subsection. Figure \ref{TractorsplineSNR3} illustrates the reconstructions of V-spline at SNR=3.


\begin{figure}
    \centering
    \begin{subfigure}{0.45\textwidth}
    \centering
    \includegraphics[width=\textwidth]{Chapters/02TractorSplineTheory/plot/ggplot/ggBlocksTractorVelocity.pdf}
    \caption{Estimated \textit{Blocks}  }
    \end{subfigure}%
    \begin{subfigure}{0.45\textwidth}
    \centering
    \includegraphics[width=\textwidth]{Chapters/02TractorSplineTheory/plot/ggplot/ggBumpsTractorVelocity.pdf}
    \caption{Estimated \textit{Bumps}  }
    \end{subfigure}
    \begin{subfigure}{0.45\textwidth}
    \centering
    \includegraphics[width=\textwidth]{Chapters/02TractorSplineTheory/plot/ggplot/ggHeaviSineTractorVelocity.pdf}
    \caption{Estimated \textit{HeaviSine}  }
    \end{subfigure}
    \begin{subfigure}{0.45\textwidth}
    \centering
    \includegraphics[width=\textwidth]{Chapters/02TractorSplineTheory/plot/ggplot/ggDopplerTractorVelocity.pdf}
    \caption{Estimated \textit{Doppler}  }
    \end{subfigure}
\caption{Estimated velocity functions by V-spline. The velocity is generated from the original simulation functions by equation \eqref{generateVelocity}}\label{numvtractor}
 \end{figure}



Figure \ref{numvtractor} demonstrates the estimated velocity functions. By taking the first derivative of fitted V-spline, it is simple to get the original four velocity functions. The fittings of velocity are not as smooth as that of position, because we only care about the smoothness of position rather than velocity in our cross-validation formula \eqref{tractorcv}. However, velocity information does help us in reconstructing the trajectory.



\subsubsection{Evaluation}

To examine the performance of the V-spline, we conduct a evaluation by comparing the mean squared errors and true mean squared errors, which are respectively calculated with the following formulas: 
\begin{align}
\mbox{MSE}&= \frac{1}{n} \sum_{i=1}^{n} \left( y_i-\hat{f}_{\eta,\gamma}(t_i) \right)^2,\\
\mbox{TMSE}&= \frac{1}{n} \sum_{i=1}^{n} \left( f(t_i)-\hat{f}_{\eta,\gamma}(t_i) \right)^2.
\end{align}


The results are shown in Tables \ref{mse3200} and \ref{tmse3200}. All of these methods have good performances in fitting noisy data. The differences of mean squared error between these methods are not significant, as can be seen from Table \ref{mse3200}. The proposed method is not the best among these simulations according to MSE. However, from Table \ref{tmse3200}, V-spline returns the smallest true mean squared errors. The difference is significant, that means the reconstruction from V-spline is closer to the true trajectory. 
 
%\begin{sidewaystable}
 \begin{table}
 	\centering
 	\caption{MSE. Mean squared errors of different methods. The numbers in bold indicate the least error among these methods under the same level. The difference is not significant.}\label{mse3200}
	\setlength\tabcolsep{1.5pt}
	\begin{tabular}{|c|c|C{1.9cm}|C{1.9cm}|C{1.9cm}|C{1.9cm}|C{1.9cm}|C{1.9cm}|}
\hline	MSE $\left(10^{-4}\right)$   & SNR & V-spline & VS$_{\footnotesize\gamma=0}$ & VS$_{\scriptsize \mbox{APT=0}}$   & P-spline & W(sure)& W(Bayes)\\ \hline
\multirow{2}{*}{\textit{Blocks}}     & 7   &  16.53& 15.99 & 16.69 & 16.14  & \textbf{15.39} & 16.68 \\ \cline{2-8}
       & 3   &  89.79 & \textbf{87.64} & 89.94  & 88.27 & 98.35 & 90.24 \\ \hline
\multirow{2}{*}{\textit{Bumps}}     & 7   & 4.40 & 4.19 & 4.55 & 4.33 & \textbf{4.18} & 4.59 \\ \cline{2-8}
      & 3   & 23.93 & \textbf{23.19} & 24.10 & 23.55 & 26.23 & 23.74 \\ \hline
\multirow{2}{*}{\textit{HeaviSine}}  & 7   & 4.16 & 4.01 &4.16 & 4.02 & \textbf{3.79} & 4.19 \\ \cline{2-8}
     & 3   & 22.63 & \textbf{22.19} & 22.65 & 22.02 & 23.53 & 22.07 \\ \hline
\multirow{2}{*}{\textit{Doppler}}    & 7   & 1.15 & \textbf{1.07} & 1.10 & 1.15  & \textbf{1.07} & 1.13  \\ \cline{2-8}
      & 3   & 6.27 & \textbf{5.94} &6.28 & 6.05  & 6.85 & 6.29  \\ \hline
	\end{tabular}
\end{table}
 %\end{sidewaystable}

%\begin{sidewaystable} 
\begin{table}
	\centering
	\caption{TMSE. True mean squared errors of different methods. The numbers in bold indicate the least error among these methods under the same level. The proposed V-spline returns the smallest TMSE among all the methods under the same level except for $\textit{Doppler}$ with SNR=7. The differences are significant. }\label{tmse3200}
	\setlength\tabcolsep{1.5pt}
	\begin{tabular}{|c|c|C{1.9cm}|C{1.9cm}|C{1.9cm}|C{1.9cm}|C{1.9cm}|C{1.9cm}|}
\hline	TMSE $\left(10^{-6}\right)$  & SNR & V-spline & VS$_{\footnotesize\gamma=0}$ & VS$_{\scriptsize \mbox{APT=0}}$   & P-spline & W(sure) &  W(Bayes)\\ \hline
		
\multirow{2}{*}{\textit{Blocks}}  & 7   & \textbf{1.75} & 54.25 &  28.68   & 54.76   & 201.02   & 182.12   \\ \cline{2-8}
	     & 3   & \textbf{16.44} & 152.5 & 30.76  & 171.59   & 1138.08  & 712.36  \\ \hline
\multirow{2}{*}{\textit{Bumps}}     & 7  & \textbf{1.64} & 23.44  & 21.10     & 24.21 & 71.71 & 69.26 \\ \cline{2-8}
        & 3  & \textbf{8.51} & 77.78  &37.12     & 77.52 & 330.77 & 238.79 \\ \hline
\multirow{2}{*}{\textit{HeaviSine}}  & 7 & \textbf{1.53}& 7.80  & 1.56     & 9.54   & 55.37  &44.88  \\ \cline{2-8}
      & 3 & \textbf{8.21}& 33.56  & 8.49 & 34.26 & 240.72& 110.49\\ \hline
\multirow{2}{*}{\textit{Doppler}}    & 7   & 1.51& 6.67  & \textbf{1.08}   &  8.26   & 14.87  & 12.01  \\ \cline{2-8}
    & 3   & \textbf{8.10} & 22.14  & 8.25   & 19.95    &81.48  &50.33   \\ \hline
	\end{tabular}	
\end{table}
%\end{sidewaystable}


\subsubsection{Residual Analysis}

The simulated data is generated by equations \eqref{tractorsplinegeneratefunctions} and the SNRs are set at 7 and 3 separately to compare the performances of different algorithms. All of the algorithms can reconstruct the true trajectory from noisy data and return acceptable MSE values, though V-spline returns the least TMSE in most of the circumstances. 

Table \ref{tablecompareSNR} is comparing the capability of V-spline in retrieving the true SNR. The measurements are generated from $f$ and $g$ with predefined SNR. The V-spline reconstructs the true trajectory and retrieves the SNR value, both of which are close to the truth. 

\begin{table}
	\centering
    \caption{Retrieved SNR. V-spline effectively retrieves the SNR, which is calculated by $\sigma_{\hat{f}} / \sigma_{(\hat{f}-y)}$. }\label{tablecompareSNR}
	\begin{tabular}{|c|C{3cm}|C{3cm}|C{3cm}|}
\hline	 SNR   & predefined value & generated $f$ & V-spline $\hat{f}$ \\ \hline
\multirow{2}{*}{\textit{Blocks}}  & 7   & 6.9442    &  6.9485     \\ \cline{2-4}
		   & 3   &  2.9761   &  2.9817   \\ \hline
\multirow{2}{*}{\textit{Bumps}}    & 7  & 6.9442    &  6.9548  \\ \cline{2-4}
		   & 3  & 2.9761    &   2.9953 	   \\ \hline
\multirow{2}{*}{\textit{HeaviSine}}  & 7 & 6.9442    &   6.9207   \\ \cline{2-4}
		  & 3 & 2.9761    &   2.9706  \\ \hline
\multirow{2}{*}{\textit{Doppler}}     & 7   & 6.9442   &  6.8757   \\ \cline{2-4}
		  & 3   & 2.9761   &  2.9625   \\ \hline
	\end{tabular}
\end{table}

Further analysis in figures \ref{tractorsplineSNR7acf} and \ref{tractorsplineSNR3acf} shows that the residuals from V-splines are independent. 


\subsection{Irregularly Sampled Time Series Data}

A set of irregularly sampled time series data has different time differences between each pair of successive points. The distribution of $\Delta T_i$ is not uniform.

In this section, it is shown that the proposed V-spline has the ability to reconstruct the true trajectory even though the data is irregularly sampled. With the same functions that are used in the previous section, we firstly generate the simulation data of length $n=1024$. Then we draw a length of 512 subsequence with indices $1,3,\ldots,1023$ from the mother dataset for regularly sampled points and another 512 random indices for irregularly sampled points. See Figure \ref{gghistIrregularTime}. 

\begin{figure}[!h]
    \centering
    \includegraphics[width=0.75\textwidth]{Chapters/02TractorSplineTheory/plot/ggplot/gghistIrregularTime.pdf}
 \caption{Histogram of $\Delta T$ for irregularly sampled data}\label{gghistIrregularTime}
 \end{figure}


The reconstructions of regularly and irregularly sampled data are very competitive. 

\begin{figure}[!h]
    \centering
    \begin{subfigure}{\textwidth}
    \centering
    \includegraphics[width=0.45\textwidth]{Chapters/02TractorSplineTheory/plot/ggplot/ggblocksreg.pdf}
    \includegraphics[width=0.45\textwidth]{Chapters/02TractorSplineTheory/plot/ggplot/ggblocksire.pdf}
    \caption{Reconstruction of \textit{Blocks} from regularly and irregularly sampled data}
    \end{subfigure}
    \begin{subfigure}{\textwidth}
    \centering
    \includegraphics[width=0.45\textwidth]{Chapters/02TractorSplineTheory/plot/ggplot/ggbumpsreg.pdf}
    \includegraphics[width=0.45\textwidth]{Chapters/02TractorSplineTheory/plot/ggplot/ggbumpsire.pdf}
    \caption{Reconstruction of \textit{Bumps} from regularly and irregularly sampled data}
    \end{subfigure}
    \begin{subfigure}{\textwidth}
    \centering
    \includegraphics[width=0.45\textwidth]{Chapters/02TractorSplineTheory/plot/ggplot/ggheavisinreg.pdf}
    \includegraphics[width=0.45\textwidth]{Chapters/02TractorSplineTheory/plot/ggplot/ggheavisinire.pdf}
    \caption{Reconstruction of \textit{HeaviSine} from regularly and irregularly sampled data}
    \end{subfigure}
    \begin{subfigure}{\textwidth}
    \centering
    \includegraphics[width=0.45\textwidth]{Chapters/02TractorSplineTheory/plot/ggplot/ggdopplerreg.pdf}
    \includegraphics[width=0.45\textwidth]{Chapters/02TractorSplineTheory/plot/ggplot/ggdopplerire.pdf}
    \caption{Reconstruction of \textit{Doppler} from regularly and irregularly sampled data}
    \end{subfigure}
 \caption{Comparison of regularly and irregularly sampled data}\label{irregularFigure}
 \end{figure}




\begin{table}
	\centering
    \caption{Retrieved SNR of reconstructions from regularly and irregularly sampled data  }\label{tablecompareSNRIreReg}
	\begin{tabular}{|c|C{3cm}|C{3cm}|}
\hline	 SNR           & Regularly & Irregularly  \\ \hline
\textit{Blocks}        &    7.0479  & 6.8692    \\  \hline
\textit{Bumps}       &    7.0241  & 7.1937     \\  \hline
\textit{HeaviSine}  &   7.2367    & 6.8793   \\ \hline
\textit{Doppler}     &    6.8692   & 7.3645    \\ \hline
	\end{tabular}
\end{table}


\begin{table}
	\centering
    \caption{MSE and TMSE of reconstructions from regularly and irregularly sampled data  }\label{tablecompareSMEIreReg}
	\begin{tabular}{|c|C{2cm}|C{2cm}|C{2cm}|C{2cm}|} \hline	
	& \multicolumn{2}{|c|}{MSE $\times 10^{-4}$} & \multicolumn{2}{c|}{TMSE $\times 10^{-6}$} \\ \cline{2-5}
	                 & Regularly & Irregularly & Regularly & Irregularly \\ \hline
\textit{Blocks}        &    8.0260 &  8.3358  & 3.5197 & 10.8596  \\  \hline
\textit{Bumps}       &    2.1374  & 2.0203  & 1.6662 & 6.2586 \\  \hline
\textit{HeaviSine}  &   2.0232   & 2.1272  &  1.1275 & 1.1077  \\ \hline
\textit{Doppler}     &  0.5251 & 0.5219 & 1.0101 & 1.7832   \\ \hline
	\end{tabular}
\end{table}

%\clearpage 

\section{Inference of Tractor Trajectories}\label{splineapplication}


In the real life application, the movement of vehicles or tractors has complicated features. The velocity of a moving object looks like a combination of the original bumps and blocks functions: it is a turbulent waves fluctuating around zero. In this section, we apply the proposed V-spline to real dataset, which is recorded by a GPS unit mounted on a tractor. The original dataset contains the information about time marks, longitude, latitude, velocity, bearing (in degrees,  heading to North) and boom status. The data is not regularly recorded and the time difference has a high variance. 

In a two or higher $d$-dimensional curve nonparametric regression, consider the general form of a length $n$ time series data points $\left\lbrace t_1,p_1,s_1\right\rbrace, \ldots, \left\lbrace t_n,p_n,s_n\right\rbrace$, such that $a \leq t_1<t_2< \cdots < t_n \leq b$, $p_i$ and $s_i$ are $d$-dimensional vectors contain position and velocity information at time $i$ respectively. The positive piecewise constant function $\lambda(t) = \lambda_i$ on each interval $t_i \leq t<t_{i+1}, t_0=a, t_{n+1}=b$.  Then the function $f:[a,b]\mapsto\mathbb{R}^d$ with $\gamma>0$ is a V-spline in the $d$-dimensional space if it is the solution to the generic form of the objective function: 
\begin{equation}\label{tractorsplineObjective2D}
J[f]= \frac{1}{n} \sum_{i=1}^{n} \lVert f(t_i)-p_i\rVert_d^2 + \frac{\gamma}{n} \sum_{i=1}^{n} \lVert f'(t_i)-s_i \rVert_d^2 +\sum_{i=0}^{n} \lambda_i\int_{t_i}^{t_{i+1}} \lVert f''(t)\rVert_d^2 dt. 
\end{equation}


Particularly, the GPS data is recorded in a 2-dimensional form, in which scenario $d=2$. Hence, in the following application, we split the 2-dimensional function $f(x,y)$ into two sub functions $f_x(t)$ on $x$-axis and $f_y(t)$ on $y$-axis with respect to time $t$. Compared with other parameters, choosing time $t$ to be the parameter has some advantages: (i) the expressions of all the constraints are simpler \citep{zhang2013cubic}; (ii) it can be simply applied from 2-dimension to 3-dimension by adding an extra $z$-axis. Without loss of generality, a dataset in a higher dimensional space can be projected into several sub-spaces, such as $p=\left\lbrace x,y,z,\ldots \right\rbrace$ and $s=\left\lbrace u,v,w,\ldots \right\rbrace$. 


Thereafter, we convert the longitude and latitude information from a 3D sphere to 2D surface first by Universal Transverse Mercator coordinate system (UTM) and then project the speed $s$ into $u$ and $v$ on $x$-axis and $y$-axis respectively by 
\begin{align}
u &=s\cdot \sin \left(\omega\frac{\pi}{180}\right),\\
v &= s\cdot \cos \left(\omega\frac{\pi}{180}\right),
\end{align}
where $\omega$ is the bearing in degrees. Boom status is tagged as 0 if it is not operating and 1 if it is. Time marks are transformed by subtracting the first mark, in which way the time starts from 0. Time duplicated data, caused by errors, have been removed from the dataset\footnote{In some cases, further data simplification to remove spurious or non-informative observations may be warranted. In Appendix \ref{appendSimp}, we introduce a new data simplification method for vehicle trajectories which compares favorably with Douglas-Peucker algorithm.  }. For the convenience of comparing with wavelet algorithm, we choose the first 512 out of 928 rows of data. The original measurements are plotted in Figure \ref{original512}.


\begin{figure}
\centering
    \begin{subfigure}{0.45\textwidth}
    \centering
    \includegraphics[width=\textwidth]{Chapters/02TractorSplineTheory/plot/ggplot/gg512Points.pdf}
    \caption{Original positions}\label{gg512Points}
    \end{subfigure}%
    \begin{subfigure}{0.45\textwidth}
    \centering
    \includegraphics[width=\textwidth]{Chapters/02TractorSplineTheory/plot/ggplot/gg512Path.pdf}
    \caption{Line-based trajectory}\label{gg512Path}
    \end{subfigure}
    \begin{subfigure}{0.45\textwidth}
    \centering
    \includegraphics[width=\textwidth]{Chapters/02TractorSplineTheory/plot/ggplot/gg512PointsX.pdf}
    \caption{Original positions on $x$-axis}\label{gg512PointsX}
    \end{subfigure}
    \begin{subfigure}{0.45\textwidth}
    \centering
    \includegraphics[width=\textwidth]{Chapters/02TractorSplineTheory/plot/ggplot/gg512PointsY.pdf}
    \caption{Original positions on $y$-axis}\label{gg512PointsY}
    \end{subfigure}
\caption{Original data points. Figure \ref{gg512Points} is th original positions recorded by GPS units. Circle points means the boom is not operating; cross points means it is operating. Figure \ref{gg512Path} is the line-based trajectory by simply connecting all points sequentially with straight lines. Figure \ref{gg512PointsX} is the original $x$ position. Figure \ref{gg512PointsY} is the original $y$ positions.}\label{original512}
 \end{figure}


In order to fit the real data, we bring the parameter $\eta_d$ to our model. Then, we are now having three parameters $\eta_d$ and $\eta_u$ regarding boom status and $\gamma$ controlling velocity residuals. The criteria of a good fitting are that it can catch more information, recognize time gaps between two points where tractor stops and return a smaller MSE. 



\subsection{1-Dimensional Trajectory}

We treat $x$ and $y$ position separately and compare how the velocity information and the adjusted penalty term of equation \eqref{adjustedpenalty} work in our model. All parameters in fitted V-spline are automatically selected by cross-validation by equation \eqref{tractorcv}. Figure \ref{1dx} and Figure \ref{1dy} compare the results of fitted methods on $x$ and $y$ axes. P-spline gives over-fitting on $x$ axis reconstruction and not applicable on $y$ axis due to errors. Wavelet(sure) misses some key points at corners when a tractor tries to turn around. V-spline without adjusted penalty term presents less fitting at time gap knots, where time marks keep increasing while position stays the same and velocity is 0. If we take the last knot $p_k$ before and the first knot $p_{k+1}$ after the time gap, Hermite spline basis will use $y_k, v_k, y_{k+1}$ and $v_{k+1}$ to build up a cubic spline, even though the velocity information is not useful. That is why we got a curve rather than a straight line. Wavelet(BayesThresh), V-spline without velocity and proposed V-spline give acceptable results.


Table \ref{1dxymse} illustrates the MSE of all methods on both $x$ and $y$ axes. The proposed V-spline returns the least errors among all methods.
\begin{figure}
    \centering
    \begin{subfigure}{0.45\textwidth}
    \centering
    \includegraphics[width=\textwidth,height=0.5\textwidth]{Chapters/02TractorSplineTheory/plot/ggplot/ggRealdataXPSpline.pdf}
    \caption{Reconstruction by P-spline}\label{ggRealdataXPSpline}
    \end{subfigure}%
    \begin{subfigure}{0.45\textwidth}
    \centering
    \includegraphics[width=\textwidth,,height=0.5\textwidth]{Chapters/02TractorSplineTheory/plot/ggplot/ggRealdataXSure.pdf}
    \caption{Reconstruction by wavelet ($\textit{sure}$)}\label{ggRealdataXSure}
    \end{subfigure}
    \begin{subfigure}{0.45\textwidth}
    \centering
    \includegraphics[width=\textwidth,height=0.5\textwidth]{Chapters/02TractorSplineTheory/plot/ggplot/ggRealdataXBayes.pdf}
    \caption{Reconstruction by wavelet ($\textit{Bayes}$)}\label{ggRealdataXBayes}
    \end{subfigure}
    \begin{subfigure}{0.45\textwidth}
    \centering
    \includegraphics[width=\textwidth,height=0.5\textwidth]{Chapters/02TractorSplineTheory/plot/ggplot/ggRealdataXTractorGamma.pdf}
    \caption{Reconstruction by V-spline setting  $\gamma=0$ }\label{ggRealdataXTractorGamma}
    \end{subfigure}
    \begin{subfigure}{0.45\textwidth}
    \centering
    \includegraphics[width=\textwidth,height=0.5\textwidth]{Chapters/02TractorSplineTheory/plot/ggplot/ggRealdataXTractorAPT.pdf}
    \caption{Reconstruction by V-spline setting with conventional penalty term}\label{ggRealdataXTractorAPT}
    \end{subfigure}
    \begin{subfigure}{0.45\textwidth}
    \centering
    \includegraphics[width=\textwidth,height=0.5\textwidth]{Chapters/02TractorSplineTheory/plot/ggplot/ggRealdataXTractor.pdf}
    \caption{Reconstruction by proposed V-spline \\ \mbox{  }}\label{ggRealdataXTractor}
    \end{subfigure}
 \caption{Fitted data points on $x$ axis. Figure \ref{ggRealdataXPSpline} Fitted by P-spline, which gives over-fitting on these points and misses some information. Figure \ref{ggRealdataXSure} Fitted by wavelet ($\textit{sure}$) algorithm. At some turning points, it gives over-fitting. Figure \ref{ggRealdataXBayes} Fitted by wavelet ($\textit{BayesThresh}$) algorithm. It fits better than ($\textit{sure}$) and the result is close to the proposed method. Figure \ref{ggRealdataXTractorGamma} Fitted by V-spline without velocity information. The reconstruction is good to get the original trajectory. Figure \ref{ggRealdataXTractorAPT} Fitted by V-spline without adjusted penalty term. It gives less fitting at boom-not-operating points because of a large time gap. Figure \ref{ggRealdataXTractor} Fitted by proposed method. It fits all data points in a good way.}\label{1dx}
 \end{figure}


\begin{figure}
    \centering
    \begin{subfigure}{0.45\textwidth}
    \centering
    \includegraphics[width=\linewidth,height=0.5\textwidth]{Chapters/02TractorSplineTheory/plot/ggplot/ggRealdataYPSpline.pdf}
    \caption{Not available for P-spline}\label{ggRealdataYPSpline}
    \end{subfigure}%
    \begin{subfigure}{0.45\textwidth}
    \centering
    \includegraphics[width=\linewidth,,height=0.5\textwidth]{Chapters/02TractorSplineTheory/plot/ggplot/ggRealdataYSure.pdf}
    \caption{Reconstruction by wavelet ($\textit{sure}$)}\label{ggRealdataYSure}
    \end{subfigure}
    \begin{subfigure}{0.45\textwidth}
    \centering
    \includegraphics[width=\linewidth,height=0.5\textwidth]{Chapters/02TractorSplineTheory/plot/ggplot/ggRealdataYBayes.pdf}
    \caption{Reconstruction by wavelet ($\textit{Bayes}$) }\label{ggRealdataYBayes}
    \end{subfigure}
    \begin{subfigure}{0.45\textwidth}
    \centering
    \includegraphics[width=\linewidth,height=0.5\textwidth]{Chapters/02TractorSplineTheory/plot/ggplot/ggRealdataYTractorGamma.pdf}
    \caption{Reconstruction by V-spline setting  $\gamma=0$ }\label{ggRealdataYTractorGamma}
    \end{subfigure}
    \begin{subfigure}{0.45\textwidth}
    \centering
    \includegraphics[width=\linewidth,height=0.5\textwidth]{Chapters/02TractorSplineTheory/plot/ggplot/ggRealdataYTractorAPT.pdf}
    \caption{Reconstruction by V-spline setting with conventional penalty term}\label{ggRealdataYTractorAPT}
    \end{subfigure}
    \begin{subfigure}{0.45\textwidth}
    \centering
    \includegraphics[width=\linewidth,height=0.5\textwidth]{Chapters/02TractorSplineTheory/plot/ggplot/ggRealdataYTractor.pdf}
    \caption{Reconstruction by proposed V-spline\\ \mbox{  }}\label{ggRealdataYTractor}
    \end{subfigure}
\caption{Fitted data points on $y$ axis. Figure \ref{ggRealdataYPSpline} Fitted P-spline is not applicable on $y$ axis as the matrix is not invertible. Figure \ref{ggRealdataYSure} Fitted by wavelet ($\textit{sure}$) algorithm. At some turning points, it gives over-fitting. Figure \ref{ggRealdataYBayes} Fitted by wavelet ($\textit{BayesThresh}$) algorithm is much better than wavelet ($\textit{sure}$). Figure \ref{ggRealdataYTractorGamma} Fitted by V-spline without velocity information. The reconstruction is good to get the original trajectory. Figure \ref{ggRealdataYTractorAPT} Fitted by V-spline without adjusted penalty term. It gives less fitting at boom-not-operating. Figure \ref{ggRealdataYTractor} Fitted by proposed method. It fits all data points in a good way.}\label{1dy}
 \end{figure}

%\begin{sidewaystable}
\begin{table}
\caption{Mean squared error. V-spline returns smallest errors among all these methods. P-spline was unable to reconstruct the $y$ trajectory as the original dataset contains 0 $\Delta_y$.} \label{1dxymse}
% \centering
	\setlength\tabcolsep{1.5pt}
\begin{center}
 	\begin{tabular}{|c|C{2cm}|C{2cm}|C{2cm}|C{2cm}|C{2cm}|C{2cm}|}
 		\hline
 		MSE   &  V-spline & VS$_{\gamma=0}$ & VS$_{\scriptsize \mbox{APT=0}}$  & P-spline &  W(sure) & W(Bayes)\\ \hline 
	\textit{$x$}   &  \textbf{0.2046} & 0.2830 & 0.3298     & 2860.5480   & 256.0494  & 6.2959  \\ \hline
	\textit{$y$}   &  \textbf{0.0020} & 0.3062 & 0.3115     & \textit{NA} & 1960.2220 & 19.3330  \\ \hline
 	\end{tabular}
 \end{center}
\end{table}
%\end{sidewaystable} 

The penalty function of the proposed V-spline is
\begin{equation}\label{penaltylamb}
\lambda(t)=b\frac{\left(\Delta T\right)^3}{\left(\Delta d\right)^2}\eta_d+(1-b)\frac{\left(\Delta T\right)^3}{\left(\Delta d\right)^2}\eta_u, \mbox{ where}
\begin{cases}
b=1 & \mbox{if boom is operating}\\
b=0 & \mbox{if boom is not operating}
\end{cases}
\end{equation}
To explain the differences more clearly, we take $\lambda(t)$ in our demonstration. Figure \ref{penaltyxygg} indicates that at turning points and long time gap knots, the adjusted penalty term will lead $\lambda(t)$ to large values, which forces the spline to be a straight line between two knots. It can be seen in Figure \ref{penaltyxyggXYPath} clearly. Histogram plots of $\lambda(t)$ show that most of the penalty values are small, which allows the V-spline to go as closer as possible to the observed points. Only a few of penalty values are large, so that V-spline gives a straight line at tricky points. 


\begin{figure}
    \centering
    \begin{subfigure}{\textwidth}
    \centering
    \includegraphics[width=0.45\linewidth]{Chapters/02TractorSplineTheory/plot/ggplot/ggRealdataXPenaltyLine2.pdf}
    \includegraphics[width=0.45\linewidth]{Chapters/02TractorSplineTheory/plot/ggplot/ggRealdataYPenaltyLine2.pdf}
    \caption{Distribution of the penalty term on $x$ and $y$}\label{penaltyxyggXYLine}
    \end{subfigure}
%    \begin{subfigure}{\textwidth}
%    \centering
%    \includegraphics[width=0.45\linewidth]{Chapters/02TractorSplineTheory/plot/ggplot/ggRealdataXPenaltyHist.pdf}
%    \includegraphics[width=0.45\linewidth]{Chapters/02TractorSplineTheory/plot/ggplot/ggRealdataYPenaltyHist.pdf}
%    \caption{Histograms of the penalty term on $x$ and $y$}\label{penaltyxyggXYHist}
%    \end{subfigure}
    \begin{subfigure}{\textwidth}
    \centering
    \includegraphics[width=0.45\linewidth]{Chapters/02TractorSplineTheory/plot/ggplot/ggRealdataXPenaltyPath2.pdf}
    \includegraphics[width=0.45\linewidth]{Chapters/02TractorSplineTheory/plot/ggplot/ggRealdataYPenaltyPath2.pdf}
    \caption{Reconstruction on $x$ and $y$}\label{penaltyxyggXYPath}
    \end{subfigure}
 %\caption{The penalty term $n\theta^\top\Omega\theta$ of V-spline on $x$ and $y$ axes. The big black dots in Figure \ref{penaltyxyggXYPath} indicate large penalty values. It can be seen that most of large penalty values occur at turnings, where the tractor likely slows down and takes breaks. }\label{penaltyxygg}
 \caption{The penalty value $\lambda(t)$ of the V-spline on $x$ and $y$ axes. Red dots are the measurements $\mathbf{y}$. The bigger red dots in Figure \ref{penaltyxyggXYPath} indicate larger penalty values. It can be seen that most of large penalty values occur at turnings, where the tractor likely slows down and takes breaks. }\label{penaltyxygg}
 \end{figure}

The 1-dimensional reconstruction gets the best fittings $\hat{f}_x$ and $\hat{f}_y$ on $x$ and $y$ axes separately by using different penalty values, denoted as $\eta_{d,x}$, $\eta_{u,x}$, $\eta_{d,y}$, $\eta_{u,y}$, $\gamma_x$ and $\gamma_y$. The final reconstruction is the combination of  $\hat{f}_x$ and $\hat{f}_y$. It is shown in Figure \ref{1DCombinedXY}. 
\begin{figure}
  \centering
    \includegraphics[width=\textwidth,height=10cm]{Chapters/02TractorSplineTheory/plot/ggplot/ggRealdataCombinedXY2.pdf} 
  \caption{Combined reconstruction on $x$ and $y$. Red dots are the measurements $\mathbf{y}$. The bigger size it is, the larger penalty value it indicates. }\label{1DCombinedXY}
\end{figure}



\subsection{2-Dimensional Trajectory}

In a 2-dimensional trajectory reconstruction, different from combined 1-dimensional reconstruction, we use the same parameters $\lambda_d$, $\lambda_u$ and $\gamma$ for both $x$ and $y$ axes. The overall best parameters return the least cross-validation score on all axes. Explicitly, it is calculated by the following formula 
\begin{equation}
\mbox{CV}=\mbox{CV}_x+\mbox{CV}_y.
\end{equation}
In the adjusted penalty term, $\Delta d$ is the Euclidean distance $\Delta_d(p_1,p_2)=\sqrt{(\Delta x)^2+(\Delta y)^2}$ between two positions on the 2D surface. Similar to 1-dimensional reconstruction, the velocity information keeps trajectory in the right direction and the penalty term makes sure that the crazy curve will disappear between long-time-gap points. Figure \ref{completecombind2dxy} demonstrates the complete 2D reconstruction of the whole dataset.  

\begin{figure}
  \centering
 \begin{subfigure}{\textwidth}
     \centering
%     \includegraphics[width=0.45\linewidth]{Chapters/02TractorSplineTheory/plot/ggplot/ggRealdataXYPenaltyPathofX.pdf}
%     \includegraphics[width=0.45\linewidth]{Chapters/02TractorSplineTheory/plot/ggplot/ggRealdataXYPenaltyPathofY.pdf}
     \includegraphics[width=0.45\linewidth]{Chapters/02TractorSplineTheory/plot/ggplot/ggRealdataXYPenaltyPathofX2.pdf}
     \includegraphics[width=0.45\linewidth]{Chapters/02TractorSplineTheory/plot/ggplot/ggRealdataXYPenaltyPathofY2.pdf}
     \caption{2-dimensional reconstruction on separate $x$ and $y$}
     \end{subfigure}
     \begin{subfigure}{\textwidth}
     \centering
     \includegraphics[width=0.9\linewidth]{Chapters/02TractorSplineTheory/plot/ggplot/ggRealdataXYPenaltyPathofXY2.pdf}
     \caption{2-dimensional reconstruction}
     \end{subfigure}
 \caption{2-dimensional reconstruction. Larger dots indicate bigger values of penalty function $\lambda(t)$.}\label{completecombind2dxy}
\end{figure}

The penalty function $\lambda(t)$ of a 2-dimensional reconstruction is shared by $x$ and $y$ axes and presented in Figure \ref{2dpenalty}. The complete penalty term is 
\begin{equation}
n\theta_x^\top\Omega_{\eta_d,\eta_u}\theta_x + n\theta_y^\top\Omega_{\eta_d,\eta_u}\theta_y.
\end{equation}
Similarly, most of the large penalty values appear at long-time-gap knots and turning points. A histogram plot of penalty function shows that most of the values are small and only a couple of them are large. 
\begin{figure}
  \centering
    \includegraphics[width=0.45\textwidth]{Chapters/02TractorSplineTheory/plot/ggplot/ggRealdataXYPenaltyLine.pdf}
    \includegraphics[width=0.45\textwidth]{Chapters/02TractorSplineTheory/plot/ggplot/ggRealdataXYPenaltyHist.pdf} 
  \caption{Penalty value of $\lambda(t)$ in 2-dimensional reconstruction.}\label{2dpenalty}
\end{figure}


The following Figure \ref{complete2DXY} is a complete reconstruction from the whole observed dataset $\left\lbrace x,u,y,v\right\rbrace$. The overall reconstruction gives a smoothing path that goes through each measurement and avoids curvatures at turning points. 
%Instead of reconstructing on $x$ and $y$ axes separately, it chooses the penalty value with respect to the balance on both of the two directions. 
\begin{figure}
\centering
\includegraphics[width=0.9\linewidth]{Chapters/02TractorSplineTheory/plot/ggplot/ggRealdataCompleteXY.pdf}
\caption{2-dimensional reconstruction. Larger dots indicate bigger values of penalty function $\lambda(t)$.}\label{complete2DXY}
\end{figure}



\section{Conclusion and Discussion}

In this chapter, a V-spline model is proposed to solve the objective function, which is consisting of both position and velocity information. The adjusted penalty function adapts to complicated curvatures. In a $d$-dimensional space, V-spline can be projected into sub-spaces with respect to $t$ and combined each solution together as a final. This method performs better when we know the position and velocity information than other methods. 

Additionally, the reconstruction of a V-spline contains $4\times (n-1)$ parameters if we have $n$ knots. By adding $2\times (n-2)$ constraints, the original function, and its first derivative are continuous at each interior knots, the degrees of freedom will be $4\times (n-1)-2\times (n-2)=2n$. Because there are $n$ position and $n$ velocity points, we do not need to specify more parameters or add more constraints to the model. 

In our experimental studies, the MSE of the V-spline were neither significantly better or worse than other methods. However, its true MSE was significantly less. 

In parameter selection, the cross-validation only focuses on the errors of $f$ ignoring that in $f'$. So the reconstruction of $f'$ is not as smooth as that of $f$, which does not affect trajectory reconstruction. A drawback of V-spline is that the computing time in finding local minimal CV score is more than B-spline. If there is an efficient way to compute matrix inverse, the calculation speed will be much faster. So in the simulation and application studies, we try to optimize our coding to make it run as faster as possible.

Another potential application of V-spline is to vessel monitoring system. The system is a fisheries surveillance that allows environmental and fisheries regulatory organization to track and monitor the activities of fishing vessels. The system calculates the position of the moving object and sends a data report to shore-side users. This information includes time, latitude and longitude positions. However, due to weak signals, the tracking system may lose useful information. The V-spline can help to reconstruct the whole trajectory for a fishery vessel and to analyze its behavior. For example, a larger penalty value indicates stops on the sea inferring that the vessel is casting nets; a smaller penalty value indicates the vessel is moving normally. 

After all, there is a wide range of applications for V-spline in real life. A future work is to implement V-spline on-line for instant estimation and to make it run faster. 



\clearemptydoublepage

\chapter{V-Spline as Bayes Estimate}\label{ChapterGPR}
\input{Chapters/03GPR/GPR_Final}
\clearemptydoublepage

\chapter{An Overview of On-line State and Parameter Estimation}\label{ChapterFR}
\input{Chapters/04Filtering/Filtering_Final}
\clearemptydoublepage

\chapter{Adaptive Sequential MCMC for On-line State and Parameter Estimation}\label{ChapterMCMC}
\section{Introduction}

Data assimilation is a sequential process, by which the observations are incorporated into a numerical model describing the evolution of this system throughout the whole process. The quality of the numerical model determines the accuracy of this system, which requires sequential combined state and parameter inferences. An enormous literature exists on pure state estimation, however, less research has been carried out on combined state and parameter estimation.

\textit{Sequential Monte Carlo} (SMC) has been well studied in scientific literature and have been applied in real world applications. It allows us to specify complex, non-linear time series patterns and enables us to perform real-time Bayesian estimations when it is coupled with \textit{Dynamic Generalized Linear Models} \citep{vieira2016online}. However, model parameters are unknown in real-world applications and this restricts the usefulness of standard SMC. Extensions to standard SMC have been considered by a number of researchers. \cite{kitagawa1998self} propose a self-organizing filter and augmenting the state vector with unknown parameters. The state and parameter are estimated simultaneously by either a non-Gaussian filter or a particle filter. \cite{liu2001combined} propose an improved particle filter to kill degeneracy, which is a common issue in static parameter estimation. They use a kernel smoothing approximation, with a correction factor to account for over-dispersion. Alternatively, \cite{storvik2002particle} propose a new filter algorithm by assuming the posterior depends on a set of sufficient statistics, which can be updated recursively. However, this approach only applies to parameters with conjugate priors \citep{stroud2018bayesian}. Unlike Storvik filter, the Particle learning approach, introduced by \cite{carvalho2010particle}, uses sufficient statistics solely to estimate parameters and promises to reduce particle impoverishment. These particle-like methods use more or less sampling and resampling algorithms to update particles recursively. 

\cite{stroud2018bayesian} propose an SMC algorithm by using ensemble Kalman filter framework for high dimensional space models with observations. Their approach combines information about the parameters from data at different time points in a formal way of Bayesian updating. \cite{polson2008practical} rely on a fixed-lag length of data approximation to filtering and sequential parameter learning in a general dynamic state-space model. This approach allows for sequential parameter learning where importance sampling has difficulties and avoids degeneracies in particle filtering. A new adaptive MCMC method yields a quick and flexible way for estimating posterior distribution in parameter estimation \citep{haario1999adaptive}. This new adaptive proposal method depends on historical data, is introduced to avoid the difficulties of tunning the proposal distribution in Metropolis-Hastings methods. 



A further question is how to find an efficient way to draw samples for $\theta$. There are a few sampling algorithms that have been discussed in literatures, such as importance sampling \citep{hammersley1964percolation, geweke1989bayesian}, rejection sampling \citep{casella2004generalized, martino2010generalized}, Gibbs sampling \citep{geman1984stochastic}, Metropolis-Hastings method \citep{metropolis1953equation, hastings1970monte} and so on. Finally, delayed acceptance MCMC has been used to speed up
computations \citep{payne2018two, quiroz2018speeding}. The main idea in delayed acceptance is to avoid computations if there is an indication that the proposed draw will ultimately be rejected.


In this chapter, an adaptive Delayed-Acceptance Metropolis-Hastings algorithm is proposed to estimate the posterior distribution for combined state and parameter with two phases. In the learning phase, a self-tuning random walk Metropolis-Hastings sampler is used to learn the parameter mean and covariance structure. In the estimation phase, the parameter mean and covariance structure informs the proposed mechanism and is also used in a delayed-acceptance algorithm, which greatly improves sampling efficiency. Information on the resulting state of the system is given by a Gaussian mixture. To keep the algorithm a higher computing efficiency for on-line estimation, it is suggested to cut off historical data and to use a fixed length of data up to the current state, like a window sliding along with time. At the end of this chapter, an application of this algorithm on irregularly sampled GPS time series data is presented. 

%
%For a generalized linear model, one may want to use Kalman Filter \citep{kalman1960new} to filter out the best state estimation from noisy signals, which is known as an optimal estimator returns a minimum mean-square error for linear model \citep{li2004recursibility}. 



\section{Bayesian Inference on Combined State and Parameter}


%In a general state-space model of the following form, either the forward map $F$ in hidden states or the observation transition matrix $G$ is linear or non-linear. We are considering the model 
%\begin{align}\label{MCMCobserY}
%\mbox{Observation:}\hspace*{0.3cm}   & y_t=G(x_t,\theta), \\
%\mbox{Hidden State:}\hspace*{0.3cm} & x_t=F(x_{t-1},\theta),\label{MCMChiddX}
%\end{align}
%where $G$ and $F$ are linear processes with Gaussian white noise $\varepsilon\sim N\left( 0,R(\theta) \right)$ and $\varepsilon'\sim N\left( 0,Q(\theta) \right)$. 

In a general state-space model, a forward map $F$ controls the stochastic evolution of the state $x_t$ and an observation model $G$ connects the observation $y_t$ to the state $x_t$. The goal of inference is to estimate the state of the system and the parameters of the forward map and the observation model. Given a probability for the initial state, $p(x_1\mid \theta)$, where $\theta$ are the parameters to be estimated, and a prior distribution over the parameters, $p(\theta)$, the general Bayesian filtering problem requires computing the posterior distribution of the current state, $p(x_t \mid y_{1:t})$. If we assume that, given $x_{t-1}$, $x_t$ is conditionally independent of states at all other times and all observations, then
\begin{equation}
p(x_t\mid y_{1:t},\theta) = \int p(x_t\mid x_{t-1},\theta)p(x_{t-1}\mid y_{1:t},\theta) dx_{t-1}, 
\end{equation}
where $y_{1:t} = \left\lbrace y_1,\dots,y_t\right\rbrace$ is the observation information up to time $t$. Then given the posterior distribution for the parameters at time $t$, $p(\theta\mid y_{1:t})$, we have 
\begin{align}\label{objecfun}
p(x_t \mid y_{1:t}) = \int p(x_t \mid y_{1:t},\theta)p(\theta\mid y_{1:t})d\theta.
\end{align}
The approach in equation \eqref{objecfun} relies on the two terms: (\romannum{1}) a conditional posterior distribution for the states with given parameters and observations; (\romannum{2}) a marginal posterior distribution for parameter $\theta$. Several methods can be used in finding the second term, such as cross validation, Expectation Maximization algorithm, Gibbs sampling, Metropolis-Hastings algorithm and so on. A Monte Carlo method is popular in research area solving this problem. Monte Carlo method is an algorithm that relies on repeated random sampling to obtain numerical results. To compute an integration of $\int f(x)dx$, one has to sample as many independent $x_i$, $(i = 1,\dots, N)$, as possible and numerically to find $\frac{1}{N}\sum_i f(x_i)$ to approximate the target function. In the target function \eqref{objecfun}, we draw samples of $\theta$ and use a numerical way to calculate its posterior distribution $p(\theta\mid y_{1:t})$. 


Additionally, the marginal posterior distribution for the parameter can be written in two different ways: 
\begin{align}\label{M1}
p(\theta \mid y_{1:t}) &\propto p(y_{1:t}\mid\theta)p(\theta),\\
p(\theta \mid y_{1:t}) &\propto p(y_t\mid y_{1:t-1}, \theta)p(\theta\mid y_{1:t-1}). \label{M2}
\end{align}
The above formula \eqref{M1} is a standard Bayesian inference requiring a prior distribution $p(\theta)$. It can be used in off-line methods, in which $\hat{\theta}$ is inferred by iterating over a fixed observation record $y_{1:t}$. By contrast, formula \eqref{M2} is defined in a recursive way over time depending on the previous posterior at time $t-1$, which is known as on-line method. $\hat{\theta}$ is estimated sequentially as a new observation $y_{t+1}$ becomes available. 


In this chapter, we propose the use of a \textit{linear} state-space model to infer the trajectory of a moving vehicle. Specifically, we suppose that the forward map and the observation model are linear and homogeneous, and the noise is Gaussian. The so-called linear Gaussian state-space model which has been extensively studied in the literature \citep{durbin2012time}. Even in the linear state-space model, parameter and state estimation is difficult. Our goal is to develop a fast and efficient MCMC algorithm for online estimation. 

% cite ref to dlm, state-space....



\subsection{The Posterior Distribution}\label{sectionlogParameter}

For sampling $\theta$, we should find its distribution function first from the covariance matrix of the joint $x_{1:t}$ and $y_{1:t}$. Under the assumption that the forward map and the observation model are linear and homogeneous, the joint distribution of the states and observations is  
\begin{equation}\label{generaljointmatrix}
\begin{bmatrix} \begin{matrix} x_{1:t}\\ y_{1:t}  \end{matrix} \biggr\rvert \theta \end{bmatrix}
\sim N\left(0, \Sigma_t \right),
\end{equation}
where $x_{1:t}$ represents the hidden states $\left\lbrace x_1,\dots,x_t\right\rbrace$, $y_{1:t}$ represents observed $\left\lbrace y_1,\dots,y_t\right\rbrace$ and $\theta$ is a set of all known and unknown parameters. The inverse of the covariance matrix $\Sigma_t^{-1}$ is the precision matrix. In our application, as we will see, it is a block matrix in the form  
\begin{equation} \Sigma_t^{-1}=
\begin{bmatrix}
A_t& -B_t \\ -B_t^\top & B_t
\end{bmatrix}, 
\end{equation}
where $A_t$ is a $t \times t$ matrix coming from the forward map, $B_t$ is a $t\times t$ diagonal matrix coming from the observation model. The structure of the matrices, such as bandwidth, sparse density, depends on the details of the model. %Temporally, we use $A$ and $B$ to stand for the $A_t$ and $B_t$ here. 
Then, we may find the covariance matrix by calculating the inverse of the precision matrix 
\begin{equation}
\begin{split}
\Sigma_t &= \begin{bmatrix}
\left(A_t-B_t^\top B_t^{-1}B_t\right) ^{-1} & -\left(A_t-B_t^\top B_t^{-1}B_t\right)^{-1}B_t^\top B_t^{-1}\\
- B_t^{-1}B_t\left(A_t-B_t^\top B_t^{-1}B_t\right)^{-1} & \left(B_t-B_t^\top A_t^{-1}B_t\right) ^{-1}
\end{bmatrix} \\
&= \begin{bmatrix}
\left(A_t-B_t\right) ^{-1} & \left(A_t-B_t\right)^{-1}\\
\left(A_t-B_t\right)^{-1} & \left(I_t- A_t^{-1}B_t\right) ^{-1}B_t^{-1}
\end{bmatrix} \\
&\triangleq \begin{bmatrix}
\Sigma_{XX} & \Sigma_{XY} \\
\Sigma_{YX}  &\Sigma_{YY} 
\end{bmatrix}.
\end{split}
\end{equation}
Because of the covariance  $\Sigma_{YY} =  \left(I_t-A_t^{-1}B_t\right)^{-1}B_t^{-1}$, therefore the inverse is 
\begin{equation}\label{inverseYY}
\Sigma_{YY}^{-1} = B_t\left(I_t-A_t^{-1}B_t\right)= B_tA_t^{-1}\Sigma_{XX}^{-1}.
\end{equation}
Given the Choleski decomposition $L_tL_t^\top = A_t$, we have
\begin{equation}
\begin{split}
\Sigma_{YY}^{-1} &=B_tL_t^{-\top}L_t^{-1}\Sigma_{XX}^{-1}\\
&=\left(L_t^{-1}B_t\right)^\top\left(L_t^{-1}\Sigma_{XX}^{-1}\right) %\\
%&=\mbox{solve}\left(L,B\right)^\top\mbox{solve}\left(L,\Sigma_{XX}^{-1}\right).
\end{split}
\end{equation}
More usefully, by given another Choleski decomposition $R_tR_t^\top=A_t-B_t=\Sigma_{XX}^{-1}$,
\begin{align}\label{sigmayy01}
%\begin{split}
%Y^\top \Sigma_{YY}^{-1} Y &= \mbox{solve}\left(L,BY\right)^\top\mbox{solve}\left(L,\Sigma_{XX}^{-1}Y\right)\\
%&\triangleq W^\top \mbox{solve}\left(L,\Sigma_{XX}^{-1}Y\right)\\
%\end{split}\\
\begin{split}
y_{1:t}^\top \Sigma_{YY}^{-1} y_{1:t} &= \left(L_t^{-1}B_ty_{1:t}\right)^\top\left(L_t^{-1}\Sigma_{XX}^{-1}y_{1:t}\right)\\
&\triangleq W_t^\top \left(L_t^{-1}\Sigma_{XX}^{-1}y_{1:t}\right)
\end{split}
\end{align}
\begin{equation}\label{sigmayy02}
\begin{split}
\det\Sigma_{YY}^{-1} &= \det B_t \det L_t^{-\top}\det L_t^{-1}\det R_t\det R_t^\top\\
&= \det B_t\left(\det L_t^{-1}\right)^2\left(\det R_t\right)^2.
\end{split}
\end{equation}
From the objective function \eqref{M1}, the posterior distribution of $\theta$ is 
\begin{equation}\label{posteriortheta}
p\left(\theta \mid y_{1:t}\right) \propto p\left(y_{1:t}\mid\theta\right)p\left(\theta\right) \propto \exp\left( -\frac{1}{2} y_{1:t} \Sigma_{YY}^{-1} y_{1:t} \right) \sqrt{\det \Sigma_{YY}^{-1}} p\left(\theta\right).
\end{equation}
Then, by taking natural logarithm on the posterior of $\theta$ and by using the useful solutions in equations \eqref{sigmayy01} and \eqref{sigmayy02}, we will have
\begin{align}\label{logposteriorL}
\ln L\left(\theta\right) &= -\frac{1}{2}y_{1:t}^\top\Sigma_{YY}^{-1}y_{1:t}+\frac{1}{2}\sum\ln\mbox{tr}\left(B_t\right)-\sum\ln\mbox{tr}\left(L_t\right)+\sum\ln\mbox{tr}\left(R_t\right) + \ln p\left(\theta\right).
\end{align}



\subsection{The Forecast Distribution}\label{sectionforecast}

From equation \eqref{M2}, a sequential way for estimating the forecast distribution is needed. Suppose it is 
\begin{equation}
y_{t}\mid y_{1:t-1},\theta \sim N\left( \bar{\mu}_{t},\bar{\sigma}_{t} \right). 
\end{equation}
Look back to the covariance matrices of observations that we found in the previous section 
\begin{equation}
\begin{split}
p(y_{1:t-1},\theta) &= N\left( 0,\Sigma_{YY}^{(t-1)} \right),\\
p(y_{t},y_{1:t-1},\theta) &= N\left( 0,\Sigma_{YY}^{(t)} \right),
\end{split}
\end{equation}
where the covariance matrix of the joint distribution is $\Sigma_{YY}^{(t)} = (I_{t}-A_{t}^{-1}B_{t})^{-1}B_{t}^{-1}$, $I_t$ is a $t\times t$ identity matrix. Then, by taking its inverse, we will get 
\begin{equation}
\begin{split}
\Sigma_{YY}^{(t) (-1)} &= B_{t}(I_{t}-A_{t}^{-1}B_{t}) \\
&= B_{t}(B_{t}^{-1}-A_{t}^{-1})B_{t} \\
&\triangleq \begin{bmatrix} 
B_t & 0 \\ 0 & B_1 \end{bmatrix}
\begin{bmatrix} 
Z_{t} & b_{t} \\
b_{t}^\top & K_{t}
\end{bmatrix} \begin{bmatrix} 
B_t & 0 \\ 0 & B_1\end{bmatrix}
\end{split}
\end{equation}
where $Z_{t}$ is a $t \times t$ matrix, $ b_{t} $ is a $t \times 1$ matrix and $K_{t}$ is a $1 \times 1$ matrix. Thus, by taking its inverse again, we will get 
\begin{equation} \Sigma_{YY}^{(t)}= \left[ \begin{matrix}
B_t^{-1} \left(Z_{t}-b_{t}K_{t}^{-1}b_{t}^\top\right)^{-1}B_t^{-1}  & - B_t^{-1}  Z_{t}^{-1}b_{t}\left(K_{t}-b_{t}^\top Z_{t}^{-1}b_{t}\right)^{-1}B_1^{-1} \\
-B_1^{-1}  K_{t}^{-1}b_{t}^\top \left(Z_{t}-b_{t}K_{t}^{-1}b_{t}^\top\right)^{-1}B_t^{-1}  & B_1^{-1}  \left(K_{t}-b_{t}^\top Z_{t}^{-1}b_{t}\right)^{-1}B_1^{-1} 
\end{matrix}\right].
\end{equation}
So, from the above covariance matrix, we can find the mean and variance of $p\left(y_{t}\mid y_{1:t-1},\theta\right)$ are 
\begin{align}
\bar{\mu}_{t} & =  B_1^{-1}K_{t}^{-1}b_{t}^\top B_{t-1}^{-1}y_{1:t-1} ,\\
\bar{\sigma}_{t}^2 & =  B_1^{-1}K_{t}B_1^{-1}  .
\end{align}




\subsection{The Estimation Distribution}\label{generalEstDistr}

From the joint distribution \eqref{generaljointmatrix}, one can find the best estimation with a given $\theta$ by
\begin{equation}\label{estimationdistribution}
\begin{split}
x_{1:t} \mid y_{1:t},\theta &\sim N \left( A_{t}^{-1}B_{t}y_{1:t}, A_{t}^{-1} \right) \\
&\sim N(L_t^{-\top}L_t^{-1}B_{t}y_{1:t-1},L^{-\top}L_t^{-1})\\
&\sim N(L_t^{-\top}W_t,L^{-\top}L_t^{-1}).
\end{split}
\end{equation}
%Consequently 
%\begin{align}
%x_{1:t} = L_t^{-\top}(W_t+Z_t),
%\end{align}
%where $Z_t \sim N(0, I(\varepsilon)_{t})$ is independent and identically distributed and drawn from a zero-mean normal distribution with variance $I(\varepsilon)_{t}$. 

For sole $x_{t}$, its joint distribution with $y_{1:t}$ is 
\begin{equation}\label{joinedXYgiventheta}
x_{t}, y_{1:t}\mid \theta \sim N\left( 0, \begin{bmatrix}
C_{t}^\top(A_{t}-B_{t}) ^{-1}C_{t} & C_{t}^\top (A_{t}-B_{t})^{-1}\\
(A_{t}-B_{t})^{-1}C_{t} & (I_t- A_{t}^{-1}B_{t}) ^{-1}B_{t}^{-1}
\end{bmatrix} \right),
\end{equation}
where $C_t^\top = \begin{bmatrix}0 & \cdots & 0 & 1\end{bmatrix}$ is $t\times 1$ vector. % that helps to extract the last element in the matrix. 
Thus, the filtering distribution of the state is 
\begin{equation}
x_{t}\mid y_{1:t},\theta \sim N\left( \mu_{t}^{(x)},\Var(x_{t}) \right),
\end{equation}
where, after simplifying, the mean and variance are  
\begin{align}\label{generalmux}
\mu_{t}^{(x)} & = C_{t}^\top A_{t}^{-1}B_{t}y_{1:t} ,\\
\Var(x_{t})& =C_{t}^\top A_{t}^{-1}C_{t}. \label{generalSigx}
\end{align}

Generally, researchers would like to find the combined estimation for $x_t$ and $\theta$ at time $t$ by
\begin{equation}
p(x_t, \theta \mid y_{1:t}) = p(x_t\mid y_{1:t},\theta)p(\theta\mid y_{1:t}).
\end{equation}
Differently, from the target equation \eqref{objecfun}, the state inference containing $N$ samples is a mixture Gaussian distribution in the following form 
\begin{equation}\label{mixtureGaussian}
p(x_t \mid y_{1:t}) = \int p(x_t\mid y_{1:t},\theta) p(\theta\mid y_{1:t})d\theta \dot{=} \frac{1}{N}\sum_{i=1}^{N}p\left(x_{t}\mid\theta^{(i)},y_{1:t}\right). 
\end{equation}
Suppose $x_t\mid y_{1:t},\theta_i \sim N\left( \mu_{ti}^{(x)},\Var(x_{ti}) \right)$ is found from equation \eqref{generalmux} and \eqref{generalSigx} for each $\theta_i$, then its mean is 
\begin{equation}\label{mixturemean}
\mu_t^{(x)} = \frac{1}{N} \sum_i \mu_{ti}^{(x)} 
\end{equation}
and  the unconditional variance of $x_t$, by law of total variance, is 
\begin{equation}\label{mixturevariance}
\begin{split}
\Var(x_t) &= \E\lbrack \Var(x_t\mid y_{1:t},\theta)\rbrack + \Var\lbrack \E(x_t\mid y_{1:t},\theta)\rbrack \\
&= \frac{1}{N} \sum_i \left( \mu_{ti}^{(x)}  \mu_{ti}^{(x)\top} +\Var(x_{ti})\right) -\frac{1}{N^2} \left(  \sum_i  \mu_{ti}^{(x)} \right) \left( \sum_i \mu_{ti}^{(x)} \right) ^\top.
\end{split}
\end{equation}

\section{Random Walk Metropolis-Hastings Algorithm}

Metropolis-Hastings algorithm is an important class of MCMC algorithms \citep{smith1993bayesian, tierney1994markov, gilks1995markov}.  This algorithm has been used extensively in physics but was little known to others until  \cite{muller1991generic, tierney1994markov} expound the value of this algorithm to statisticians. The algorithm is extremely powerful and versatile and has been included in a list of ``The Top 10 Algorithms''  with the greatest influence on the development and practice of science and engineering in the 20th century \citep{dongarra2000guest, medova2008bayesian}. 

Given essentially a probability distribution $\pi(\cdot)$ (the target distribution), MH algorithm provides a way to generate a Markov chain $x_1, x_2,\ldots, x_t$, who has the target distribution as a stationary distribution, for the uncertain parameters $x$ requiring only that this density can be calculated at $x$. Suppose that we can evaluate $\pi(x)$ for any $x$. The transition probabilities should satisfy the detailed balance condition
\begin{equation}
\pi\left(x^{(t)}\right)q\left(x', x^{(t)}\right) = \pi\left(x'\right)q\left(x^{(t)}, x'\right),
\end{equation}
which means that the transition from the current state $\pi(x^{(t)})$ to the new state $\pi(x')$ has the same probability as that from $\pi(x')$ to $\pi(x^{(t)})$. In sampling method, drawing $x_i$ first and then drawing $x_j$ should have the same probability as drawing $x_j$ and then drawing $x_i$. However, in most situations, the details balance condition is not satisfied. Therefore we introduce a function $\alpha(x,y)$ satisfying 
\begin{equation}
\pi\left(x'\right)q\left(x', x^{\left(t\right)}\right)\alpha\left(x',x^{\left(t\right)}\right) = \pi\left(x^{\left(t\right)}\right)q\left(x^{\left(t\right)}, x'\right)\alpha\left(x^{\left(t\right)},x'\right).
\end{equation}
In this way, a tentative new state $x'$ is generated from the proposal density $q\left(x';x^{\left(t\right)}\right)$ and it is accepted or rejected according to acceptance probability 
\begin{equation}\label{alphabalance}
\alpha=\frac{\pi\left(x'\right)}{\pi\left(x^{\left(t\right)}\right)}\frac{q\left(x^{\left(t\right)}, x'\right)}{q\left(x', x^{\left(t\right)}\right)}.
\end{equation}
If $\alpha \geq 1$, the new state is accepted. Otherwise, the new state is accepted with probability $\alpha$.

Here comes an issues of how to choose $q\left(\cdot\mid x^{(t)}\right)$. The most widely used subclass of MCMC algorithms is based on the \textit{random walk Metropolis} (RWM). The RWM updating scheme was first applied by \cite{metropolis1953equation} and proceeds as follows. Given a current value of the $d$-dimensional Markov chain $x^{(t)}$, a new value $x'$ is obtained by proposing a jump $\epsilon = \lvert x' - x^{(t)} \rvert  $ from the pre-specified Lebesgue density 
\begin{equation}\label{stepsizeep}
\tilde{\gamma}\left(\epsilon^\star;\lambda\right) = \frac{1}{\lambda^d}\gamma \left( \frac{\epsilon^\star}{\lambda} \right),
\end{equation}
with $\gamma(\epsilon) = \gamma(-\epsilon)$ for all $\epsilon$. Here, the positive $\lambda$ governs the overall distance of the proposed jump and plays a crucial role in determining the efficiency of any algorithm. In a random walk, the proposal density function $q(\cdot)$ can be chosen for some suitable normal distribution, and hence $q\left(x'\mid x^{\left(t\right)}\right)=N\left(x'\mid x^{\left(t\right)},\epsilon^2\right)$ and $q\left(x^{\left(t\right)}\mid x'\right)=N\left(x^{\left(t\right)}\mid x',\epsilon^2\right)$ cancel in the above equation \eqref{alphabalance} \citep{sherlock2016adaptive}. To decide whether to accept the new state, we compute the the probability of accepting the new state by 
\begin{equation}
\alpha=\min \left\lbrace 1,\frac{\pi\left(x'\right) q\left( x^{\left(t\right)}\mid x'\right) }{\pi\left(x^{\left(t\right)}\right)  q\left( x'\mid x^{\left(t\right)} \right) }  \right\rbrace= \min \left\lbrace 1,\frac{\pi\left(x'\right)  }{\pi\left(x^{\left(t\right)}\right) }  \right\rbrace.
\end{equation}
If the proposed value is accepted it becomes the next current value $x^{(t+1)}= x'$; otherwise the current value is left unchanged $x^{(t+1)} = x^{(t)}$ \citep{sherlock2010random}. 


\subsection{The Self-tuning Metropolis-Hastings Algorithm}

The self-tuning MH algorithm automatically tunes the step sizes for different parameters by one-variable-at-a-time random walk. Aiming at the target acceptance rates for each parameter, the algorithm efficiently and accurately explore the structure of the $d$-dimensional parameter space. 

By assuming the parameters are independent, the idea of this algorithm is that in each iteration, only one parameter is proposed and the others remain to be changed. After the step, take $n$ samples out of the total amount of iterations $N$ as new sequences. In Figure \ref{randomwalk}, examples of different proposing methods are compared. 
\begin{figure}[h]
\centering
 \begin{subfigure}[b]{0.32\textwidth}
 \begin{tikzpicture}
     \node[anchor=south west,inner sep=0] (image) at (0,0) {\includegraphics[width=\textwidth]{Chapters/05MCMCOU/plots/ggoneRW_Final.pdf}};
     \begin{scope}[
         x={(image.south east)},
         y={(image.north west)}
     ]
     \node [black, font=\bfseries] at (0.5,-0.05) {$x$};
     \node [black, font=\bfseries] at (-0.05,0.5) {$y$};
     \end{scope}
 \end{tikzpicture}
  \caption{\footnotesize One-variable-at-a-time random walk.}\label{MCMConevariableRW}
\end{subfigure}
\begin{subfigure}[b]{0.32\textwidth}
\begin{tikzpicture}
      \node[anchor=south west,inner sep=0] (image) at (0,0) {\includegraphics[width=\textwidth]{Chapters/05MCMCOU/plots/ggindRW_Final.pdf}};
      \begin{scope}[
          x={(image.south east)},
          y={(image.north west)}
      ]
      \node [black, font=\bfseries] at (0.5,-0.05) {$x$};
      %\node [black, font=\bfseries] at (-0.05,0.5) {$y$};
      \end{scope}
  \end{tikzpicture}
    \caption{\footnotesize Independent multi-variable-at-a-time random walk.}\label{MCMCMultivariableRW}
\end{subfigure}
\begin{subfigure}[b]{0.32\textwidth}
  \begin{tikzpicture}
      \node[anchor=south west,inner sep=0] (image) at (0,0) {\includegraphics[width=\textwidth]{Chapters/05MCMCOU/plots/ggcorRW_Final.pdf}};
      \begin{scope}[
          x={(image.south east)},
          y={(image.north west)}
      ]
      \node [black, font=\bfseries] at (0.5,-0.05) {$x$};
      %\node [black, font=\bfseries] at (-0.05,0.5) {$y$};
      \end{scope}
  \end{tikzpicture}
  \caption{\footnotesize Correlated multi-variable-at-a-time random walk.}\label{MCMCCorrelatedRW}
\end{subfigure}
\caption{Examples of 2-Dimensional random walk Metropolis-Hastings algorithm. Figure \ref{MCMConevariableRW} is the trace of one-variable-at-a-time random walk. At each time, only one variable is changed and the other one stay constant. Figure \ref{MCMCMultivariableRW} and \ref{MCMCCorrelatedRW} present the traces by multi-variable-at-a-time random walk. In Figure \ref{MCMCMultivariableRW}, the proposal for each step is independent, but in Figure \ref{MCMCCorrelatedRW} the proposal are proposed correlated.}
\label{randomwalk}
\end{figure}

To gain the target acceptance rates $\alpha_i$, $(i = 1, \dots, d)$, the step size $s_i$ for each parameter is tuned automatically. The concept of the algorithm is if the proposal is accepted, we are more confident on the direction and step size that were made. In this scenario, the next moving step should be further. In another word, the step size $s_{t+1}$ in the next step is bigger than $s_t$. Otherwise, a conservative proposal is made with a shorter distance, which is $s_{t+1}\leq s_t$. 

Let $a$ and $b$ be non-negative numbers indicating the distances of a forward movement, the new step size $s_{t+1}$ from current $s_t$ is 
\begin{align}\ln s_{t+1} = 
\begin{cases}
\ln s_t + a & \mbox{with probability } \alpha \\
\ln s_t - b & \mbox{with probability } 1 - \alpha 
\end{cases},
\end{align}
where the logarithm guarantees the step size is positive. 
By taking its expectation  
\begin{align}
\E\lbrack\ln s_{t+1}\mid \ln s_t\rbrack = \alpha(\ln s_t+a) + (1-\alpha)(\ln s_t-b), 
\end{align}
and simplifying to 
\begin{align}
\mu= \alpha(\mu+a) + (1-\alpha)(\mu-b), 
\end{align}
one can find that 
\begin{equation}\label{autostepab}
a = \frac{1-\alpha}{\alpha}  b. 
\end{equation}
Thus, if the proposal is accepted, the step size $s_t$ is tuned to $s_{t+1}=s_te^a$, otherwise $s_{t+1}=s_t/e^b$. 

The complete one-variable-at-a-time MH is summarized in the following: 
\begin{algorithm}[h]
Initialization: Given an arbitrary positive step size $s_i^{(1)}$ for each parameter. Set up a value for $b$ and find $a$ by using the formula \eqref{autostepab}. Set up a target acceptance rate $\alpha_i$ for each parameter, where $i = 1,\dots, d$. \\
Run sampling algorithm: \For{$k$ from 1 to $N$}{
Randomly select a parameter $\theta_i^{(k)}$, propose a new one by $\theta_i'\sim N\left(\theta_i^{(k)}, \epsilon s_i^{(k)}\right)$ and leave the rest unchanged.\label{stRWMHselect}\\
Accept $\theta_i'$ with probability $\alpha=\min\left\lbrace  1,\frac{\pi\left(\theta'\right)q\left(\theta^{\left(k\right)},\theta'\right)}{\pi\left(\theta^{\left(k\right)}\right)q\left(\theta', \theta^{(k)}\right)}  \right\rbrace$. \\
If it is accepted, tune step size to $s_i^{(k+1)}=s_i^{(k)}e^a$, otherwise $s_i^{(k+1)}=s_i^{(k)}/e^b$. \\
Set $k=k+1$ and move to step \ref{stRWMHselect} until $N$.
}
Take $n$ samples out from $N$ with equal spaced index for each parameter being a new sequence. 
\caption{Self-tuning Random Walk Metropolis-Hastings Algorithm}\label{algoonevarible}
%\caption{One-variable-at-a-time Metropolis-Hastings Sampling Algorithm.}
\end{algorithm}


The advantage of the Algorithm \ref{algoonevarible} is that it returns a more accurate estimation for $\theta$ and is more reliable to learn the structure of the parameter space. However, if $\pi(\cdot)$ has an singular structure, the algorithm becomes time-consuming and low efficient. To solve the issue, the \textit{Delayed-Acceptance Metropolis-Hastings} (DA MH) algorithm is utilized to speed up the computation.





\subsection{Adaptive Delayed-Acceptance Metropolis-Hastings Algorithm}

The DA MH algorithm proposed in \citep{christen2005markov} is a two-stage Metropolis-Hastings algorithm in which, typically, proposed parameter values are accepted or rejected at the first stage based on a computationally cheap surrogate $\hat{\pi}(x)$ for the likelihood $\pi(x)$. In stage one, the quantity $\alpha_1$ is found by a standard MH acceptance formula 
\begin{equation}
\alpha_1=\min\left\lbrace  1,\frac{\hat{\pi}(x')q\left(x^{(t)}, x'\right)}{\hat{\pi}(x^{(t)})q\left(x', x^{(t)}\right)}  \right\rbrace ,
\end{equation}
where $\hat{\pi}(\cdot)$ is a cheap estimation for $x$ and a simple form is $\hat{\pi}(\cdot)=N\left(\cdot\mid \hat{x},\epsilon\right)$. Once $\alpha_1$ is accepted, the process goes into stage two and the acceptance probability $\alpha_2$ is
\begin{equation}\label{dahalpha2}
\alpha_2=\min \left\lbrace  1,\frac{\pi(x')\hat{\pi}\left(x^{(t)}\right) }{\pi\left(x^{(t)}\right)\hat{\pi}(x')} \right\rbrace,
\end{equation}
where the overall acceptance probability $\alpha_1\alpha_2$ ensures that detailed balance is satisfied with respect to $\pi(\cdot)$; however if a rejection occurs at stage one, the expensive evaluation of $\pi(x)$ at stage two is unnecessary.

For a symmetric proposal density kernel $q\left(x', x^{(t)}\right)$ such as is used in the random walk MH algorithm, the acceptance probability in stage one is simplified to
\begin{equation} \label{dahalpha1}
\alpha_1= \min \left\lbrace 1,\frac{\pi(x')}{\pi\left(x^{(t)}\right)}  \right\rbrace.
\end{equation}
If the true posterior is available, the delayed-acceptance Metropolis-Hastings algorithm is obtained by substituting this for the unbiased stochastic approximation in \eqref{dahalpha2} \citep{sherlock2015efficiency}.


To accelerate the MH algorithm, DA MH requires a cheap approximate estimation $\hat{\pi}(\cdot)$ in formula \eqref{dahalpha1}. Intuitively, the approximation should be efficient with respect to time and accuracy to the true posterior $\pi(\cdot)$. A sensible option is assuming the parameter distribution at each time $t$ is following a normal distribution with mean $m_t$ and covariance $C_t$. So the posterior density is given by 
\begin{equation}
\hat{\pi}(\theta\mid y_{1:t}) \propto \exp\left( -\frac{1}{2}(\theta-m_t)^\top C_t^{-1}(\theta-m_t)\right). 
\end{equation}
A lazy $C_t$ is chosen as an identity matrix, in which way all the parameters are independent. In terms of $m_t$, in most of circumstances, 0 is not an idea choice. To find an optimal or suboptimal $m_t$ and $C_t$, several algorithms have been discussed. \cite{stroud2018bayesian} use a second-order expansion of $l(\theta)$ at the mode and the mean and covariance become $m_t=\arg \max l(\theta)$ and $C_t = - \left[ \frac{\partial l(\theta)}{\partial \theta_i \partial \theta_j} \right]_{\theta=m_t}^{-1}$ respectively. The drawback of this estimation is a global optimum is not guaranteed. \cite{mathew2012bayesian} propose a fast adaptive MCMC sampling algorithm, which is a consist of two phases. In the learning phase, they use hybrid Gibbs sampler to learn the covariance structure of the variance components. In phase two, the covariance structure is used to formulate an effective proposal distribution for a MH algorithm. 


Likewise, we are suggesting that use a batch of data with length $L<t$ to learn the parameter space by using self-tuning random walk MH algorithm in the learning phase first. This algorithm tunes each parameter at its own optimal step size and explores the surface in different directions. When the process is done, we have a sense of the surface for $\theta\approx\hat{\theta}$ and its mean $\hat{\mu}\approx m_L$ and covariance $\hat{\Sigma}\approx C_L$ can be estimated. Then, we can move to the second phase: DA MH algorithm. The new $\theta'$ is proposed from  $N\left(\theta^{(t)}\mid m_L,C_L\right)$, which is in the following form 
\begin{equation}
\theta' = \theta^{(t)} + R\epsilon z,
\end{equation}
where $R^\top R = C_L$ is the Cholesky decomposition, $\epsilon$ is the tuned step size and $z\sim N(0,1)$ is Gaussian white noise. This proposing method reduces the impact of drawing $\theta'$ from a correlation space. 


%Adaptive Metropolis-Hastings algorithm was introduced in \citep{haario1999adaptive}. Because the choice of a suitable MCMC method and its proposal are crucial for the convergence of the Markov chain. 




\subsection{Efficiency of Metropolis-Hastings Algorithm}\label{effMHA}

In equation \eqref{stepsizeep}, the jump size $\epsilon$ determines the efficiency of RWM algorithm. For a general RWM, it is intuitively clear that we can make the algorithm arbitrarily poor by making $\epsilon$ either very large or very small \citep{sherlock2010random}. Assuming $\epsilon$ is extremely large, the proposal $x'\sim N\left(x^{(t)},\epsilon\right)$, for example, is taken a further distance from current value $x^{(t)}$. Therefore the algorithm will reject most of its proposed moves and stay where it was for a few iterations. On the other hand, if $\epsilon$ is extremely small, the algorithm will keep accepting the proposed $x'$ since $\alpha$ is always approximately be 1 because of the continuity of $\pi(x)$ and $q(\cdot)$ \citep{roberts2001optimal}. Thus, RWM takes a long time to explore the posterior space and converge to its stationary distribution. So, the balance between these two extreme situations must exist. This appropriate step size $\hat{\epsilon}$ is optimal, sometimes is suboptimal, the solution to gain a Markov chain. 

Figure \ref{largesmallstepsize} illustrates the performances of RWM with different $\epsilon$. From these figures, one can see that either too large or too small $\epsilon$ causes high correlation chains, indicating bad samples in sampling algorithm. An appropriate $\epsilon$ decorrelates samples and returns a stationary chain. That is considered highly efficient. 


%\begin{figure}[h]
%\centering
% \begin{subfigure}[b]{0.3\textwidth}
%     \includegraphics[width=\textwidth]{Chapters/05MCMCOU/plots/largechain.pdf}
%     \includegraphics[width=\textwidth]{Chapters/05MCMCOU/plots/largeacf.pdf}
%     \caption{With a large step size.}
%\end{subfigure}
%\begin{subfigure}[b]{0.3\textwidth}
%    \includegraphics[width=\textwidth]{Chapters/05MCMCOU/plots/smallchain.pdf}
%    \includegraphics[width=\textwidth]{Chapters/05MCMCOU/plots/smallacf.pdf}
%    \caption{With a small step size.}
%\end{subfigure}
%\begin{subfigure}[b]{0.3\textwidth} \
%    \includegraphics[width=\textwidth]{Chapters/05MCMCOU/plots/bestchain.pdf}
%    \includegraphics[width=\textwidth]{Chapters/05MCMCOU/plots/bestacf.pdf}
%    \caption{With an appropriate step size.}
%\end{subfigure}
%\caption{Metropolis algorithm sampling for a single parameter with (a) a large step size, (b) a small step size, (c) an appropriate step size. The upper plots show the sample chain and lower plots indicate the autocorrelation for each case.}
%\label{largesmallstepsize}
%\end{figure}


\begin{figure}[h]
\centering
 \begin{subfigure}[b]{0.32\textwidth}
   \includegraphics[width=\textwidth]{Chapters/05MCMCOU/plots/gglargechain_Final.pdf}
    \includegraphics[width=\textwidth]{Chapters/05MCMCOU/plots/gglargeacf_Final.pdf}
     \caption{With a large step size}\label{MCMClargestep}
\end{subfigure}
\begin{subfigure}[b]{0.32\textwidth}
    \includegraphics[width=\textwidth]{Chapters/05MCMCOU/plots/ggsmallchain_Final.pdf}
    \includegraphics[width=\textwidth]{Chapters/05MCMCOU/plots/ggsmallacf_Final.pdf}
    \caption{With a small step size}\label{MCMCsmallstep}
\end{subfigure}
\begin{subfigure}[b]{0.32\textwidth}
    \includegraphics[width=\textwidth]{Chapters/05MCMCOU/plots/ggbestchain_Final.pdf}
    \includegraphics[width=\textwidth]{Chapters/05MCMCOU/plots/ggbestacf_Final.pdf}
    \caption{With a proper step size}\label{MCMCproperstep}
\end{subfigure}
\caption{Metropolis-Hastings sampler for a single parameter with: \ref{MCMClargestep} a large step size, \ref{MCMCsmallstep} a small step size, \ref{MCMCproperstep} an appropriate step size. The upper plots show the sample chains and lower plots indicate the autocorrelation values for each case.}
\label{largesmallstepsize}
\end{figure}


Plenty of work has been done in determining the efficiency of Metropolis-Hastings algorithm in recent years. \cite{gelman1996efficient} work with algorithms consisting of a single Metropolis move (not multi-variable-at-a-time), and obtain many interesting results for the $d$-dimensional spherical multivariate normal problem with symmetric proposal distributions, including that the optimal scale is approximately $2.4/\sqrt{d}$ times the scale of target distribution, which implies optimal acceptance rates of $0.44$ for $d = 1$ and $0.23$ for $d\rightarrow \infty$ \citep{gilks1995markov}. \cite{roberts2001optimal} evaluate scalings that are optimal (in the sense of integrated autocorrelation times) asymptotically in the number of components. They find that an acceptance rate of 0.234 is optimal in many random walk Metropolis situations, but their studies are also restricted to algorithms that consist of only a single step in each iteration, and in any case, they conclude that acceptance rates between 0.15 and 0.5 do not cost much efficiency. Other researchers, such as \citep{roberts1997weak, bedard2007weak, beskos2009optimal, sherlock2009optimal, sherlock2013optimal}, have been tackled for various shapes of target on choosing the optimal scale of the RWM proposal and led to the similar rule: choose the scale so that the acceptance rate is approximately 0.234. Although nearly all of the theoretical results are based upon limiting arguments in high dimension, the rule of thumb appears to be applicable even in relatively low dimensions \citep{sherlock2010random}. 



In terms of the step size $\epsilon$, it is pointed out that for a stochastic approximation procedure, its step size sequence $\left\lbrace \epsilon_i\right\rbrace$ should satisfy $\sum_{i=1}^\infty \epsilon_i=\infty $ and $\sum_{i=1}^\infty \epsilon_i^{1+\lambda}<\infty $ for some $\lambda>0$. The former condition somehow ensures that any point of $X$ can eventually be reached, while the second condition ensures that the noise is contained and does not prevent convergence \citep{andrieu2008tutorial}. \cite{sherlock2010random} tune various algorithms to attain target acceptance rates, and one of the algorithms tunes step sizes of univariate updates to attain the optimal efficiency of Markov chain at the acceptance rates between 0.4 and 0.45. Additionally, \cite{graves2011automatic} mentions that it is certain that one could use the actual arctangent relationship to try to choose a good $\epsilon$: in the univariate example, if $\alpha$ is the desired acceptance rate, then $\epsilon = 2\sigma / \tan \left(\pi/2\alpha\right)$, where $\sigma$ is the posterior standard deviation, will be obtained. In fact, some explorations infer a linear relationship between acceptance rate and step size, which is $\mathtt{logit}(\alpha) \approx 0.76-1.12\ln \epsilon/\sigma$, and the slope of the relationship is nearly equal to the constant -1.12 independently. 

However, in multi-variable-at-a-time RWM, one expects that the proper interpretation of $\sigma$ is not the posterior standard deviation but the average conditional standard deviation, which is presumably more difficult to estimate from a Metropolis algorithm. In a higher $d$-dimensional space, or propose multi-variable-at-a-time, suppose $\Sigma$ is known or could be estimated, then $X'$ can be proposed from $q\sim N\left(X,\epsilon^2\Sigma\right)$. Thus,the optimal step size $\epsilon$ is required. A concessive way of RWM in high dimension is proposing one-variable-at-a-time and treating them as separate one dimensional space individually. In any case, however, the behavior of RWM on a multivariate normal distribution is governed by its covariance matrix $\Sigma$, and it performs better than a fixed $N\left(X,\epsilon^2I_d\right)$ distribution \citep{roberts2001optimal}.


To explore the efficiency of a MCMC process, we introduce some notions first. For an arbitrary square integrable function $g$, \cite{roberts2001optimal} define its \textit{integrated autocorrelation time} by 
\begin{equation}
\tau_g = 1+ 2\sum_{i=1}^{\infty} \mathrm{Corr}\left( g(X_0),g(X_i) \right),
\end{equation}
where $X_0$ is assumed to be distributed according to $\pi$. Because central limit theorem, the variance of the estimator $\bar{g} = \sum_{i=1}^{n}g(X_i)/n$ for estimating $\E\lbrack g(X) \rbrack$ is approximately $\Var_\pi\lbrack g(X)\rbrack \times \tau_g/n$. The variance tells us the accuracy of the estimator $\bar{g}$. The smaller it is, the faster the chain converges. Therefore, they suggest that the efficiency of Markov chains (Eff) can be found by comparing the reciprocal of their integrated autocorrelation time, which is 
\begin{equation}
e_g(\sigma)\propto \left(\tau_g\Var_\pi\lbrack g(X)\rbrack \right)^{-1}. 
\end{equation}
However, the disadvantage of their method is that the measurement of efficiency is highly dependent on the function $g$. Instead, an alternative approach uses \textit{effective sample size} (ESS) \citep{kass1998markov, robert2004monte},  which is defined in \citep{gong2016practical} in the following form of  
\begin{equation}
\mbox{ESS} =  \frac{n}{1+2\sum_{k=1}^{\infty}\rho_k(X)} \approx \frac{n}{1+2\sum_{k=1}^{k_\text{\tiny cut}}\rho_k(X)}= \frac{n}{\tau}, 
\end{equation}
where $n$ is the amount of samples, $k_\mathtt{cut}$ is lag of the first $\rho_k<0.01$  or $0.05$ , and $\tau$ is the integrated autocorrelation time. Given a Markov chain having $n$ iterations, the ESS measures the size of \iid samples with the same standard error. Moreover, a wide support among both statisticians \citep{geyer1992practical} and physicists \citep{sokal1997monte} use the following cost of independent samples to evaluate the performance of MCMC, that is 
\begin{equation}
\frac{n}{\mbox{ESS}}\times \mbox{cost per step} = \tau \times  \mbox{cost per step}.
\end{equation} 

Being inspired by their research, we now define the Efficiency in Unit Time (EffUT)  and ESS in Unit Time (ESSUT) as follows: 
\begin{align}
\mbox{EffUT}   &= \frac{e_g}{T},\\
\mbox{ESSUT} &= \frac{\mbox{ESS}}{T},
\end{align} 
where $T$ represents the computation time, which is also known as running time. The computation time is the length of time, in minutes or hours, etc, required to perform a computational process. The best Markov chain with an appropriate step size $\epsilon$ should not only have a lower correlation, as illustrated in Figure \ref{largesmallstepsize}, but also have less time-consuming. The standard efficiency $e_g$ and ESS do not depend on the computation time, but EffUT and ESSUT do. The best-tuned step size gains the balance between the size of effective proposed samples and cost of time. 




\section{Simulation Studies}

In this section, we consider the model in regular and irregular spaced time difference separately. For an one dimensional state-space model, we consider the hidden state process $\left\lbrace x_t, t\geq 1\right\rbrace$ is a stationary and ergodic Markov process and transited by $F(x'\mid x)$. In this paper, we assume that the state of a system has an interpretation as the summary of the past one-step behavior of the system. The states are not observed directly but by another process $\left\lbrace y_t, t\geq 1\right\rbrace$, which is assumed depending on $\left\lbrace x_t\right\rbrace$ by the process $G(y\mid x)$ only and independent with each other. When observed on discrete time $T_1,\ldots,T_k$, the model is summarized on the directed acyclic following graph  
\begin{align}
\begin{matrix}
\mbox{State}  & x_0     &  \rightarrow& x_1   & \rightarrow \cdots  & x_k  & \rightarrow \cdots & x_t & \rightarrow \cdots\\
          & &       & \downarrow &         &\downarrow &        &\downarrow &   \\
\mbox{Observation}& && y_1               &          & y_k               &        & y_t               &   \\
\mbox{Time } & &       & T_1               &          & T_k               &        & T_t               &   \\
\end{matrix}
\end{align}
We define $\Delta_k = T_k-T_{k-1}$. If $\Delta_t$ is constant, we retrieve a standard  $\textit{AR(1)}$ model process with regular spaced time steps; if $\Delta_t$ is not constant, then the model becomes more complicated with irregular spaced time steps. (Note that we do not consider $x_0$ below: given an appropriate prior, $x_0$ can always be integrated out of the model probability.)

%If the transition processes $F$ and $G$ are linear and normal distributed, we call this model $\textit{Linear Gaussian State-Space Model}$. 


\subsection{Simulation on Regularly Sampled Time Series Data}

If the time steps are evenly spaced, the model can be written as a simple linear homogeneous state-space model: 
\begin{equation}
\begin{split}
y_t\mid x_t      &\sim N\left(x_t,\sigma^2\right) \\
x_t\mid x_{t-1} &\sim N\left(\phi x_{t-1},\tau^2\right),
\end{split}
\end{equation}
where $\sigma$ and $\tau$ are \iid  errors occurring in processes and $\phi$ is a static process parameter in the forward map. An initial value $x_1\sim N(0,L^2)$. 

The joint distribution is $p(x_{1:t},y_{1:t}) = p(y_t\mid x_t)p(x_t\mid x_{t-1})\cdots p(y_1\mid x_1)p(x_1)$. Given the form of the Gaussian density, the joint distribution becomes 
\begin{equation}
\left[ \begin{matrix} x\\y  \end{matrix}\bigg\rvert \theta \right]
\sim N\left(0, \Sigma  \right),
\end{equation}
where $\theta = \left\lbrace \phi,\sigma,\tau\right\rbrace$, and the precision matrix $\Sigma^{-1}$ is  
%\begin{equation}
%\begin{bmatrix}
%\frac{1}{L^2}+\frac{\phi^2}{\tau^2} & \frac{-\phi}{\tau^2} & \cdots & 0 & 0 & 0& \cdots & 0\\
%\frac{-\phi}{\tau^2}   & \frac{1+\phi^2}{\tau^2}+\frac{1}{\sigma^2}& \cdots & 0 & -\frac{1}{\sigma^2} &0 & \cdots & 0 \\
%0 & \frac{-\phi}{\tau^2}   &  \cdots & 0 & 0& -\frac{1}{\sigma^2} & \cdots & 0\\
%\vdots & \vdots & \ddots & \vdots & \vdots & \vdots & \ddots & \vdots \\
%0 & 0   &  \cdots & \frac{1}{\tau^2}+\frac{1}{\sigma^2} & 0 & 0 & \cdots &-\frac{1}{\sigma^2}\\
%0 & -\frac{1}{\sigma^2}  & \cdots & 0 & \frac{1}{\sigma^2} & 0 & \cdots & 0 \\
%0& 0 & \cdots & 0 & 0 &  \frac{1}{\sigma^2} & \cdots & 0\\
%\vdots & \vdots & \ddots & \vdots & \vdots & \vdots & \ddots & \vdots\\
%0 & 0& \cdots &-\frac{1}{\sigma^2} & 0 & 0 & \cdots &  \frac{1}{\sigma^2}
%\end{bmatrix},
%\end{equation}
\begin{equation}
\begin{bmatrix}
\frac{1}{L^2}+\frac{\phi^2}{\tau^2} +\frac{1}{\sigma^2} & -\frac{\phi}{\tau^2}  & \cdots & 0 & 0 & -\frac{1}{\sigma^2}  & 0 & \cdots & 0 & 0 \\
-\frac{\phi}{\tau^2}   & \frac{1+\phi^2}{\tau^2}+\frac{1}{\sigma^2}  & \cdots & 0 & 0 & 0 & -\frac{1}{\sigma^2} & \cdots & 0 & 0 \\
\vdots & \vdots & \ddots & \vdots & \vdots & \vdots & \vdots & \ddots & \vdots & \vdots \\
0 & 0  &  \cdots & \frac{1+\phi^2}{\tau^2} + \frac{1}{\sigma^2} & -\frac{\phi}{\tau^2} & 0 & 0 & \cdots & -\frac{1}{\sigma^2} & 0 \\
0 & 0   &  \cdots & -\frac{\phi}{\tau^2}  & \frac{1}{\tau^2}+\frac{1}{\sigma^2} & 0 & 0 & \cdots & 0 & -\frac{1}{\sigma^2}\\
-\frac{1}{\sigma^2}  & 0 & \cdots & 0 & 0 & \frac{1}{\sigma^2} & 0 & \cdots & 0 & 0\\
0&-\frac{1}{\sigma^2}   & \cdots & 0 & 0 & 0 &  \frac{1}{\sigma^2} & \cdots & 0& 0\\
\vdots & \vdots & \ddots & \vdots & \vdots & \vdots & \vdots & \ddots & \vdots &  \vdots\\
0 & 0& \cdots &-\frac{1}{\sigma^2} & 0 &  0 & 0 & \cdots &  \frac{1}{\sigma^2}  & 0\\ 
0 & 0& \cdots & 0 & -\frac{1}{\sigma^2} & 0 & 0 & \cdots & 0 &  \frac{1}{\sigma^2}
\end{bmatrix},
\end{equation}\label{precisionMatrix}
and denoted as $\Sigma^{-1}\triangleq\begin{bmatrix} A_t & -B_t \\ -B_t & B_t \end{bmatrix}$, 
%Its inverse is the covariance matrix 
%\begin{equation}
%\Sigma=\begin{bmatrix} (A_t-B_t)^{-1} &  (A_t-B_t)^{-1} \\ (A_t-B_t)^{-1} & (I_{t}-A_t^{-1}B_t)^{-1}B_t^{-1} \end{bmatrix} \triangleq \begin{bmatrix}
%\Sigma_{XX} & \Sigma_{XY}  \\ \Sigma_{YX} & \Sigma_{YY} 
%\end{bmatrix},
%\end{equation}
%where $B$ is a $t\times t$ diagonal matrix with elements $\frac{1}{\sigma^2}$. The covariance matrices $\Sigma_{XX} =  \left(A_t-B_t\right)^{-1}$ and $\Sigma_{YY}=\left(I_{t}-A_t^{-1}B_t\right)^{-1}B_t^{-1}$ are found. 
where $B_t$ is a diagonal matrix $\frac{1}{\sigma^2}I$ proportional to an identity matrix. 


\subsubsection*{Parameter Estimation}

In the formula \eqref{M1}, the parameter posterior is estimated with observation $y_{1:t}$. By using the Algorithm \ref{algoonevarible}, although it may take a longer time, we will achieve a precise estimation. Similarly with Section \ref{sectionlogParameter}, from the objective function, the posterior distribution of $\theta$ is 
%\begin{equation}
%p(\theta \mid Y) \propto p(Y\mid\theta)p(\theta) \propto \exp\left( {-\frac{1}{2} Y \Sigma_{YY}^{-1} Y } \right) \sqrt{\det \Sigma_{YY}^{-1}} p(\theta).
%\end{equation}
the same as equation \eqref{posteriortheta}. 
Then, by taking natural logarithm on the posterior of $\theta$ and by using the useful solutions in equations \eqref{sigmayy01} and \eqref{sigmayy02}, we will have
%\begin{equation}\label{linearlogL}
%\ln L(\theta) = -\frac{1}{2}Y^\top\Sigma_{YY}^{-1}Y+\frac{1}{2}\sum\ln\mbox{tr}(B_t)-\sum\ln\mbox{tr}(L_t)+\sum\ln\mbox{tr}(R_t) + \ln p(\theta).
%\end{equation}
the same log-likelihood function \eqref{logposteriorL}.

In a simple linear case, we are choosing the parameter $\theta = \left\lbrace \phi=0.9,\tau^2=0.5,\sigma^2=1\right\rbrace$ as the author did in \citep{lopes2011particle} and using $n=500$ dataset, setting initial $L=0$. Instead of inferring $\tau$ and $\sigma$, we are estimating $\nu_1 = \ln \tau^2$ and $\nu_2 = \ln \sigma^2$ in the RW-MH to avoid singular proposals. After the process, the parameters can be transformed back to original scale. Therefore the new parameter  $\theta^* =  \left\lbrace \phi,\nu_1,\nu_2\right\rbrace = \left\lbrace \phi,\ln\tau^2,\ln\sigma^2\right\rbrace$. 

By using Algorithm \ref{algoonevarible} and aiming the optimal acceptance rate at 0.44, after 10\,000 iterations we get the acceptance rates for each parameters are $\alpha_\phi = 0.4409, \alpha_{\nu_1}= 0.4289$ and $\alpha_{\nu_2}= 0.4505$, and the estimations are $\phi =0.8794, \nu_1= -0.6471$ and $\nu_2= -0.0639$ respectively. Thus, we have the cheap surrogate $\hat{\pi}(\cdot)$. Keep going to the DA MH with another 10\,000 iterations, the algorithm returns the best estimation with $\alpha_1=0.1896$ and $\alpha_2 = 0.8782$. In Figure \ref{linearmarginplots}, the trace plots illustrates that the Markov chain of $\hat{\theta}$ is stably fluctuating around the true $\theta$. 

\begin{figure}[h]
\centering
 \begin{subfigure}[b]{0.32\textwidth}
     \includegraphics[width=\textwidth]{Chapters/05MCMCOU/plots/linear_phi.pdf}
     \caption{Trace plot of $\phi$}
\end{subfigure}
\begin{subfigure}[b]{0.32\textwidth}
    \includegraphics[width=\textwidth]{Chapters/05MCMCOU/plots/linear_tau2.pdf}
     \caption{Trace plot of $\tau^2$}
\end{subfigure}
\begin{subfigure}[b]{0.32\textwidth} \
    \includegraphics[width=\textwidth]{Chapters/05MCMCOU/plots/linear_sig2.pdf}
     \caption{Trace plot of $\sigma^2$}
\end{subfigure}
\caption{Linear simulation with true parameter $\theta = \{ \phi=0.9,\tau^2=0.5,\sigma^2=1\}$. By transforming back to the original scale, the estimation of $\hat{\theta}$ is $\{\phi = 0.8810, \tau^2 = 0.5247, \sigma^2= 0.9416\}$. }
\label{linearmarginplots}
\end{figure}



\subsubsection*{Recursive Forecast Distribution}\label{sectionlinearRecursive}

The calculation of log-posterior of the parameters requires finding out the forecast distribution of $p(y_{1:t}\mid y_{1:t-1},\theta)$. A general way is to use the joint distribution of $y_{t}$ and $y_{1:t-1}$, which is $p(y_{1:t}\mid \theta)\sim N(0,\Sigma_{YY})$, and following the procedure in Section \ref{sectionforecast} to work out the inverse matrix of a multivariate normal distribution. For example, one may find the inverse of the covariance matrix 
\begin{align}
\Sigma_{YY}^{-1} = B_t(I_t-A_t^{-1}B_t) =\frac{1}{\sigma^4}(\sigma^2 I_t-A_t^{-1}) \triangleq \frac{1}{\sigma^4} 
\begin{bmatrix} 
Z_{t} & b_{t} \\
b_{t}^\top & K_{t}
\end{bmatrix},
\end{align}
and the original form of this covariance is 
\begin{align} \Sigma_{YY} =\sigma^4 \begin{bmatrix}
\left(Z_t-b_tK_t^{-1}b_t^\top\right)^{-1} & -Z_t^{-1}b_t\left(K_t-b_t^\top Z_t^{-1}b_t\right)^{-1}\\
-K_t^{-1}b_t^\top \left(Z_t-b_tK_t^{-1}b_t^\top\right)^{-1} & \left(K_t-b_t^\top Z_t^{-1}b_t\right)^{-1}
\end{bmatrix}. 
\end{align}
By denoting $C_{t}^\top = \begin{bmatrix} 0 & \cdots & 0 & 1\end{bmatrix}$ and post-multiplying $\Sigma_{YY}^{-1}$, we will have  
\begin{equation}\label{beforeSMformula}
\Sigma_{YY}^{-1} C_{t}= \frac{1}{\sigma^4}\left(\sigma^2 I_t-A_t^{-1} \right)C_{t}= \frac{1}{\sigma^4} \begin{bmatrix} b_{t} \\ K_{t} \end{bmatrix}.
\end{equation} 


A recursive way of calculating $b_t$ and $K_t$ is to use the Sherman-Morrison-Woodbury formula. In the late 1940s and the 1950s,
%Sherman and Morrison\citep{sherman1950adjustment}, Woodbury \citep{woodbury1950inverting}, Bartlett \citep{bartlett1951inverse} and Bodewig \citep{bodewig1956matrix} 
\cite{sherman1950adjustment, woodbury1950inverting, bartlett1951inverse, bodewig1956matrix} 
discovered the following Theorem \ref{theoremSMW}. The original Sherman-Morrison-Woodbury (for short SMW) formula has been used to consider the inverse of matrices \citep{deng2011generalization}. In this paper, we will consider the more generalized case. 
\begin{theorem}\label{theoremSMW}
(Sherman-Morrison-Woodbury formula). Let $A \in B(H)$ and $G \in B(K)$ both be invertible, and $Y, Z \in B(K, H)$. Then, $A + YGZ^*$ is invertible if and only if $G^{-1} + Z^∗A^{-1}Y$ is invertible. In which case,
\begin{equation}\label{SMWformula}
\left(A+YGZ^*\right)^{-1}= A^{-1}-A^{-1}Y\left(G^{-1}+Z^∗A^{-1}Y\right)^{-1}Z^∗A^{-1}.
\end{equation}
A simple form of SMW formula is Sherman-Morrison formula represented in the following statement \citep{bartlett1951inverse}:
Suppose $A\in R^{n\times n}$ is an invertible square matrix and $u,v\in R^n$ are column vectors. Then, $A+uv\top$ is invertible $\iff 1+u^\top A^{-1}v\neq 0$. If $A+uv\top$ is invertible, then its inverse is given by
\begin{equation}\label{SMformula}
\left(A+uv^{T}\right)^{-1}=A^{-1}-{A^{-1}uv^{T}A^{-1} \over 1+v^{T}A^{-1}u}.
\end{equation}
\end{theorem}

By using the formula \eqref{SMformula}, one can update $K_{t}$ and $b_{t}$  in such a recursive way
\begin{align}\label{linearOUKreg}
K_{t}  &=\frac{\sigma^4}{\tau^2+\sigma^2+\phi^2(\sigma^2-K_{t-1})},\\
b_{t} &= \begin{bmatrix}
\frac{b_{t-1}\phi K_{t}}{\sigma^2} \\ \label{linearOUbreg} \frac{K_{t}(\sigma^2+\tau^2)-\sigma^4 }{\phi\sigma^2}
\end{bmatrix}. 
\end{align}
With the above formula, the recursive way of updating the mean and covariance is in the following formula: 
\begin{align}
\bar{\mu}_{t}       & = \frac{\phi}{\sigma^2}K_{t-1}\bar{\mu}_{t-1} + \phi \left(1 - \frac{K_{t-1}}{\sigma^2}\right)y_{t-1}, \\
\bar{\Sigma}_{t}  &= \sigma^4K_{t}^{-1},
\end{align}
%where $K_1=\frac{\sigma^4}{\sigma^2+\tau^2+L^2\phi^2}$. 
where $K_1=\frac{\sigma^4}{\sigma^2+L^2}$. 
For calculation details, we refer readers to Appendix \ref{linearcalculation}.


\subsubsection*{The Estimation Distribution}

%As discussed in Section \ref{generalEstDistr}, from the joint distribution of $x_{1:t}$ and $y_{1:t}$, one can find the best estimation with a given $\theta$ by
%\begin{align}
%x_{1:t} \mid y_{1:t},\theta \sim N\left(L_t^{-\top}W_t,L_t^{-\top}L_t^{-1}\right),
%\end{align}
%where $W_t = L_t^{-1}B_{t}y_{1:t-1}$. 
%%Consequently 
%\begin{align}
%x_{1:t} = L_t^{-\top}(W_t+Z_t),
%\end{align}
%where $Z_t \sim N\left(0, I(\varepsilon)_{t}\right)$ is independent and identically distributed and drawn from a zero-mean normal distribution with variance $ I(\varepsilon)_t$. Moreover, 
%The mixture Gaussian distribution $p(x_t \mid y_{1:t})$ can be found by 
%\begin{align}
%\mu_t^{(x)} &= \frac{1}{N} \sum_i \mu_{ti}^{(x)} \label{linearmu}  \\
%\Var(x_t) &= \frac{1}{N} \sum_i \left( \mu_{ti}^{(x)}  \mu_{ti}^{(x)\top} +\Var(x_{ti})\right) -\frac{1}{N^2} \left(  \sum_i  \mu_{ti}^{(x)} \right) \left( \sum_i \mu_{ti}^{(x)} \right) ^\top.\label{linearsigma} 
%\end{align}
%To find $\mu_{ti}^{(x)}$ and $\Var(x_{ti})$, we will use the joint distribution of $x_{t}$ and $y_{1:t}$, which is $p(x_{t}, y_{1:t}  \mid  \theta)= N(0,\Gamma)$ and 
%\begin{equation}
%\Gamma=\begin{bmatrix} C_{t}^\top(A_t-B_t)^{-1}C_{t} & C_{t}^\top(A_t-B_t)^{-1}\\(A_t-B_t)^{-1}C_{t} & (I_t-A_t^{-1}B_t)^{-1}B_t^{-1} \end{bmatrix}.
%\end{equation}


As discussed in Section \ref{generalEstDistr}, from the joint distribution of $x_{1:t}$ and $y_{1:t}$, one can find the estimation distribution \eqref{estimationdistribution}, a further joint distribution for $x_t,y_{1:t}$ \eqref{joinedXYgiventheta}, and the mixture Gaussian distribution \eqref{mixtureGaussian} with mean \eqref{mixturemean} and variance \eqref{mixturevariance}. Because of 
\begin{align}
C_{t}^\top A_{t}^{-1} = \begin{bmatrix} - b_{t}^\top & \sigma^2- K_{t} \end{bmatrix},
\end{align}
thus, for any given $\theta$, we have $x_{t}\mid y_{1:t},\theta \sim N\left( \mu_{t}^{(x)},\Var(x_t) \right)$, where
\begin{align}
\mu_{t}^{(x)} &  =  \frac{K_{t}\bar{\mu}_{t}}{\sigma^2}+\left(1-\frac{K_{t}}{\sigma^2}\right)y_{t} \\
\Var(x_t)&= \sigma^2-K_{t}.
\end{align}
By substituting them into the equation \eqref{mixturemean} and \eqref{mixturevariance}, the estimated $x_t$ is obtained. For calculation details, we refer readers to Appendix \ref{linearcalculation}.


\begin{figure}[h]
\centering
\begin{subfigure}[b]{0.45\textwidth}
    \includegraphics[width=\textwidth]{Chapters/05MCMCOU/plots/linearsimuXall.pdf}
     \caption{Estimation of $x_{1:t}$}\label{MCMClinearsimuXall}
\end{subfigure}
\begin{subfigure}[b]{0.45\textwidth}
    %\includegraphics[width=\textwidth]{Chapters/05MCMCOU/plots/gglinearestXt.pdf}
	\includegraphics[width=\textwidth]{Chapters/05MCMCOU/plots/gglinearestXt2.pdf}
     \caption{Estimation of a single $x_t$}\label{MCMClinearsimuXt2}
\end{subfigure}
\caption{Linear simulation of $x_{1:t}$ and single $x_t$.In Figure \ref{MCMClinearsimuXall}, the dots is the true $x_{1:t}$ and the solid line is the estimation $\hat{x}_{1:t}$. In Figure \ref{MCMClinearsimuXt2}, the estimation $\hat{x}_t$ is very close to the true $x$. In fact, the true $x$ falls in the interval $\lbrack \hat{x}-\varepsilon,\hat{x}+\varepsilon\rbrack$.}
\label{linearmarginXt}
\end{figure}



\subsection{Simulation on Irregularly Sampled Time Series Data}

Irregularly sampled time series data is painful for scientists and researchers. In spatial data analysis, several satellites and buoy networks provide continuous observations of wind speed, sea surface temperature, ocean currents, etc. However, data was recorded with irregular time-step, with generally several data each day but also sometimes gaps of several days without any data. \cite{tandeo2011linear} adapt a continuous-time state-space model to analyze this kind of irregular time-step data, in which the state is supposed to be an Ornstein-Uhlenbeck process. 

The OU process is an adaptation of Brownian Motion, which models the movement of a free particle through a liquid and was first developed by \cite{einstein1956investigations}. 
%The Brownian motion is used to construct the Ornstein-Uhlenbeck (OU) process, which has become a popular tool for modeling interest rates and vehicle moving. The derivative of the Brownian motion $x_t$ does not exist at any point in time. Thus, if $x_t$ represents the position of a particle, we might be interested in obtaining its velocity, which is the derivative of the motion. The OU process is an alternative model to the Brownian motion that overcomes the preceding problem. 
By considering the velocity $u_t$ of a Brownian motion at time $t$, over a small time interval, two factors affect the change in velocity: the frictional resistance of the surrounding medium whose effect is proportional to $u_t$ and the random impact of neighboring particles whose effect can be represented by a standard Wiener process. Thus, because mass times velocity equals force, the process in a differential equation form is 
\begin{equation}
mdu_t = -\omega u_tdt+dW_t,
\end{equation}
where $\omega>0$ is called the friction coefficient and $m>0$ is the mass. If we define $\gamma = \omega /m$ and $\lambda = 1/m$, we obtain the OU process \citep{Schobel1999Stochastic},  %\citep{vaughan2015goodness}
which was introduced with the following differential equation:
\begin{equation}
du_t= -\gamma u_tdt+\lambda dW_t.
\end{equation}


The OU process is used to describe the velocity of a particle in a fluid and is encountered in statistical mechanics. It is the model of choice for random movement toward a concentration point. It is sometimes called a continuous-time Gauss Markov process, where a Gauss Markov process is a stochastic process that satisfies the requirements for both a Gaussian process and a Markov process. Because a Wiener process is both a Gaussian process and a Markov process, in addition to being a stationary independent increment process, it can be considered a Gauss-Markov process with independent increments \citep{kijima1997markov}. 

To apply OU process on irregularly sampled data, we assume that the latent process $\left\lbrace x_{1:t}\right\rbrace$ is a simple OU process, that is a stationary solution of the following stochastic differential equation : 
\begin{equation}\label{linearOUequation}
dx_t= -\gamma x_tdt+\lambda dW_t, 
\end{equation}
where $W_t$ is a standard Brownian motion, $\gamma>0$ represents the slowly evolving transfer between two neighbor data and $\lambda$ is the forward transition variability. It is not hard to find the solution of equation \eqref{linearOUequation} is 
\begin{equation}
x_t = x_{t-1}e^{-\gamma t} +\int_{0}^{t} \lambda e^{-\gamma (t-s)}dW_s. 
\end{equation}
For any arbitrary time step $t$, the general form of the process satisfies 
\begin{equation}
x_t = x_{t-1}e^{-\gamma \Delta_t} + \tau,
\end{equation}
where $\Delta_t = T_t-T_{t-1}$ is the time difference between two consecutive data points, $\tau$ is a Gaussian white noise with mean zero and variances $\frac{\lambda^2}{2\gamma}\left(1-e^{-2\gamma\Delta_t}\right)$. 

The observed $y_{1:t}$ is measured by 
\begin{equation}
y_t = Hx_t + \varepsilon,
\end{equation}
where $\varepsilon\sim N(0,\sigma)$ is a Gaussian white noise. 

To run the simulations, we generate an irregular time-lag sequence $\left\lbrace \Delta_t\right\rbrace$ first from an \textit{Inverse Gamma} distribution with parameters $\alpha=2, \beta=0.1$. Then, the following parameters were chosen for the numerical simulation: $\gamma = 0.5$, $\lambda^2 = 0.1$, $\sigma^2=1$. 



\begin{figure}[h]
\centering
\includegraphics[width=0.45\textwidth,height=5cm]{Chapters/05MCMCOU/plots/simudataOUdataview.pdf}
\includegraphics[width=0.45\textwidth,,height=5cm]{Chapters/05MCMCOU/plots/simudataOUDelThist2.pdf}
\caption{Simulated data. The solid dots indicate the true state $x$ and cross dots indicate observation $y$. Irregular time lag $\Delta_t$ are generated from \textit{Inverse Gamma}(2,0.1) distribution.}
\label{simuOUreview}
\end{figure}


Similarly, we can get the joint distribution for $x_{1:t}$ and $y_{1:t}$ 
%\begin{equation}
%\begin{bmatrix} \begin{matrix} x\\y \end{matrix} \bigg\rvert \theta \end{bmatrix}
%\sim N\left(0, \Sigma  \right)
%\end{equation}
%from the precision matrix 
%\begin{equation}
%\begin{bmatrix}
%\frac{1}{L^2}+\frac{\phi_1^2}{\tau_1^2} +\frac{1}{\sigma^2}& \frac{-\phi_1}{\tau_1^2} & \cdots & 0 & 0 & 0& \cdots & 0\\ 
%\frac{-\phi_1}{\tau_1^2}   &\frac{1}{\tau_1^2}+\frac{\phi_2^2}{\tau_2^2}+\frac{1}{\sigma^2}& \cdots & 0 & -\frac{1}{\sigma^2} &0 & \cdots & 0 \\
%0 & \frac{-\phi_2}{\tau_2^2}   &  \cdots & 0 & 0& -\frac{1}{\sigma^2} & \cdots & 0\\
%\vdots & \vdots & \ddots & \vdots & \vdots & \vdots & \ddots & \vdots \\
%0 & 0   &  \cdots & \frac{1}{\tau_t^2}+\frac{1}{\sigma^2} & 0 & 0 & \cdots &-\frac{1}{\sigma^2}\\
%0 & -\frac{1}{\sigma^2}  & \cdots & 0 & \frac{1}{\sigma^2} & 0 & \cdots & 0 \\
%0& 0 & \cdots & 0 & 0 &  \frac{1}{\sigma^2} & \cdots & 0\\
%\vdots & \vdots & \ddots & \vdots & \vdots & \vdots & \ddots & \vdots\\
%0 & 0& \cdots &-\frac{1}{\sigma^2} & 0 & 0 & \cdots &  \frac{1}{\sigma^2}
%\end{bmatrix}
%\end{equation}
having a similar precision matrix shown in equation \eqref{precisionMatrix}, where $\phi_t = e^{-\gamma\Delta_t}, \tau^2_t = \frac{\lambda^2}{2\gamma}\left(1-e^{-2\gamma\Delta_t}\right)$, $\theta$ represents for the unknown parameters. 
%Denoted by $\Sigma^{-1}=\begin{bmatrix} A_t & -B_t \\ -B_t & B_t\end{bmatrix}$, covariance matrix is 
%\begin{equation}
%\Sigma=\begin{bmatrix} \left(A_t-B_t\right)^{-1} &  \left(A_t-B_t\right)^{-1} \\ \left(A_t-B_t\right)^{-1} & \left(I-A_t^{-1}B_t\right)^{-1}B_t^{-1} \end{bmatrix} \triangleq \begin{bmatrix}
%\Sigma_{XX} & \Sigma_{XY}  \\ \Sigma_{YX} & \Sigma_{YY} 
%\end{bmatrix},
%\end{equation}
%where $B_t$ is a $t\times t$ diagonal matrix with elements $\frac{1}{\sigma^2}$. The covariance matrices $\Sigma_{XX} =  \left(A_t-B_t\right)^{-1}$ and $\Sigma_{YY} =  \left(I-A_t^{-1}B_t\right)^{-1}B_t^{-1}$. 



\subsubsection*{Parameter Estimation}

%To use the Algorithm \ref{algoonevarible}, similarly with Section \ref{sectionlogParameter}, we need to find the posterior distribution of $\theta$ with observations $y_{1:t}$ fist, which in fact is 
%\begin{equation}
%p(\theta \mid Y) \propto p(Y\mid\theta)p(\theta) \propto \exp\left( -\frac{1}{2} Y \Sigma_{YY}^{-1} Y \right) \sqrt{\det \Sigma_{YY}^{-1}} p(\theta).
%\end{equation}
%By taking natural logarithm on the posterior of $\theta$ and by using the useful solutions in equations \eqref{sigmayy01} and \eqref{sigmayy02}, we have 
%\begin{equation}\label{simuOUlogL}
%\ln L(\theta) = -\frac{1}{2}Y^\top\Sigma_{YY}^{-1}Y+\frac{1}{2}\sum\ln\mbox{tr}(B_t)-\sum\ln\mbox{tr}(L_t)+\sum\ln\mbox{tr}(R_t) + \ln p(\theta).
%\end{equation}


By implementing he Algorithm \ref{algoonevarible}, similarly with Section \ref{sectionlogParameter}, from the objective function, the posterior distribution of $\theta$ is the same as equation \eqref{posteriortheta}. By taking natural logarithm on the posterior of $\theta$ and by using the useful solutions in equations \eqref{sigmayy01} and \eqref{sigmayy02}, we have the same log-likelihood function \eqref{logposteriorL}.

Because of all parameters are positive, we are estimating $\nu_1=\ln\lambda$, $\nu_2=\ln\gamma^2$ and $\nu_3=\ln\sigma^2$ instead. When the estimation process is done, we can transform them back to the original scale by taking exponential. 

After running the whole process, it gives us the best estimation of $\hat{\theta}$ is
$\{ \gamma=0.4841, \lambda^2=0.1032, \sigma^2=0.9276\}$. In Figure \ref{simuOUmarginplots}, we can see that the $\theta$ chains are skew to the true value with tails.
\begin{figure}[h]
\centering
 \begin{subfigure}[b]{0.3\textwidth}
     \includegraphics[width=\textwidth]{Chapters/05MCMCOU/plots/simudataOUtracegam.pdf}
     \caption{Trace plot of $\gamma$}
\end{subfigure}
\begin{subfigure}[b]{0.3\textwidth}
    \includegraphics[width=\textwidth]{Chapters/05MCMCOU/plots/simudataOUtracelab2.pdf}
     \caption{Trace plot of $\lambda^2$}
\end{subfigure}
\begin{subfigure}[b]{0.3\textwidth}
    \includegraphics[width=\textwidth]{Chapters/05MCMCOU/plots/simudataOUtracesig2.pdf}
     \caption{Trace plot of $\sigma^2$}
\end{subfigure}
\caption{Irregular time step OU process simulation. The estimation of $\hat{\theta}$ is $\{\gamma=0.4841, \lambda^2=0.1032, \sigma^2=0.9276\}$. In the plots, the horizontal dark lines are the true $\theta$. }
\label{simuOUmarginplots}
\end{figure}



\subsubsection*{Recursive Calculation and State Estimation}

Follow the procedure in Section \ref{sectionforecast} and do similar calculation with Section \ref{sectionlinearRecursive}, one can find the following recursive way to update $K_{t}$ and $b_{t}$: 
\begin{align} \label{linearOUKirreg}
K_{t}  &=\frac{\sigma^4}{\tau_t^2+\sigma^2+\phi_t^2(\sigma^2-K_{t-1})},\\
b_{t} &= \begin{bmatrix}
\frac{b_{t-1}\phi_t K_{t}}{\sigma^2} \\\label{linearOUbirreg} \frac{K_{t}(\sigma^2+\tau_t^2)-\sigma^4 }{\phi_t\sigma^2}
\end{bmatrix}. 
\end{align}
With the above formula, the recursive way of updating the mean and covariance are 
\begin{align} \label{linearOUmu}
\bar{\mu}_{t}       & = \frac{\phi_t}{\sigma^2}K_{t-1}\bar{\mu}_{t-1} + \phi_t \left(1 - \frac{K_{t-1}}{\sigma^2}\right)y_{t-1}, \\
\bar{\Sigma}_{t}  &= \sigma^4K_{t}^{-1}, \label{linearOUsigma}
\end{align}
%where $K_1=\frac{\sigma^4}{\sigma^2+\tau_1^2+L^2\phi_1^2}$. 
where $K_1=\frac{\sigma^4}{\sigma^2+L^2}$. 
%
%Additionally, as discussed in Section \ref{generalEstDistr}, the best estimation of $x_{1:t}$ with a given $\theta$ is 
%\begin{align}
%x_{1:t} \mid y_{1:t},\theta \sim N\left(L_t^{-\top}W_t,L_t^{-\top}L_t^{-1}\right),
%\end{align}
%where $W_t = L_t^{-1}B_{t}y_{1:t-1}$, and the mixture Gaussian distribution for $p(x_t \mid y_{1:t})$ is 
%\begin{align}
%\mu_t^{(x)} &= \frac{1}{N} \sum_i \mu_{ti}^{(x)}  \\
%\Var(x_t) &= \frac{1}{N} \sum_i \left( \mu_{ti}^{(x)}  \mu_{ti}^{(x)\top} +\Var(x_{ti})\right) -\frac{1}{N^2} \left(  \sum_i  \mu_{ti}^{(x)} \right) \left( \sum_i \mu_{ti}^{(x)} \right) ^\top.
%\end{align}


With a given $\theta$, the estimation is $x_{t}\mid y_{1:t},\theta \sim N\left(\mu_{t}^{(x)},\Var(x_t)\right)$, where
\begin{align}\label{linearOUmean}
\mu_{t}^{(x)} &  =  \frac{K_{t}\bar{\mu}_{t}}{\sigma^2}+\left(1-\frac{K_{t}}{\sigma^2}\right)y_{t} \\ \label{linearOUvar}
\Var(x_t)&= \sigma^2-K_{t}.
\end{align}
By substituting them into the equation \eqref{mixturemean} and \eqref{mixturevariance}, the estimated $x_t$ is obtained. 

Notice that the difference between equations \eqref{linearOUKreg}\eqref{linearOUbreg} and equations  \eqref{linearOUKirreg} \eqref{linearOUbirreg} is that in the latter ones the parameters are dependent on the time lag $\Delta_t$. 

\begin{figure}[h]
\centering
\begin{subfigure}[h]{0.45\textwidth}
\includegraphics[width=\textwidth]{Chapters/05MCMCOU/plots/simudataOUallX.pdf}
\caption{Batch method of estimating $x_{1:t}$}\label{MCMCOUallX}
\end{subfigure}
\begin{subfigure}[h]{0.45\textwidth}
%    \includegraphics[width=\textwidth]{Chapters/05MCMCOU/plots/simudataOUXt.pdf}
\includegraphics[width=\textwidth]{Chapters/05MCMCOU/plots/simudataOUXt2.pdf}
\caption{Sequential method of estimating $x_t$}\label{MCMCOUallXt2}
\end{subfigure}
\caption{Irregular time step OU process simulation of $x_{1:t}$ and sole $x_t$. In Figure \ref{MCMCOUallX}, the dots is the true $x_{1:t}$ and the solid line is the estimation $\hat{x}_{1:t}$. In Figure \ref{MCMCOUallXt2}, the chain in solid line is the estimation $\hat{x}_t$; dotted line is the true value of $x$; dot-dash line on top is the observed value of $y$; dashed lines are the estimated error. }
\label{simuOUxt}
\end{figure}




\section{High Dimensional Ornstein-Uhlenbeck Process Application}\label{SectionHighDimensionalOU}

Tractors moving on an orchard are mounted with GPS units, which are recording data and transfer to the remote server. This data infers longitude, latitude, bearing, etc, with unevenly spaced time mark. However, one dimensional OU process containing either only position or velocity is not enough to infer a complex movement. 

\begin{figure}[h]
\centering
\begin{tikzpicture}
     \node[anchor=south west,inner sep=0] (image) at (0,0) {\includegraphics[width=0.45\textwidth]{Chapters/05MCMCOU/plots/realdatapath_Final.pdf}};
     \begin{scope}[
         x={(image.south east)},
         y={(image.north west)}
     ]
     \node [black, font=\bfseries] at (0.5,-0.05) {Easting};
     \node [black, font=\bfseries,rotate=90] at (-0.05,0.5) {Northing};
     \end{scope}
\end{tikzpicture}
\begin{tikzpicture}
     \node[anchor=south west,inner sep=0] (image) at (0,0) {\includegraphics[width=0.45\textwidth]{Chapters/05MCMCOU/plots/realdatahistdeltaT_Final.pdf}};
     \begin{scope}[
         x={(image.south east)},
         y={(image.north west)}
     ]
     \node [black, font=\bfseries] at (0.5,-0.05) {Lag};
     %\node [black, font=\bfseries,rotate=90] at (-0.05,0.5) {Frequency};
     \end{scope}
\end{tikzpicture}
\caption{Demonstration of line-based trajectory of a moving tractor. The time lags (right side figure) obtained from GPS units are irregular.}
\label{realdatareview}
\end{figure}

In this section, we are introducing an Ornstein-Uhlenbeck process (OU-process) model combing both position and velocity with the following equations  
\begin{equation}\label{OUprocess}
\begin{cases}
du_t = -\gamma u_t dt+ \lambda dW_t,\\
dx_t = u_t dt+\xi dW_t'.
\end{cases}
\end{equation}
The solution can be found by integrating $dt$ out, that gives us 
\begin{align}
\begin{cases}
u_t =u_{t-1}e^{-\gamma t} +\int_{0}^{t} \lambda e^{-\gamma (t-s)}dW_s,\\
x_t =x_{t-1} +\frac{u_{t-1}}{\gamma}\left(1- e^{-\gamma t}\right) + \int_{0}^{t} \frac{\lambda}{\gamma}e^{\gamma  s} \left(1-e^{-\gamma t}\right)dW_s + \int_{0}^{t}\xi dW_s'.
\end{cases}
\end{align}
As a result, the joint distribution is 
\begin{align}
\begin{bmatrix} x_t \\ u_t \end{bmatrix} &\sim N\left(
\begin{bmatrix}\mu_t^{(x)} \\ \mu_t^{(u)}  \end{bmatrix} , 
\begin{bmatrix}
\sigma_t^{(x)2} & \rho_t\sigma_t^{(x)} \sigma_t^{(u)} \\
\rho_t\sigma_t^{(x)} \sigma_t^{(u)} & \sigma_t^{(u)2}
\end{bmatrix} \right),
\end{align}
where $\mu_t^{(x)}$ and $\mu_t^{(u)} $ are from the forward map process 
\begin{align}
\begin{bmatrix}\mu_t^{(x)} \\ \mu_t^{(u)}  \end{bmatrix}  = 
\begin{bmatrix}
1 & \frac{1-e^{-\gamma \Delta_t}}{\gamma} \\ 0 &  e^{-\gamma \Delta_t}
\end{bmatrix}  \begin{bmatrix} x_{t-1}^{(x)} \\ u_{t-1}  \end{bmatrix} \triangleq \Phi \begin{bmatrix} x_{t-1}^{(x)} \\ u_{t-1}  \end{bmatrix},
\end{align}
and 
\begin{align}
\begin{cases}
\sigma_t^{(x)2} &=\frac{\lambda^2 \left(e^{2 \gamma\Delta_t}-1\right) \left(1 -e^{-\gamma\Delta_t}\right)^2}{2 \gamma ^3 } + \xi^2\Delta_t\\
\sigma_t^{(u)2} &= \frac{\lambda ^2 \left(1- e^{-2 \gamma\Delta_t}\right)}{2 \gamma } \\
\rho_t\sigma_t^{(x)}\sigma_t^{(u)} & =\frac{\lambda ^2 \left(e^{\gamma\Delta_t} -1\right) \left(1-e^{-2\gamma\Delta_t}\right)}{2 \gamma ^2}
\end{cases}
\end{align}
In the above equations $\Delta_t = T_t-T_{t-1}$ and initial values are $\Delta_1=0$, %$x_0\sim N\left(0,L_x^2\right), u_0\sim N(0,L_u^2)$, %% was x_0
$x_1\sim N\left(0,L_x^2\right), u_1\sim N(0,L_u^2)$, 
$\rho_t^2 = 1-\frac{\xi^2 \Delta_t}{\sigma_t^{(x)^2}}$. To be useful, we use $\frac{1}{1-\rho_t^2} =\frac{\sigma_t^{(x)2}}{\xi^2 \Delta_t}$ in the calculation. 

Furthermore, the independent observation process is 
\begin{equation}\label{obseq}
\begin{cases} y_t=x_t+\varepsilon_t,\\ v_t=u_t+\varepsilon'_t, \end{cases} 
\end{equation}
where $\varepsilon_t\sim N(0,\sigma),\varepsilon'_t\sim N(0,\tau)$ are normally distributed independent errors. Thus, the joint distribution of observations is 
\begin{align}\label{obmodel}
\begin{bmatrix} y_t \\ v_t \end{bmatrix} &\sim N\left(
\begin{bmatrix}x_t \\ u_t \end{bmatrix} , 
\begin{bmatrix}
\sigma^2 & 0\\
0 & \tau^2
\end{bmatrix} \right).
\end{align}
Consequently, the parameter $\theta$ of an entire Ornstein-Uhlenbeck process is a set of five parameters from both hidden status and observation process, which is represented as $\theta = \left\lbrace \gamma,\xi^2,\lambda^2,\sigma^2,\tau^2 \right\rbrace$. 


Starting from the joint distribution of $x_{0:t},u_{0:t}$ and $y_{1:t},v_{1:t}$ by given $\theta$, it can be found that
\begin{equation}\label{jointmatrix}
\begin{bmatrix} \begin{matrix} \tilde{X}\\ \tilde{Y}  \end{matrix} \biggr\rvert \theta \end{bmatrix}
\sim N\left(0, \tilde{\Sigma} \right),
\end{equation}
where $\tilde{X}$ represents for the hidden statues $\left\lbrace x,u\right\rbrace$, $\tilde{Y}$ represents for observed $\left\lbrace y,v\right\rbrace$, $\theta$ is the set of five parameters.  The inverse of the covariance matrix $\tilde{\Sigma}^{-1}$ is the precision matrix in the form of
\begin{align} \tilde{\Sigma}^{-1}=
\begin{bmatrix}
Q_{xx} & Q_{xu} & -\frac{1}{\sigma^2}I & 0\\
Q_{ux} & Q_{uu} & 0 &-\frac{1}{\tau^2}I \\
-\frac{1}{\sigma^2}I & 0 & \frac{1}{\sigma^2}I  & 0\\
 0  &  -\frac{1}{\tau^2}I  & 0 & \frac{1}{\tau^2}I 
\end{bmatrix}.
\end{align}
To make the covariance matrix a more beautiful form and convenient computing, $\tilde{X}$, $\tilde{Y}$ and $\tilde{\Sigma}$ can be rearranged in a time series order, that makes $X_{1:t} = \left\lbrace x_1,u_1,x_2,u_2,\ldots, x_t, u_t \right\rbrace$, $Y_{1:t} = \left\lbrace y_1,v_1,y_2,v_2,\ldots, y_t, v_t \right\rbrace$ and the new precision matrix $\Sigma^{-1}$ looks like 
\begin{align} \Sigma^{-1} &=
\begin{bmatrix}
\sigma_{11}^{(x)2}+\frac{1}{\sigma^2} & \sigma_{11}^{(xu)2} & \cdots & \sigma_{1t}^{(x)2} & \sigma_{1t}^{(xu)2}  &  -\frac{1}{\sigma^2} & 0 & \cdots & 0 & 0\\
\sigma_{11}^{(ux)2}   & \sigma_{11}^{(u)2} +\frac{1}{\tau^2} & \cdots & \sigma_{1t}^{(ux)2} & \sigma_{1t}^{(x)2}  &  0 & -\frac{1}{\tau^2} & \cdots & 0 & 0 \\
\vdots & \vdots & \ddots & \vdots & \vdots & \vdots & \vdots &\ddots & \vdots & \vdots \\
\sigma_{t1}^{(x)2}   & \sigma_{t1}^{(xu)2} & \cdots & \sigma_{tt}^{(x)2} +\frac{1}{\sigma^2}  & \sigma_{tt}^{(xu)2}  &  0 & 0 & \cdots & -\frac{1}{\sigma^2} & 0 \\
\sigma_{t1}^{(ux)2}   & \sigma_{t1}^{(u)2} & \cdots & \sigma_{tt}^{(ux)2} & \sigma_{tt}^{(u)2} +\frac{1}{\tau^2}  &  0 & 0 & \cdots & 0 &-\frac{1}{\tau^2} \\
- \frac{1}{\sigma^2} & 0 & \cdots & 0 & 0 &  \frac{1}{\sigma^2} & 0 & \cdots & 0 & 0\\
0  & -\frac{1}{\tau^2}& \cdots & 0 & 0 &  0 &  \frac{1}{\tau^2} & \cdots & 0 & 0 \\
\vdots & \vdots & \ddots & \vdots & \vdots & \vdots & \vdots &\ddots & \vdots & \vdots \\
0 & 0& \cdots & -\frac{1}{\sigma^2}  &0&  0 & 0 & \cdots & \frac{1}{\sigma^2} & 0 \\
0 & 0 & \cdots & 0 & -\frac{1}{\tau^2}   &  0 & 0 & \cdots & 0 & \frac{1}{\tau^2}
\end{bmatrix} \\ 
& \triangleq \begin{bmatrix}
A_t& -B_t \\ -B_t^\top & B_t
\end{bmatrix},
\end{align}
where $B_t$ is a $2t\times 2t$ diagonal matrix of observation errors at time $t$ in the form of $\begin{bmatrix}
\frac{1}{\sigma^2}& \cdot & \cdot &  \cdot  &  \cdot \\  \cdot & \frac{1}{\tau^2} & \cdot &  \cdot  &  \cdot  \\ 
\vdots & \vdots & \ddots & \vdots & \vdots \\
 \cdot  &  \cdot  & \cdot  & \frac{1}{\sigma^2}&  \cdot \\  \cdot  &  \cdot & \cdot  &  \cdot  & \frac{1}{\tau^2}
\end{bmatrix}$. 
In fact, the matrix $A_t$ is a $2t \times 2t$ bandwidth six sparse matrix at time $t$ in the process. Then, we may find the covariance matrix by calculating the inverse of the precision matrix as 
\begin{align}
\Sigma & %=\begin{bmatrix}
%\left(A_t-B_t^\top B_t^{-1}B_t\right) ^{-1} & -\left(A_t-B_t^\top B_t^{-1}B_t\right)^{-1}B_t^\top B_t^{-1}\\
%- B_t^{-1}B_t\left(A_t-B_t^\top B_t^{-1}B_t\right)^{-1} & \left(B_t-B_t^\top A_t^{-1}B_t\right) ^{-1}
%\end{bmatrix} 
%&= \begin{bmatrix}
%\left(A_t-B_t\right) ^{-1} & \left(A_t-B_t\right)^{-1}\\
%\left(A_t-B_t\right)^{-1} & \left(I_t- A_t^{-1}B_t\right) ^{-1}B_t^{-1}
%\end{bmatrix} \\
\triangleq \begin{bmatrix}
\Sigma_{XX} & \Sigma_{XY} \\
\Sigma_{YX}  &\Sigma_{YY} 
\end{bmatrix}.
\end{align}
A detailed structure of the covariance matrix $\Sigma_{XX} $ is presented in Appendix \ref{covMatrixdetails}. 

\subsection{The Posterior Distribution}

To find the log-posterior distribution of $X_{1:t}$ and $Y_{1:t}$, we start from the joint distribution. Similarly, the inverse of the covariance matrix is 
%\begin{align}
%\Sigma_{YY}^{-1} &= B_t(I_t-A_t^{-1}B_t)= B_tA_t^{-1}\Sigma_{XX}^{-1}.
%\end{align}
the same as equation \eqref{inverseYY}. Additionally, the posterior distribution and log-likelihood function are the same form as equations \eqref{posteriortheta} and \eqref{logposteriorL}. 
%By using Choleski decomposition and similar technical solution, second term in the integrated objective function is 
%\begin{align}
%p(\theta \mid Y) &\propto p(Y\mid\theta)p(\theta) \propto \exp\left( -\frac{1}{2} Y \Sigma_{YY}^{-1} Y \right) \sqrt{\det \Sigma_{YY}^{-1}} P(\theta).
%\end{align}
%Then, by taking natural logarithm on the posterior of $\theta$ and by using the useful solutions in equations \eqref{sigmayy01} and \eqref{sigmayy02}, we will have
%\begin{align}\label{logL}
%\ln L(\theta) &= -\frac{1}{2}Y^\top\Sigma_{YY}^{-1}Y+\frac{1}{2}\sum\ln\mbox{tr}(B_t)-\sum\ln\mbox{tr}(L_t)+\sum\ln\mbox{tr}(R_t).
%\end{align}




\subsection{The Forecast Distribution}

It is known that 
\begin{align}
p(Y_{1:t-1},\theta) &= N\left( 0,\Sigma_{YY}^{(t-1)} \right)\\
p(Y_{t},Y_{1:t-1},\theta) &= N\left( 0,\Sigma_{YY}^{(t)} \right)\\
p(Y_{t}\mid Y_{1:t},\theta) &= N\left( \bar{\mu}_{t},\bar{\Sigma}_{t} \right)
\end{align}
where the covariance matrix of the joint distribution is $\Sigma_{YY}^{(t)} = \left(I_{t}-A_{t}^{-1}B_{t}\right)^{-1}B_{t}^{-1}$. Then, by taking its inverse, one can obtain 
\begin{align}
\Sigma_{YY}^{(t) (-1)} = B_{t}(I_{t}-A_{t}^{-1}B_{t}).
\end{align}
To be clear, the matrix $B_{t}$ is short for the matrix $B_{t}(\sigma^2,\tau^2)$, which is $2t\times 2t$ diagonal matrix with elements $\frac{1}{\sigma^2},\frac{1}{\tau^2}$ repeating for $t$ times on its diagonal. For instance, the very simple $B_1(\sigma^2,\tau^2) = 
\begin{bmatrix}
\frac{1}{\sigma^2} & 0  \\
0 & \frac{1}{\tau^2}
\end{bmatrix}_{2\times 2}$ is a $2\times 2$ matrix. 

Because of $A_t$ is symmetric and invertible, $B_t$ is the diagonal matrix defined as above, therefore they have the following property 
\begin{align}
& A_tB_t=A_t^\top B_t^\top = \left(B_tA_t\right)^\top, \\
& A_t^{-1}B_t = A_t^{-\top}B_t^\top = \left(B_tA_t^{-1}\right)^\top. 
\end{align}
Followed up the form of $\Sigma_{YY}^{(t) (-1)}$, we can define that 
\begin{align}
\Sigma_{YY}^{(t) (-1)} \triangleq \begin{bmatrix} 
B_{t-1} & 0 \\ 0 & B_1 \end{bmatrix}
\begin{bmatrix} 
Z_{t} & b_{t} \\
b_{t}^\top & K_{t}
\end{bmatrix} \begin{bmatrix} 
B_{t-1} & 0 \\ 0 & B_1\end{bmatrix}
\end{align}
where $Z_{t}$ is a $2t \times 2t$ matrix, $ b_{t} $ is a $2t \times 2$ matrix and $K_{t}$ is a $2 \times 2$ matrix. Thus,by taking its inverse again, we will get 
\begin{align} 
\Sigma_{YY}^{\left(t\right)}= \begin{bmatrix}
B_{t-1}^{-1} \left(Z_{t}-b_{t}K_{t}^{-1}b_{t}^\top\right)^{-1}B_{t-1}^{-1}  & - B_{t-1}^{-1}  Z_{t}^{-1}b_{t}\left(K_{t}-b_{t}^\top Z_{t}^{-1}b_{t}\right)^{-1}B_1^{-1} \\
-B_1^{-1}  K_{t}^{-1}b_{t}^\top \left(Z_{t}-b_{t}K_{t}^{-1}b_{t}^\top\right)^{-1}B_{t-1}^{-1}  & B_1^{-1}  \left(K_{t}-b_{t}^\top Z_{t}^{-1}b_{t}\right)^{-1}B_1^{-1} 
\end{bmatrix}.
\end{align}

It is easy to find the relationship of $A_{t-1}$ and $A_{t}$ satisfies  
\begin{align} A_{t} = 
\begin{bmatrix}
A_{t-1} & \cdot & \cdot  \\ \cdot &\frac{1}{\sigma^2} &\cdot  \\ \cdot  & \cdot  & \frac{1}{\tau^2} 
\end{bmatrix} + U_{t}U_{t}^\top \triangleq M_{t}  + U_{t}U_{t}^\top,
\end{align}
where, in fact, $M_{t} = \begin{bmatrix}
A_{t-1} & \cdot & \cdot  \\ \cdot &\frac{1}{\sigma^2} &\cdot  \\ \cdot  & \cdot  & \frac{1}{\tau^2}
\end{bmatrix}  = \begin{bmatrix}
A_{t-1} & 0 \\ 0 & B_1
\end{bmatrix}$ 
and its inverse is $M_{t}^{-1} =\begin{bmatrix}
A_{t-1}^{-1} & 0 \\ 0 & B_1^{-1}
\end{bmatrix}$. By using the Sherman-Morrison-Woodbury formula, one can find the inverse of $A_{t}$ in such a recursive way that 
\begin{equation}
A_{t}^{-1} = \left(M_{t}+U_{t}U_{t}^\top\right)^{-1}= M_{t}^{-1}-M_{t}^{-1}U_{t}\left(I_t+U_{t}^\top M_{t}^{-1}U_{t}\right)^{-1}U_{t}^\top M_{t}^{-1}.
\end{equation}
Consequently, after being simplified, it gives us 
\begin{equation}\label{OUupdatingK}
K_{t} =B_1^{-1}D_{t} \left(I_t+ S_{t}^\top \left(B_1^{-1} - K_{t-1}\right)  S_{t} +D_{t}^\top B_1^{-1}D_{t}  \right)^{-1}  D_{t}^\top B_1^{-1},
\end{equation}
and
\begin{align}
b_{t} = \begin{bmatrix}
-b_{t-1} \\ B_1^{-1}-K_{t-1} 
\end{bmatrix}  S_{t} \left(I_t+ S_{t}^\top \left(B_1^{-1} - K_{t-1}\right)  S_{t} +D_{t}^\top B_1^{-1}D_{t}  \right)^{-1} D_{t}^\top B_1^{-1}, 
\end{align}
by which, $K_t$ and $b_t$ are updated in a recursive way. As a result, one can obtain the following recursive updating formula for the mean and covariance matrix 
\begin{align}
\begin{split}
\bar{\mu}_{t}&=\Phi_{t} K_{t-1}B_1\bar{\mu}_{t-1} + \Phi_{t} \left(I_t-K_{t-1}B_1\right)Y_{t-1}\\
\bar{\Sigma}_{t}&=\left(B_1K_{t}B_1\right)^{-1}
\end{split}
\end{align}
The matrix $K_{t}$ is updated via equation \eqref{OUupdatingK}, or updating its inverse in the following form makes the computation faster, that is 
\begin{align}
K_{t}^{-1} &= B_1D_{t}^{-\top}D_{t}^{-1}B_1 + B_1\Phi_{t} \left(B_1^{-1} - K_{t-1}\right) \Phi_{t}^\top B_1+ B_1,\\
\bar{\Sigma}_{t} &= D_{t}^{-\top}D_{t}^{-1}+ \Phi_{t} \left(B_1^{-1} - K_{t-1}\right) \Phi_{t}^\top + B_1^{-1}
\end{align}
and $K_1 =B_1^{-1} - A_1^{-1} = \begin{bmatrix}
\frac{\sigma^4}{\sigma^2 +L_x^2} & 0 \\ 0 &\frac{\tau^4}{\tau^2 +L_u^2}
\end{bmatrix} $. For calculation details, readers can refer to Appendix \ref{OUcalculation}. 


\subsection{The Estimation Distribution}

Because of the joint distribution \eqref{jointmatrix}, one can find the estimation with a given $\theta$ is in the same form as equation \eqref{estimationdistribution}. 
%\begin{equation}
%X_{1:t} \mid Y_{1:t},\theta \sim N\left(L_t^{-\top}W_t,L_t^{-\top}L_t^{-1}\right),
%\end{equation}
%thus
%\begin{align}
%X _{1:t}= L_t^{-\top}\left(W_t+Z_t\right).
%\end{align}
%where $Z_t \sim N\left(0, I(\sigma,\tau)_t\right)$.
Being explicitly, for $X_{t}$, the joint distribution with $Y_{1:t}$ updated to time $t$ is 
\begin{align}
X_{t}, Y_{1:t} \mid \theta \sim N\left( 0, \begin{bmatrix}
C_{t}^\top\left(A_{t}-B_{t}\right) ^{-1}C_{t} & C_{t}^\top \left(A_{t}-B_{t}\right)^{-1}\\
\left(A_{t}-B_{t}\right)^{-1}C_{t} & \left(I_t- A_{t}^{-1}B_{t}\right) ^{-1}B_{t}^{-1}
\end{bmatrix} \right),
\end{align}
where $C_{t}^\top=\begin{bmatrix}
0 & \cdots & 0 & 1 & 0 \\ 0 & \cdots & 0 & 0 & 1 
\end{bmatrix}$ is a $2\times 2t$ matrix. Thus,
\begin{align}
X_{t}\mid Y_{1:t},\theta \sim N\left(\mu_{t}^{\left(X\right)},\Sigma_{t}^{\left(X\right)}\right),
\end{align}
where
\begin{align}
\mu_{t}^{\left(X\right)} & = C_{t}^\top A_t^{-1}B_tY_t =C_{t}^\top L_t^{-\top}W_t,\\
\Sigma_{t}^{\left(X\right)} & =C_{t}^\top A_t^{-1}C_{t} =U_{t}^\top U_{t},
\end{align}
and $U_{t} = L_t^{-1} C_{t}$.
%The filtering distribution of the state given parameters is $p\left(X_t\mid Y_{1:t}, \theta \right)$. To find its form, one can use the joint distribution of $X_{t}$ and $Y_{1:t}$, which is $p\left(X_{t}, Y_{1:t}  \mid  \theta\right)\sim N\left(0,\Gamma\right)$, where
%\begin{equation}
%\Gamma=\begin{bmatrix} C_{t}^\top\left(A-B\right)^{-1}C_{t} & C_{t}^\top\left(A-B\right)^{-1}\\\left(A-B\right)^{-1}C_{t} & \left(I-A^{-1}B\right)^{-1}B^{-1} \end{bmatrix}.
%\end{equation}
The recursive updating formula is  
\begin{align}
\mu_{t}^{\left(X\right)}  &=  K_{t}B_1\bar{\mu}_{t} + \left(I_t- B_1K_{t}\right)Y_{t}  \\
\Sigma_{t}^{\left(X\right)}  &=B_1^{-1}-K_{t}.
\end{align}





\subsection{Prior Distribution for Parameters}

The well known \textit{Hierarchical Linear Model}, where the parameters vary at more than one level, is first introduced by \cite{lindley1972bayes, smith1973general}. Hierarchical Model can be used on data with many levels, although 2-level models are the most common ones. The state-space model in equations \eqref{statemodel1} and \eqref{statemodel2} is one of Hierarchical Linear Model if $G_t$ and $F_t$ are linear, and non-linear model if $G_t$ and $F_t$ are non-linear processes. Researchers have made a few discussions and work on these both linear and non-linear models. In this section, we only discuss on the prior for parameters in these models. 

Various informative and non-informative prior distributions have been suggested for scale parameters in hierarchical models. \cite{gelman2006prior} give a discussion on prior distributions for variance parameters in hierarchical models. General considerations include using invariance \citep{jeffries1961theory}, maximum entropy \citep{jaynes1983papers} and agreement with classical estimators \citep{box2011bayesian}. Regarding informative priors, the author suggests to distinguish them into three categories: (\romannum{1}) is traditional informative prior. A prior distribution giving numerical information is crucial to statistical modeling and it can be found from a literature review, an earlier data analysis or the property of the model itself. (\romannum{2}) is weakly informative prior. This genre prior is not supplying any controversial information but are strong enough to pull the data away from inappropriate inferences that are consistent with the likelihood. Some examples and brief discussions of weakly informative priors for logistic regression models are given in \citep{gelman2008weakly}. (\romannum{3}) is uniform prior, which allows the information from the likelihood to be interpreted probabilistically. 

\cite{stroud2007sequential} discuss a model with different structures in the errors. The two errors $\omega_t$ and $\varepsilon_t$ are assumed normally distributed as
\begin{align}
\omega_t &\sim N(0,\alpha Q),\\
\varepsilon_t &\sim N(0,\alpha R),
\end{align}
where the two matrices $R$ and $Q$ are known and $\alpha$ is an unknown scale factor to be estimated. (Note that the forward map will be deterministic if $Q=0$.) The density of the Gaussian state-space model therefore becomes 
\begin{align}
p(y_t\mid x_t,\alpha) &= N(F(x_t),\alpha R),\\
p(x_t\mid x_{t-1},\alpha) &= N(G(x_{t-1}),\alpha Q).
\end{align}
The parameter $\alpha$ is assumed \textit{Inverse Gamma} distribution. 

For the priors of all the parameters in an OU-process, shown in equation \eqref{OUprocess} and \eqref{obseq}, first of all, we should understand what meanings of these parameters are standing for. The reciprocal of $\gamma$ is typical velocity falling in the reasonable range of 0.1 to 100 $m/s$. $\xi$ is the error occurs in transition process, $\sigma$ and $\tau$ are errors in the forward map for position and velocity respectively. Generally, the error is a positive finite number. Considering prior distributions for these parameters, before looking at the data, we have an idea of ranges where these parameters are falling in. Conversely, we do not have any assumptions about the true value of $\lambda$, which means it could be anywhere. According to this assumption, the prior distributions are 
\begin{align}
\gamma   &\sim IG(10,0.5),\\
\xi^2        &\sim IG(5,2.5),\\
\sigma^2 &\sim IG(5,2.5),
\end{align}
where $IG(\alpha,\beta)$ represents the \textit{Inverse Gamma} distribution with two parameters $\alpha$ and $\beta$. 
\begin{figure}[h]
\centering
\includegraphics[width=0.45\textwidth]{Chapters/05MCMCOU/plots/ggIGPDF_Final.pdf}
\includegraphics[width=0.45\textwidth]{Chapters/05MCMCOU/plots/ggIGCDF_Final.pdf}
\caption{Probability density function and cumulative distribution function of \textit{Inverse Gamma} with two parameters $\alpha$ and $\beta$. }
\label{IGPDFCDF}
\end{figure}


\subsection{Efficiency of Delayed-Acceptance Metropolis-Hastings Algorithm}

We have discussed the efficiency of Metropolis-Hastings (MH) algorithm and how it is affected by the step size. To explain it explicitly, here we give an example comparing Eff (efficiency), EffUT (efficiency in unit time), ESS (effective sample size) and ESSUT (effective sample size in unit time), which are calculated by using the dataset, which is demonstrated in Figure \ref{realdatareview}, and running 10\,000 iterations of DA MH. We are taking a sequence from 0.1 to 4 with equal-space of 0.3, so that $s=\left\lbrace 0.1,\dots,4\right\rbrace$, and to solve criterion formula with each of the value. Table \ref{effeutessessutexampletable} and Figure \ref{effeutessessutexamplefigure} show the compares the results of the calculation. 

The best step size found by Eff is 1, which is as the same as that found by ESS. Let $s=1$ and run 1\,000 iterations, the DA MH takes 36.35 seconds to get the Markov chain for $\theta$ and the acceptance rates $\alpha_1$ for approximating $\hat{\pi}(\cdot)$ and $\alpha_2$ for estimating the posterior distribution $\pi(\cdot)$ are 0.3097 and 0.8324 respectively. By using EffUT and ESSUT, the best step size is 2.5, which is bigger. One of the advantages of using this step size is the significant decreasing of the computation time to 5.10 seconds. It is because the surrogate $\hat{\pi}(\cdot)$ takes bad proposals out and only good ones are accepted to pass to the next level. It can be seen from the lower rates $\alpha_1$ in Table \ref{effeutessessutexampletable}. 
%\begin{table}[h]
%\centering
%\begin{tabular}{\mid c\mid c\mid c\mid c\mid c\mid c\mid }
%\hline
%          & Values      & Time (in seconds) & Step Size & $\alpha_1$ & $\alpha_2$ \\ \hline
%Eff      & 0.0532      & 182.55 & 1.0   & 0.3270    & 0.7011     \\ \hline
%EffUT    & 0.0005      & 41.16 & 2.2   & 0.0687   & 0.5555    \\ \hline
%ESS     & 1275.6400 & 123.13 & 1.3   & 0.2180     & 0.6573   \\ \hline
%ESSUT & 13.8781   & 29.31   & 2.5   & 0.0469   & 0.5090   \\ \hline
%\end{tabular}
%\caption{An example of Eff, EffUT, ESS and ESSUT found by using the same data.  }
%\label{effeutessessutexampletable}
%\end{table}
%\begin{figure}[h]
%\centering
%\includegraphics[width=8cm,height=4cm]{Chapters/05MCMCOU/plots/eff.pdf}
%\includegraphics[width=8cm,height=4cm]{Chapters/05MCMCOU/plots/eut.pdf}
%\includegraphics[width=8cm,height=4cm]{Chapters/05MCMCOU/plots/ess.pdf}
%\includegraphics[width=8cm,height=4cm]{Chapters/05MCMCOU/plots/essut.pdf}
%\caption{An example of Eff, EffUT, ESS and ESSUT found by using the same data. }
%\label{effeutessessutexamplefigure}
%\end{figure}
\begin{table}[h]
\centering
\caption{An example of Eff, EffUT, ESS and ESSUT found by running 10\,000 iterations with same data. The computation time is measured in seconds~$s$. }
\label{effeutessessutexampletable}
\begin{tabular}{|c|C{2cm}|C{2cm}|C{2cm}|C{2cm}|C{2cm}|}
\hline
          & Values     & Time & Step Size & $\alpha_1$ & $\alpha_2$ \\ \hline
Eff      & 0.0515     & 36.35 & 1.0   & 0.3097    & 0.8324    \\ \hline
EffUT  & 0.0031     & 5.10   & 2.5   & 0.0360   & 0.7861   \\ \hline
ESS     & 501.4248 & 36.35 & 1.0   & 0.3097    & 0.8324     \\ \hline
ESSUT & 29.8912   & 5.10   & 2.5   & 0.0360   & 0.7861    \\ \hline
\end{tabular}
\end{table}
\begin{figure}[h]
\centering
\begin{subfigure}[t]{0.45\textwidth}
	\includegraphics[width=\textwidth]{Chapters/05MCMCOU/plots/ggeff.pdf}
	\caption{Efficiency against different step sizes}
\end{subfigure}
\begin{subfigure}[t]{0.45\textwidth}
	\includegraphics[width=\textwidth]{Chapters/05MCMCOU/plots/ggeut.pdf}
	\caption{EffUT against different step sizes}
\end{subfigure}
\begin{subfigure}[t]{0.45\textwidth}
	\includegraphics[width=\textwidth]{Chapters/05MCMCOU/plots/ggess.pdf}
	\caption{ESS against different step sizes}
\end{subfigure}
\begin{subfigure}[t]{0.45\textwidth}
	\includegraphics[width=\textwidth]{Chapters/05MCMCOU/plots/ggessut.pdf}
	\caption{ESSUT against different step sizes}
\end{subfigure}
\caption{Influences of different step sizes on sampling efficiency (Eff), efficiency in unit time (EffUT), effective sample size (ESS) and effective sample size in unit time (ESSUT) found with the same data}
\label{effeutessessutexamplefigure}
\end{figure}


On the surface, a bigger step size causes lower acceptance rates $\alpha_1$ and it might not be a smart choice. However, on the other hand, one should notice the less time cost. To make it sensible, we are running the DA MH with different step sizes, as presented in Table \ref{effeutessessutexampletable},  for the same (or similar) amount of time. Because of the bigger step size takes less time than smaller one, so we achieve a longer chain. To be more clear, we take 1\,000 samples out from a longer chain, such as 8\,500, and calculate Eff, EffUT, ESS and ESSUT separately by the embedded function \textsf{IAT}, \citep{christen2010general}, and \textsf{ESS} of the package \textsf{LaplacesDemon} in \textit{R} and the above formulas . As we can see from the outcomes, by running the similar amount of time, the Markov chain with a bigger step size has a higher efficiency and effective sample size in unit time. More intuitively, the advantage of using larger step size is the sampling algorithm generates more representative samples per second. Figure \ref{1koutof8kfigures} is comparing different $\theta$ chains found by using different step sizes but running the same amount of time. As we can see that $\theta$ with the optimal step size has a lower correlated relationship. 
\begin{table}[h]
\centering
\caption{Comparison of Eff, EffUT, ESS and ESSUT values with different step size. The $1000^\star$ means taking 1\,000 samples from a longer chain, like 1\,000 out of 5\,000 sample chain. The computation time is measured in seconds~$s$.}
\label{stepsizecompare}
\begin{tabular}{|c|C{1.5cm}|C{1.5cm}|C{1.5cm}|C{1.5cm}|C{1.5cm}|C{1.5cm}|}
\hline
Step Size& Length & Time & Eff   & EffUT & ESS & ESSUT \\ \hline
1.0    &   1\,000        & 3.48   & 0.0619 & 0.0178   &  69.4549     & 19.9583   \\ \hline
\multirow{2}{*}{1.3}    &   1\,400        & 3.40   & 0.0547 & 0.0161   &  75.3706   & 22.1678 \\ \cline{2-7}
    &   $1\,000^\star$ & 3.40 & 0.0813 & 0.0239  & 72.5370  & 21.3344   \\ \hline
\multirow{2}{*}{2.2}     &   5\,000          &  3.31 & 0.0201 &  0.0061  &  96.6623    & 29.2031   \\ \cline{2-7}
    &   $1\,000^\star$ & 3.31  &  0.0941 & 0.0284 & 94.2254 &  28.4669 \\ \hline
\multirow{2}{*}{2.5}     &   7\,000          &  3.62  & 0.0161 &0.0044  & 112.3134   &  31.0258    \\ \cline{2-7}
  &   $1\,000^\star$ &  3.62 & \textbf{0.1095} &  \textbf{0.0302}  &  \textbf{\small 113.4063} & \textbf{31.3277} \\ \hline
\end{tabular}
\end{table}


%\begin{table}[h]
%\centering
%\begin{tabular}{|c|c|c|cc|c|c|}
%\hline
%Step Size& Length of Data  & Time (in seconds)  & Eff   & EffUT & ESS & ESSUT \\ \hline
%1.0    &   1\,000        & 3.57  & 0.0625 & 0.0175   &  61.6266     & 17.2623   \\ \hline
%1.3    &   1\,400        & 3.35  & 0.0463 & 0.0138   &  64.5524     & 19.26938  \\ \hline
%1.3    &   $1\,000^\star$ & 3.35 & 0.0638 & 0.0190   & 64.1237     & 19.1414   \\ \hline
%%2.2    &   5\,000          &  3.79&  0.0215 &  0.0057  & 109.5759    & 28.9118    \\ \hline
%%2.2    &   $1\,000^\star$ & 3.79 & 0.1025 & 0.0270  &  108.9920 & 28.7578 \\ \hline
%%2.5    &   7\,000          & 3.62  & 0.0157 & 0.0043  & 100.9355   & 27.8827    \\ \hline
%%2.5    &   $1\,000^\star$ & 3.62  & 0.1079 & 0.0298  &  97.1703 & 26.8426 \\ \hline
%2.2    &   5\,000          &  3.67 & 0.0214 &  0.0058  &  92.1885    & 25.1195   \\ \hline
%2.2    &   $1\,000^\star$ & 3.67  &  0.0963 & 0.0262 &  89.6414 & 24.4255 \\ \hline
%2.5    &   7\,000          &  3.70  & 0.0157 & 0.0043  & 103.9234   & 28.0874    \\ \hline
%2.5    &   $1\,000^\star$ &  3.70 & \textbf{0.1089} &  \textbf{0.0294}  &  \textbf{113.8122} & \textbf{30.7601} \\ \hline
%\end{tabular}
%\caption{Comparing Eff, EffUT, ESS and ESSUT values using different step size. The $1000^\star$ means taking 1\,000 samples from a longer chain, like 1\,000 out of 5\,000 sample chain. }
%\label{stepsizecompare}
%\end{table}



%
%\begin{table}[h]
%\centering
%\begin{tabular}{|c|c|c|c|c|c|c|}
%\hline
%Step Size& Length of Data  & Time (in seconds)  & Eff   & EffUT & ESS & ESSUT \\ \hline
%1.0    &   1000               & 4.07  & 0.0517 & 0.0127   & 41.5770     & 10.2155   \\ \hline
%1.3    &   1300               & 4.11  & 0.0417 & 0.0101   & 49.4266     & 12.0259   \\ \hline
%1.3    &   $1000^\star$ & 4.11  & 0.0545 & 0.0133   & 49.8249     & 12.1228   \\ \hline
%2.2    &   5000               &  4.10 &  0.0181  & 0.0044  & 79.3274    & 19.3481    \\ \hline
%2.2    &   $1000^\star$ & 4.10 & \textbf{0.0893} & \textbf{0.0218}  &  \textbf{82.5684} & \textbf{20.1386}  \\ \hline
%2.5    &   8500               & 4.06  & 0.0096 & 0.0024   & 73.4414     & 18.0890    \\ \hline
%2.5    &   $1000^\star$ & 4.06  & 0.0779& 0.0191   &  71.4099  & 17.5887  \\ \hline
%\end{tabular}
%\caption{Comparing Eff, EffUT, ESS and ESSUT values using different step size. The $1000^\star$ means taking 1000 samples from a longer chain, like 1000 out of 5000 sample chain. }
%\label{stepsizecompare}
%\end{table}

\subsection{A Sliding Window State and Parameter Estimation Approach}

The length of data used in the algorithm really affects the computation time. The forecast distribution $p(Y_{t}\mid Y_{1:t-1},\theta)$ and estimation distribution $p(X_{t}\mid Y_{1:t},\theta)$ require finding the inverse of the covariance $\Sigma_{YY}^{(t+1)}$, however, which is time consuming if the sample size is big to generate a large sparse matrix. For a moving vehicle, one is more willing to get the estimation and moving status instantly rather than being delayed. Therefore a compromise solution is the fixed-length sliding window sequential filter. A fixed-lag sequential parameter learning method was proposed by \cite{polson2008practical} and named as \textit{Practical Filtering}. The authors rely on the approximation of 
\begin{equation}
p(x_{0:n-L},\theta\mid y_{0:n-1}) \approx p(x_{0:n-L},\theta \mid y_{0:n})
\end{equation}
for large $L$. The new observations coming after the $n$th data has little influence on $x_{0:n-L}$. 

Being inspired, we do not use the first $0$ to $n-1$ date and ignore the latest $n$th, on the contrary, use all the latest date with truncating the first few history ones. Suppose we are given a fixed-length $L$, up to time $t$ ($t>L$),  we estimate the $x_t$ by using all the retrospective observations to the point at $t-L+1$. In another word, the estimation distribution for the current state is 
\begin{equation}
p(X_{t}\mid Y_{t-L+1:t},\theta),
\end{equation}
where $t>L$. We name this method the \textit{Sliding Window Sequential Parameter Learning Filter}. 

The next question is how to choose an appropriate $L$. The length of data used in MH and DA MH algorithms has an influence on the efficiency and accuracy of parameter learning and state estimation. Being tested on real data set, there is no doubt that the more data be in use, the more accurate the estimation is, and lower efficient is in computation. In Table \ref{lengthofdatacompare}, one can see the pattern of parameters $\gamma,\xi,\tau$ follow the same trend with the choice of $L$ and $\sigma$ increases when $L$ decreases. Since estimation bias is inevitable, we are indeed to keep the bias as small as possible, and in the meantime, the higher efficiency and larger effective sample size are bonus items. In Figure \ref{compareLengthData}, we can see that the efficiency and effective sample size is not varying along with the sample size used in sampling algorithm, but in unit time, they are decreasing rapidly as data size increases. 
\begin{figure}[h]
\centering
\begin{subfigure}[t]{0.45\textwidth}
    \includegraphics[width=\textwidth]{Chapters/05MCMCOU/plots/simudataOUlengtheff.pdf}
    \caption{Efficiency against data length}
\end{subfigure}
\begin{subfigure}[t]{0.45\textwidth}
    \includegraphics[width=\textwidth]{Chapters/05MCMCOU/plots/simudataOUlengtheffut.pdf}
    \caption{EffUT against data length}
\end{subfigure}
\begin{subfigure}[t]{0.45\textwidth}
    \includegraphics[width=\textwidth]{Chapters/05MCMCOU/plots/simudataOUlengthess.pdf}
    \caption{ESS against data length}
\end{subfigure}
\begin{subfigure}[t]{0.45\textwidth}
    \includegraphics[width=\textwidth]{Chapters/05MCMCOU/plots/simudataOUlengthessut.pdf}
    \caption{ESSUT against data length}
\end{subfigure}
\caption{Comparison of efficiency (Eff), efficiency in unit time (EffUT), effective sample size (ESS) and effective sample size in unit time (ESSUT) against the different length of data. Increasing data length does not significantly improve the efficiency and ESSUT.}\label{compareLengthData}
\end{figure}
In addition, from a practical point of view, the observation error $\sigma$ should be kept at a reasonable level, let's say $50cm$, and the computation time should be as less as possible. To reach that level, $L=100$ is an appropriate choice. For a one-dimensional linear model, $L$ can be chosen larger and that does not change too much. If the data up to time $t$ is less than or equal to the chosen $L$, the whole data set is used in learning $\theta$ and estimating $X_t$. 


For the true posterior, the algorithm requires a cheap estimation $\hat{\pi}(\cdot)$, which is found by one-variable-at-a-time Metropolis-Hastings algorithm. The advantage is getting a precise estimation of the parameter structure, and disadvantage is, obviously, lower efficiency. Luckily, we find that it is not necessary to run this MH every time when estimate a new state from $x_{t-1}$ to $x_t$. In fact, in the DA MH process, the cheap $\hat{\pi}$ does not vary too much in the filtering process with new data coming into the dataset. We may use this property in the algorithm. First of all, we use all available data from $1$ to $t$ with length up to $L$ to learn the structure of $\theta$ and find out the cheap approximation $\hat{\pi}$. Then, use DA MH to estimate the true posterior $\pi$ for $\theta$ and $x_t$. After that, extend dataset to $1:t+1$ if $t\leq L$ or shift the data window to $2:t+1$ if $t>L$ and run DA MH again to estimate $\theta$ and $x_{t+1}$. From figures \ref{batchwindowkeyfeature} and \ref{batchwindowparameter}, we can see that the main features and parameters in the estimating process between batch method and sliding window method have not significant differences. 


To avoid estimation bias, which is caused by sampling degeneracy, we are introducing a \textit{threshold} criterion and a \textit{cutting-off} value. In a certain circumstance, the cheap $\hat{\pi}(\cdot)$ is not accurate and is replaced by a new one $\hat{\pi}_{\mbox{\scriptsize new}}(\cdot)$. The \textit{cutting-off} procedure stops the algorithm when a large $\Delta_t$ occurs in the progress. A large time gap indicates a break of the vehicle at a time point and it causes irregularity and bias. A smart way is to stop the process and to wait for new data coming in. By running testings on real data, the \textit{threshold} is chosen $\alpha_2<0.7$ and the \textit{cutting-off} value is set at $\Delta_t\geq 300$. For each time, if the acceptance rate $\alpha_2$ is less than $0.7$, we update the mean of $\hat{\pi}$ and remain the covariance unchanged. 


In fact, the mean of the estimation may vary upon the data but the covariance matrix does not change too much, as is shown in Figure \ref{ParameterEvolutionVisualization}. Actually, these two values are on researchers' choices. Figures \ref{comparenotanupDAL} and \ref{comparenotanupfeatures} compare the performances of using and not using the \textit{threshold} criterion to update the mean of the parameters. We can see that by using the \textit{threshold} criterion, we effectively avoid estimation bias and obtain more effective samples. 

 
\begin{figure}[ht]
\centering
\begin{subfigure}[t]{0.45\textwidth}
	\includegraphics[width=\textwidth]{Chapters/05MCMCOU/plots/realdataestbiaslogDAnoupdate.pdf}
	\caption{$\ln DA$ surfaces of not-updating-mean}
\end{subfigure}
\begin{subfigure}[t]{0.45\textwidth}
	\includegraphics[width=\textwidth]{Chapters/05MCMCOU/plots/realdataestbiaslogDAupdate.pdf}
	\caption{$\ln DA$ surfaces of updating-mean}
\end{subfigure}
\begin{subfigure}[t]{0.45\textwidth}
	\includegraphics[width=\textwidth]{Chapters/05MCMCOU/plots/realdataestbiaslogLnoupdate.pdf}
	\caption{$\ln L$ surfaces of not-updating-mean}
\end{subfigure}
\begin{subfigure}[t]{0.45\textwidth}
	\includegraphics[width=\textwidth]{Chapters/05MCMCOU/plots/realdataestbiaslogLupdate.pdf}
	\caption{$\ln L$ surfaces of updating-mean}
\end{subfigure}
\caption{Comparing $\ln DA$ and $\ln L$ surfaces between not-updating-mean and updating-mean methods. It is obviously that the updating-mean method has higher dense log-surfaces, which contain more effective samples.} \label{comparenotanupDAL}
\end{figure}

\begin{figure}[ht]
\centering
\begin{subfigure}[t]{0.45\textwidth}
\includegraphics[width=\textwidth]{Chapters/05MCMCOU/plots/realdatacomparea1notupandup2.pdf}
   \caption{Comparing $\alpha_1$}
\end{subfigure}
\begin{subfigure}[t]{0.45\textwidth}
\includegraphics[width=\textwidth]{Chapters/05MCMCOU/plots/realdatacomparea2notupandup2.pdf}
   \caption{Comparing $\alpha_2$}
\end{subfigure}
\begin{subfigure}[t]{0.45\textwidth}
\includegraphics[width=\textwidth]{Chapters/05MCMCOU/plots/realdatacompareeffutnotupandup2.pdf}
   \caption{Comparing EffUT}
\end{subfigure}
\begin{subfigure}[t]{0.45\textwidth}
\includegraphics[width=\textwidth]{Chapters/05MCMCOU/plots/realdatacompareessutnotupandup2.pdf}
   \caption{Comparing ESSUT}
\end{subfigure}
\caption{Comparing acceptance rates $\alpha_1$, $\alpha_2$, EffUT and ESSUT between not-updating-mean and updating-mean methods. Black solid dots $\bullet$ indicate values obtained from not-updating-mean method and black solid triangular $\blacktriangle$ indicate values obtained from updating-mean method. The acceptance rates of the updating-mean method are more stable and effective samples are larger in unit computation time. }\label{comparenotanupfeatures}
\end{figure}


Consequently, a complete form of this algorithm is summarized in the following Algorithm \ref{algorithmslidingwindow}: 
\begin{algorithm}[ht]
\SetAlgoLined 
Initialization: Set up $L$, \textit{threshold} and  \textit{cutting-off} criteria. \\
Learning phase: Estimate $\theta$ with $p\left(\theta\mid Y_{1:\min \left\lbrace t,L\right\rbrace } \right) \propto p\left(Y_{1:\min \left\lbrace t,L\right\rbrace } \mid \theta \right)p\left(\theta \right)$ by one-variable-at-a-time Random Walk Metropolis-Hastings algorithm gaining the target acceptance rates and find out the structure of $\theta\sim N\left(\mu,\Sigma\right)$ and the approximation $\hat{\pi}\left(\cdot\right)$. \label{algorithmlearningsurface}\\
Estimation phase: draw samples for $\theta$ and $X_{ \max\left\lbrace 1,t-L+1 \right\rbrace :\min \left\lbrace t,L\right\rbrace }$ given $Y_{ \max\left\lbrace 1,t-L+1 \right\rbrace :\min \left\lbrace t,L\right\rbrace }$: \For{$i$ from 1 to $N$}{ \label{algorithmestimaiton}
Propose $\theta_i^*$ from $N\left(\theta_i\mid\mu,\Sigma\right)$, accept it with probability $\alpha_1=\min\left\lbrace  1,\frac{\hat{\pi}\left(\theta_i^*\right)q\left(\theta_i, \theta_i^*\right)}{\hat{\pi}\left(\theta_i\right)q\left(\theta_i^*, \theta_i\right)}  \right\rbrace$ and go to next step; otherwise go to step \ref{algorithmDA}.\label{algorithmDA}\\
Accept $\theta_i^*$ with probability $\alpha_2=\min \left\lbrace  1,\frac{\pi\left(\theta_i^*\right)\hat{\pi}\left(\theta_i\right) }{\pi\left(\theta_i\right)\hat{\pi}\left(\theta_i^*\right)} \right\rbrace$ and go to next step; otherwise go to step \ref{algorithmDA}. \\
Calculate $\mu_i^{\left(t\right)},\Sigma_i^{\left(t\right)}$ for $X_t$ and $\mu_i^{\left(t+s\right)},\Sigma_i^{\left(t+s\right)}$ for $X_{t+s}$.\\
}
Calculate $\mu_X^{\left(t\right)} = \frac{1}{N} \sum_i \mu_i^{\left(t\right)}$, $\Var\lbrack X^{\left(t\right)}\rbrack = \frac{1}{N} \sum_i \left(\mu_i^{\left(t\right)} \mu_i^{\left(t\right)\top} +\Sigma_i\right) -\frac{1}{N^2} \left(\sum_i  \mu_i^{\left(t\right)}\right) \left(\sum_i \mu_i^{\left(t\right)}\right)^\top$ and $\mu_X^{\left(t+s\right)}$, $\Var\lbrack X^{\left(t+s\right)}\rbrack$ with the same formula.  \\
Check \textit{threshold} and  \textit{cutting-off} criteria. \uIf{\textit{threshold} is TRUE}{Update $\theta\sim N\left(\mu,\Sigma\right)$}\uElseIf{ \textit{cutting-off} is TRUE}{Stop process. }
\Else{ Go to next step.}
Shift the window by setting $t = t+1$ and go back to step \ref{algorithmestimaiton}.
 \caption{Sliding Window Adaptive MCMC}\label{algorithmslidingwindow}
\end{algorithm}


\clearpage

\subsection{Application on 2-Dimensional GPS Data}

An application of the Algorithm \ref{algorithmslidingwindow} is to track the position of a moving tractor on a farm. The original GPS dataset is plotted in Figure \ref{realGPSdataset}. In a 2-dimensional trajectory filtering problem, we use the same parameter $\theta=\lbrace \gamma,\xi^2,\lambda^2,\sigma^2,\tau^2 \rbrace$ for both easting and northing directions. The observations on these two directions are denoted as $Y_E$ and $Y_N$ respectively. The hidden states on easting and northing directions are $X_E$ and $X_N$. 


To speed up the estimation, we should get an idea of what the parameter space looks like by running step \ref{algorithmlearningsurface} of the algorithm with a subset of observations. By setting $L=100$ and running 5\,000 iterations, we find 5\,000 samples for $\theta$ in 59 seconds. For each parameter of $\theta$, we take 1\,000 sub-samples out as a new sequence. The new $\theta^*$ is representative for the parameter space. Then the traces and correlation are derived from $\theta^*$. Meanwhile, the acceptance rates for each parameter are $\alpha_\gamma = 0.453,\alpha_{\xi^2}=0.433, \alpha_{\lambda^2}=0.435, \alpha_{\sigma^2}=0.414, \alpha_{\tau^2}=0.4490$ respectively. Hence, the structure of $\hat{\theta}\sim N\left( m_t,C_t\right)$, which is depicted in Figure \ref{realdatacorMatrix}, is obtained. 

\begin{figure}[h]
	\centering
	\includegraphics[width=0.5\textwidth]{Chapters/05MCMCOU/plots/realdatalearningcorMatrix.pdf}
	\caption{Visualization of the parameters correlation matrix, which is found in the learning phase. }\label{realdatacorMatrix}
\end{figure}

\begin{figure}[h]
\centering
\begin{subfigure}[t]{0.45\textwidth}
	\includegraphics[width=\textwidth]{Chapters/05MCMCOU/plots/realdatalearninggam.pdf}
    \caption{Trace plot of $\gamma$}
\end{subfigure}
\begin{subfigure}[t]{0.45\textwidth}
	\includegraphics[width=\linewidth]{Chapters/05MCMCOU/plots/realdatalearningxi2.pdf}
 	\caption{Trace plot of $\xi^2$}
\end{subfigure}
\begin{subfigure}[t]{0.45\textwidth}
	\includegraphics[width=\linewidth]{Chapters/05MCMCOU/plots/realdatalearninglab2.pdf}
	\caption{Trace plot of $\lambda^2$}
\end{subfigure}
\begin{subfigure}[t]{0.45\textwidth}
	\includegraphics[width=\linewidth]{Chapters/05MCMCOU/plots/realdatalearningsig2.pdf}
	\caption{Trace plot of $\sigma^2$}
\end{subfigure}
\begin{subfigure}[t]{0.45\textwidth}
	\includegraphics[width=\linewidth]{Chapters/05MCMCOU/plots/realdatalearningtau2.pdf}
	\caption{Trace plot of $\tau^2$}
\end{subfigure}
\caption{Trace plots of $\theta$ after taking 1\,000 burn-in samples out from 5\,000 from the learning phase.}
\end{figure}



\clearpage

Since a cheap surrogate $\hat{\pi}(\cdot)$ for the true $\pi(\cdot)$ is found in step \ref{algorithmlearningsurface} (the learning phase), it is time to move on to the estimation phase. Algorithm \ref{algorithmslidingwindow} takes fixed $L$ length data from $\{Y_E,Y_N\}_{1:L}$ to $\{Y_E,Y_N\}_{t-L+1:t}$ until an irregular large time lag meets the cutting-off criterion. In the implementation, the first cutting-off occurs after the $648$-th point. The first $100$ estimates $\{X_E,X_N\}_{1:100}$ were found in the learning phase and $\{X_E,X_N\}_{101:648}$ were found sequentially in the estimation phase with approximate 9 seconds per 10\,000 iterations for each $\{X_E,X_N\}_s, s\in \lbrack 101,648\rbrack$. The outcome is depicted in Figure \ref{MCMCfirstportionestimation}. 


\begin{figure}[h]
\centering
%\includegraphics[width=0.8\textwidth]{Chapters/05MCMCOU/plots/realdatabatchPosition2whole.pdf}
%\includegraphics[width=0.8\textwidth]{Chapters/05MCMCOU/plots/realdatabatchVelocity2.pdf}
\begin{tikzpicture}
	\node[anchor=south west,inner sep=0] (image) at (0,0) {\includegraphics[width=0.9\textwidth]{Chapters/05MCMCOU/plots/realdatabatchPosition3.pdf}};
	\begin{scope}[
	x={(image.south east)},
	y={(image.north west)}
	]
	\node [black, font=\bfseries] at (0.5,0) {Easting};
	\node [black, font=\bfseries,rotate=90] at (0,0.5) {Northing};
	\end{scope}
\end{tikzpicture}
\caption{Estimations of $Z$ found by combined batch and sequential methods. The red line is the estimation by batch method and the green line is the sequential MCMC filtering estimation. Black dots are the measurements.}\label{MCMCfirstportionestimation}
\end{figure}

The means of uncertainties in the estimation for each direction are about 0.5 meters. Figure \ref{MCMCErrorFill} depicts uncertainties of the estimation before the first cutting-off procedure activated. The shaded blue filling indicates that there are larger uncertainties at turning points. In the estimation phase, Algorithm \ref{algorithmslidingwindow} is able to estimate $X_t$ and to predict $X_{t+s}$. However, when $s$ goes along time $t$, the uncertainty becomes larger. When a new observation $X_{t+1}$ comes into the data stream, the uncertainty shrinks. 



\begin{figure}[h]
	\centering
	\begin{tikzpicture}
	\node[anchor=south west,inner sep=0] (image) at (0,0) {\includegraphics[width=0.45\textwidth]{Chapters/05MCMCOU/plots/tracErrorFillX.pdf}};
	\begin{scope}[
	x={(image.south east)},
	y={(image.north west)}
	]
	\node [black, font=\bfseries] at (0.5,0) {$t$};
	\node [black, font=\bfseries,rotate=90] at (0,0.5) {Easting};
	\end{scope}
	\end{tikzpicture}
		\begin{tikzpicture}
	\node[anchor=south west,inner sep=0] (image) at (0,0) {\includegraphics[width=0.45\textwidth]{Chapters/05MCMCOU/plots/tracErrorFillY.pdf}};
	\begin{scope}[
	x={(image.south east)},
	y={(image.north west)}
	]
	\node [black, font=\bfseries] at (0.5,0) {$t$};
	\node [black, font=\bfseries,rotate=90] at (0,0.5) {Northing};
	\end{scope}
	\end{tikzpicture}
	\caption{Uncertainties on easting and northing directions before the first cutting-off procedure. The means of uncertainties on each direction are about 0.5 meters. }\label{MCMCErrorFill}
\end{figure}



%\begin{figure}[h]
%\centering
%\includegraphics[width=0.45\textwidth]{Chapters/05MCMCOU/plots/realdataEstXYwithEr.pdf}
%\includegraphics[width=0.45\textwidth]{Chapters/05MCMCOU/plots/realdataestwitherror.pdf}
%\caption{Zoom in on estimations. For each estimation $\hat{X}_i (i=1,\dots,t)$, there is an error circle around it. }
%\end{figure}


After a cutting-off procedure is activated, the adaptive MCMC algorithm goes off-line and accumulates measurements until there are enough, for example 100, for continued estimation \footnote{Alternatively, a ``hot start'' is possible in which the priors are the posteriors of the previous estimation phase and no learning phase is required.}. In the application, we can see that the first 100 observations are used for parameter estimation in the learning phase. Once this step is done, the sequential estimation phase goes on-line for filtering calculation. Because there is a large time lag between the 648-th and 649-th points, the algorithm goes off-line again to accumulate data in the learning phase, and then goes back to on-line for filtering estimation. 

Figure \ref{MCMCwholeestimation} gives the whole estimated trajectory by the proposed sliding window MCMC algorithm. There are two main learning phase occur on the entire dataset. The first learning phase uses the first 100 data and the second learning phase uses the data from 649 to 748. With the information obtained from the learning phase, two sequential estimation phases estimate the data from 101 to 648 and from 749 to 1121 respectively. However, when the third large time lag occurs after the 1121-th point, the algorithm has insufficient observations to run a third learning phase. In this figure, the on-line/off-line switching points are colored in yellow. 


\begin{figure}[h]
	\centering
	\begin{tikzpicture}
	\node[anchor=south west,inner sep=0] (image) at (0,0) {\includegraphics[width=0.9\textwidth]{Chapters/05MCMCOU/plots/realdatabatchPositionwhole2.pdf}};
	\begin{scope}[
	x={(image.south east)},
	y={(image.north west)}
	]
	\node [black, font=\bfseries] at (0.5,0) {Easting};
	\node [black, font=\bfseries,rotate=90] at (0,0.5) {Northing};
	\node [black, font=\bfseries] at (0.68,0.68) {1};
	\node [black, font=\bfseries] at (0.15,0.75) {2};
	\node [black, font=\bfseries] at (0.94,0.83) {3};
	\node [black, font=\bfseries] at (0.64,0.08) {4};
	\end{scope}
	\end{tikzpicture}
	\caption{Two learning phases are colored by red and two sequential estimation phases are colored by green. The algorithm is not able to estimate the data from 1121 till the end because of the lack of observations. Point 1 is the first point of the data stream. Points 2 and 3 are the switching points. Point 4 is the last point of the data stream. }\label{MCMCwholeestimation}
\end{figure}



\section{Discussion and Future Work}


In this chapter, an adaptive MCMC algorithm is proposed for estimating combined state and parameter in a homogeneous linear state-space model. The whole process is split into two phases: learning phase and estimation phase. In the learning phase, a self-tuning one-variable-at-a-time random walk Metropolis-Hastings algorithm is used to learn the structure of the parameter space. After getting a cheap surrogate for the expensive posterior distribution, it is then used in a delayed-acceptance algorithm in the estimation phase. 

Note that in the learning phase, we determine an approximation for the posterior distribution to be used in the delayed-acceptance MH algorithm. This is quite different to population MCMC \citep{laskey2003population}, in which multiple chains are used to determine a better proposal distribution. This does, however, suggest that multiple chains could be used to improve the learning phase. 


In on-line mode, the algorithm is adaptive to maintain sampling efficiency and uses a sliding window approach to maintain sampling speed. At the end of this chapter, the algorithm is applied to on-line estimation on a 2-dimensional GPS dataset. 


The advantage of this algorithm is that it is easy to understand and to implement in practice. In contrast, Particle Learning algorithm is highly efficient, however, the sufficient statistics are not available at all times. 

The sliding window adaptive MCMC algorithm should be contrasted with the V-spline algorithm proposed in Chapter \ref{ChapterTS}. The sliding window adaptive MCMC algorithm is a filtering algorithm that is designed for fast estimation. The V-spline is a smoothing algorithm that uses all the data for entire trajectory estimation and parameter optimization can be time consuming. On the other hand, the V-spline has piecewise continuous second derivatives, whereas the forward map \eqref{OUprocess} built into our sliding window adaptive MCMC algorithm implies sample paths are not twice differentiable.

The gradient boosting V-spline, discussed in Chapter \ref{ChapterFuture}, is potentially a much faster algorithm that could also be employed in on-line mode. Like the V-spline, the forward map used in the adaptive MCMC algorithm could also incorporate vehicle operating characteristics. However, it would be important to maintain the efficiency of the MCMC sampler in higher dimensions. 


\clearemptydoublepage


\chapter{Future Work}\label{ChapterFuture}
\input{Chapters/07FutureWork/Future_Final}
\clearemptydoublepage


\chapter{Summary}\label{ChapterSummary}
\input{Chapters/08Summary/Summary_Final}
\clearemptydoublepage


%% Appendices
% % % % \appendix


\begin{appendices}
\addtocontents{toc}{\protect\setcounter{tocdepth}{1}}
\makeatletter
\addtocontents{toc}{%
  \begingroup
  \let\protect\l@chapter\protect\l@section
  \let\protect\l@section\protect\l@subsection
}
\makeatother
 \chapter{Proofs and Figures of V-Spline Theorems}\label{appendTS}


\numberwithin{equation}{section}
\numberwithin{lemma}{section}

\section{Penalty Matrix in \eqref{tractormse}}\label{PenaltyTermDetails}

The $i$-th $\Omega^{(i)}$ is a $2n \times 2n$ bandwidth four symmetric matrix and its non-zero elements on the upper triangular are 
\begin{align}
\Omega_{2i-1,2i-1}^{(i)} & =\int_{t_{i}}^{t_{i+1}} \frac{d^2 h_{00}^{(i)}(t)}{dt^2}  \frac{d^2 h_{00}^{(i)}(t)}{dt^2} dt=\frac{12}{\Delta_i^3}\\
\Omega_{2i-1,2i}^{(i)} &=\int_{t_{i}}^{t_{i+1}} \frac{d^2 h_{00}^{(i)}(t)}{dt^2}  \frac{d^2 h_{10}^{(i)}(t)}{dt^2} dt=\frac{6}{\Delta_i^2}\\
\Omega_{2i-1,2i+1}^{(i)} &=\int_{t_{i}}^{t_{i+1}} \frac{d^2 h_{00}^{(i)}(t)}{dt^2}  \frac{d^2 h_{01}^{(i)}(t)}{dt^2} dt=\frac{-12}{\Delta_i^3}\\
\Omega_{2i-1,2i+2}^{(i)} &=\int_{t_{i}}^{t_{i+1}} \frac{d^2 h_{00}^{(i)}(t)}{dt^2}  \frac{d^2 h_{11}^{(i)}(t)}{dt^2} dt=\frac{6}{\Delta_i^2}\\
\Omega_{2i,2i}^{(i)} &=\int_{t_{i}}^{t_{i+1}} \frac{d^2 h_{10}^{(i)}(t)}{dt^2}  \frac{d^2 h_{10}^{(i)}(t)}{dt^2} dt=\frac{4}{\Delta_i} \\
\Omega_{2i,2i+1}^{(i)} &=\int_{t_{i}}^{t_{i+1}} \frac{d^2 h_{10}^{(i)}(t)}{dt^2}  \frac{d^2 h_{01}^{(i)}(t)}{dt^2} dt=\frac{-6}{\Delta_i^2}\\
\Omega_{2i,2i+2}^{(i)} &=\int_{t_{i}}^{t_{i+1}} \frac{d^2 h_{10}^{(i)}(t)}{dt^2}  \frac{d^2 h_{11}^{(i)}(t)}{dt^2} dt=\frac{2}{\Delta_i}\\
\Omega_{2i+1,2i+1}^{(i)} &=\int_{t_{i}}^{t_{i+1}} \frac{d^2 h_{01}^{(i)}(t)}{dt^2}  \frac{d^2 h_{01}^{(i)}(t)}{dt^2} dt=\frac{12}{\Delta_i^3}\\
\Omega_{2i+1,2i+2}^{(i)} &=\int_{t_{i}}^{t_{i+1}} \frac{d^2 h_{01}^{(i)}(t)}{dt^2}  \frac{d^2 h_{11}^{(i)}(t)}{dt^2} dt=\frac{-6}{\Delta_i^2}\\
\Omega_{2i+2,2i+2}^{(i)} &=\int_{t_{i}}^{t_{i+1}} \frac{d^2 h_{11}^{(i)}(t)}{dt^2}  \frac{d^2 h_{11}^{(i)}(t)}{dt^2} dt=\frac{4}{\Delta_i}
\end{align}
where $\Delta_i=t_{i+1}-t_i$ and $i=1,2,\ldots,n-1$. Then 
\begin{equation*}
\mathbf{\Omega}=\sum_{i=1}^{n-1}\lambda_i\Omega^{(i)}
\end{equation*}


\section{Proof of Theorem \ref{TractorSplineTheorem}}\label{AppendixTractorSplineProof}

\begin{proof}
%The proof is based on Ex. 5.7 of \citep{esl2009}, and is essentially a proof by contradiction. If $g:[a,b]\to \mathbb{R}$ is a proposed minimizer, then we can construct a cubic spline $f(t)$ that agrees with $g(t)$ and its first derivatives at $t_1,\ldots,t_n$, and is component-wise linear on $[a, t_1]$ and $[t_n, b]$. This means the first two terms of the objective function will be the same for $f(t)$ and $g(t)$. We now show that the curvature penalty ...
If $g:[a,b]\mapsto \mathbb{R}$ is a proposed minimizer, construct a cubic spline $f(t)$ that agrees with $g(t)$ and its first derivatives at $t_1,\ldots,t_n$, and is component-wise linear on $[a, t_1]$ and $[t_n, b]$. Let $h(t) = g(t)-f(t)$. Then, for $i = 1,\dots ,n-1$, 
\begin{align*}
\int_{t_i}^{t_{i+1}}f''(t)h''(t)dt &=f''(t)h'(t) \bigg\rvert_{t_i}^{t_{i+1}}-\int_{t_i}^{t_{i+1}}f'''(t)h'(t)dt \\
&= 0-f'''\left(t_i^+\right)\int_{t_i}^{t_{i+1}}h'(t)dt \\
&= -f'''\left(t_i^+\right)\left( h(t_{i+1}) -h(t_i) \right)\\
&=0.
\end{align*}
Additionally, $\int_{a}^{t_1}f''(t)h''(t)dt=\int_{t_n}^{b}f''(t)h''(t)dt=0$, since $f(t)$ is assumed linear outside the knots. Thus, for $i=0,\ldots,n$, 
\begin{align*}
\int_{t_i}^{t_{i+1}}\lvert g''(t) \rvert^2dt &= \int_{t_i}^{t_{i+1}}\lvert f''(t)+h''(t)\rvert^2 dt\\
&= \int_{t_i}^{t_{i+1}}\lvert f''(t)\rvert^2dt+2\int_{t_i}^{t_{i+1}}f''(t)h''(t)dt+\int_{t_i}^{t_{i+1}}\lvert h''(t)\rvert^2dt\\
&=\int_{t_i}^{t_{i+1}}\lvert f''(t)\rvert^2dt+\int_{t_i}^{t_{i+1}}\lvert h''(t)\rvert^2dt\\
&\geq \int_{t_i}^{t_{i+1}}\lvert f''(t)\rvert^2dt.
\end{align*}
The result $J[f]\leq J[g]$ follows since $\lambda_i>0$. 

Furthermore, equality of the curvature penalty term only holds if $g(t) = f(t)$. On $[t_1, t_n]$, we require $h''(t) = 0$ but since $h(t_i) = h'(t_i) = 0$ for $i = 1, \ldots , n$, this means $h(t) = 0$. Meanwhile on $[a, t_1]$ and $[t_n, b]$, $f''(t) = 0$ so that equality requires $g''(t)=0$. Since $f(t)$ agrees with $g(t)$ and its first derivatives at $t_1$ and $t_n$, equality is forced on both intervals.
\end{proof}


\section{Proof of Corollary \ref{TractorsplineCorollary}}\label{proofofCorollary}
\begin{proof}
By setting $\gamma\to 0$, the velocity information $v$ is taken away. The degrees of freedom of parameters decreases from $2n$ to $n$. Hence, there exists an $n\times 2n$ matrix $Q_\lambda$ restricting $n$ degrees of freedom of $\hat{\theta}$ and satisfying $Q_\lambda\hat{\theta}=0$.

The matrices $B$ and $C$ have the following property:  
\begin{align*}
& BB^\top=CC^\top=I_n, \\
& C^\top CB^\top=B^\top BC^\top=0.
\end{align*}
Denoting $G=B^\top B+ \gamma C^\top C +n\Omega_\lambda$ and giving $\hat{\theta} =(B^\top B+ \gamma C^\top C +n\Omega_\lambda)^{-1}(B^\top y+\gamma C^\top v)$, we will have $G\hat{\theta}=B^\top y+\gamma C^\top v$ and 
\begin{align*}
& BG\hat{\theta}=y+\gamma BC^\top v\\
& CG\hat{\theta}=CB^\top y+\gamma v.
\end{align*}
Further, $C^\top CG\hat{\theta}=C^\top (CB^\top y+\gamma v) = \gamma C^\top v$. If by setting $\gamma=0$, one will get $Q_\lambda = C^\top CG$, which consists of the even rows of $\Omega_\lambda$.

%For $1<i<n$, 
%\begin{align*}
%&\lambda_i\frac{6}{\Delta_i^2}\theta_{2i-3}+\lambda_i\frac{2}{\Delta_i}\theta_{2i-2}+\left( \lambda_i\frac{-6}{\Delta_i^2}+ \lambda_{i+1}\frac{6}{\Delta_{i+1}^2} \right) \theta_{2i-1} \\
%+ & \left( \lambda_i\frac{4}{\Delta_i} + \lambda_{i+1}\frac{4}{\Delta_{i+1}}\right)\theta_{2i} 
%+ \lambda_{i+1}\frac{-16}{\Delta_{i+1}^2}\theta_{2i+1} +  \lambda_{i+1}\frac{2}{\Delta_{i+1}}\theta_{2i+2}
%\end{align*}

By integrating by parts and using properties of the basis functions at the knots, one can get the even rows of $\Omega^{(i)}$, which are 
\begin{align*}
\Omega^{(i)}_{2i,2i-1}&=N''_{2i-1}\left(t_i^-\right) - N''_{2i-1}\left(t_i^+\right)  \\
\Omega^{(i)}_{2i,2i}   &=N''_{2i}\left(t_i^-\right) - N''_{2i}\left(t_i^+\right)  \\
\Omega^{(i)}_{2i,2i+1}&=N''_{2i+1}\left(t_i^-\right) - N''_{2i+1}\left(t_i^+\right) \\
\Omega^{(i)}_{2i,2i+2}&=N''_{2i+2}\left(t_i^-\right) - N''_{2i+2}\left(t_i^+\right) \\
\Omega^{(i)}_{2i+2,2i-1}&=N''_{2i-1}\left(t_{i+1}^-\right) - N''_{2i-1}\left( t_{i+1}^+ \right) \\
\Omega^{(i)}_{2i+2,2i}&=N''_{2i}\left(t_{i+1}^-\right) - N''_{2i}\left(t_{i+1}^+\right) \\
\Omega^{(i)}_{2i+2,2i+1}&=N''_{2i+1}\left(t_{i+1}^-\right) - N''_{2i+1}\left(t_{i+1}^+\right) \\
\Omega^{(i)}_{2i+2,2i+2}&=N''_{2i+2}\left(t_{i+1}^-\right) - N''_{2i+2}\left(t_i^+\right) 
\end{align*}
Thus 
\begin{equation*}
Q_\lambda = nC^\top C\Omega_\lambda = nC^\top C\sum_i\lambda_i \Omega^{(i)}.
\end{equation*}
Consequently, if and only if $\lambda$ is constant, $Q_\lambda\hat{\theta}=-\lambda\left( f''\left(t_i^+\right)-f''\left(t_i^-\right) \right)=0$, for $i=1,\ldots,n$, otherwise $Q_\lambda\theta=0$ is true but does not represent $f''\left(t_i^+\right)-f''\left(t_i^-\right)$. 

As a result, $f''(t)$ is continuous at the knots $t_i$ if $\lambda(t)$ is constant and $\gamma=0$. 


\end{proof}




\section{Proof of Lemma \ref{cvlemma}}
\begin{proof}
For any smooth curve $f$ with $\mathbf{y}^*$, we have 
\begin{align*}
&\frac{1}{n} \sum_{j=1}^n\left(y_j^*-f(t_j)\right)^2+\frac{\gamma}{n} \sum_{j=1}^n\left(v_j^*-f'(t_j)\right)^2+\sum_{j=1}^{n} \lambda_j\int_{t_j}^{t_{j+1}} f''^2dt \\
\geq &\frac{1}{n} \sum_{j\neq i}\left(y_j^*-f(t_j)\right)^2+\frac{\gamma}{n} \sum_{j\neq i}\left(v_j^*-f'(t_j)\right)^2+\sum_{j=1}^{n} \lambda_j\int_{t_j}^{t_{j+1}} f''^2dt\\
\geq &\frac{1}{n}\sum_{j\neq i}\left(y_j^*-\hat{f}^{(-i)}(t_j)\right)^2+\frac{\gamma}{n} \sum_{j\neq i}\left(v_j^*-\hat{f}'^{(-i)}(t_j)\right)^2+\sum_{j=1}^{n} \lambda_j\int_{t_j}^{t_{j+1}}  \left(\hat{f}^{''(-i)}\right)^2dt\\
= &\frac{1}{n}\sum_{j=1}^{n}\left(y_j^*-\hat{f}^{(-i)}(t_j)\right)^2+\frac{\gamma}{n} \sum_{j=1}^{n}\left(v_j^*-\hat{f}'^{(-i)}(t_j)\right)^2+\sum_{j=1}^{n} \lambda_j\int_{t_j}^{t_{j+1}}  \left(\hat{f}^{''(-i)}\right)^2dt\\
\end{align*}
%For any spline $f$ and a given $\lambda(t)$, 
%\begin{equation}
%\begin{split}
%&\frac{1}{n}\sum_{j=1}^{n}\left(y_j^*-f(t_j)\right)^2+\frac{\gamma}{n} \sum_{j=1}^{n}\left(v_j^*-f'(t_j)\right)^2+\int\lambda(t) f''^2 \\
%\geq&\frac{1}{n}\sum_{j\neq i}^{n}\left(y_j^*-f(t_j)\right)^2+\frac{\gamma}{n} \sum_{j\neq i}^{n}\left(v_j^*-f'(t_j)\right)^2+\int\lambda(t) f''^2\\
%\geq&\frac{1}{n}\sum_{j\neq i}^{n}\left(y_j^*-\hat{f}^{(-i)}(t_j)\right)^2+\frac{\gamma}{n} \sum_{j\neq i}^{n}\left(v_j^*-\hat{f}'^{(-i)}(t_j)\right)^2+\int\lambda(t) \hat{f}^{(-i)''2}\\
%=&\frac{1}{n}\sum_{j=1}^{n}\left(y_j^*-\hat{f}^{(-i)}(t_j)\right)^2+\frac{\gamma}{n} \sum_{j=1}^{n}\left(v_j^*-\hat{f}'^{(-i)}(t_j)\right)^2+\int\lambda(t) \hat{f}^{(-i)''2}
%\end{split}
%\end{equation}
by the definition of $\mathbf{\hat{f}}^{(-i)}$, $\mathbf{\hat{f}}'^{(-i)}$ and the fact that $y_i^*=\hat{f}^{(-i)}(t_i)$, $v_i^*=\hat{f}'^{(-i)}(t_i)$. It follows that $\hat{f}^{(-i)}$ is the minimizer of the objective function \eqref{tractorsplineObjective}, so that
\begin{align*}
\mathbf{\hat{f}}^{(-i)}&=S\mathbf{y}^*+\gamma T\mathbf{v}^*\\
\mathbf{\hat{f}}'^{(-i)}&=U\mathbf{y}^*+\gamma V\mathbf{v}^*
\end{align*}
as required.
\end{proof}


\section{Proof of Theorem \ref{tractorsplinecvscore}}

\begin{proof}
\begin{equation}\label{th3proofeq1}
\begin{split}
\hat{f}^{(-i)}(t_i)-y_i=& \sum_{j=1}^{n}S_{ij}y_j^*+ \gamma \sum_{j=1}^{n}T_{ij}v_j^*-y_i^*\\
=&\sum_{j\neq i}^{n}S_{ij}y_j+ \gamma \sum_{j\neq i}^{n}T_{ij}v_j+S_{ii}\hat{f}^{(-i)}(t_i)+\gamma T_{ii}\hat{f}'^{(-i)}(t_i)-y_i\\
=&\sum_{j=1}^{n}S_{ij}y_j+ \gamma \sum_{j=1}^{n}T_{ij}v_j+S_{ii}\left(\hat{f}^{(-i)}(t_i)-y_i\right)+\gamma T_{ii}\left(\hat{f}'^{(-i)}(t_i)-v_i\right)-y_i\\
=&\left(\hat{f}(t_i)-y_i\right)+S_{ii}\left(\hat{f}^{(-i)}(t_i)-y_i\right)+\gamma T_{ii}\left(\hat{f}'^{(-i)}(t_i)-v_i\right).
\end{split}
\end{equation}
Additionally, 
\begin{equation}
\begin{split}
\hat{f}'^{(-i)}(t_i)-v_i=& \sum_{j=1}^{n}U_{ij}y_j^*+ \gamma \sum_{j=1}^{n}V_{ij}v_j^*-v_i^*\\
=&\sum_{j\neq i}^{n}U_{ij}y_j+ \gamma \sum_{j\neq i}^{n}V_{ij}v_j+U_{ii}\hat{f}^{(-i)}(t_i)+\gamma V_{ii}\hat{f}'^{(-i)}(t_i)-v_i\\
=&\sum_{j=1}^{n}U_{ij}y_j+ \gamma \sum_{j=1}^{n}V_{ij}v_j+U_{ii}\left(\hat{f}^{(-i)}(t_i)-y_i\right)+\gamma V_{ii}\left(\hat{f}'^{(-i)}(t_i)-v_i\right)-v_i\\
=&\left(\hat{f}'(t_i)-v_i\right)+U_{ii}\left(\hat{f}^{(-i)}(t_i)-y_i\right)+\gamma V_{ii}\left(\hat{f}'^{(-i)}(t_i)-v_i\right).
\end{split}
\end{equation}
Thus 
\begin{equation}\label{th3proofeq2}
\hat{f}'^{(-i)}(t_i)-v_i = \frac{\hat{f}'(t_i)-v_i}{1-\gamma V_{ii}}+ \frac{U_{ii}\left(\hat{f}^{(-i)}(t_i)-y_i\right)}{1-\gamma V_{ii}}.
\end{equation}
By substituting equation \eqref{th3proofeq2} to \eqref{th3proofeq1}, we get
\begin{equation*}
\hat{f}^{(-i)}(t_i)-y_i=\frac{\hat{f}(t_i)-y_i+\gamma \frac{T_{ii}}{1-\gamma V_{ii}}\left(\hat{f}'(t_i)-v_i\right)}{1-S_{ii}-\gamma\frac{T_{ii}}{1-\gamma V_{ii}}U_{ii}}.
\end{equation*}
Consequently, 
\begin{equation*}
CV(\lambda,\gamma)=\frac{1}{n}\sum_{i=1}^{n}\left( \frac{\hat{f}(t_i)-y_i+\gamma \frac{T_{ii}}{1-\gamma V_{ii}}\left(\hat{f}'(t_i)-v_i\right)}{1-S_{ii}-\gamma\frac{T_{ii}}{1-\gamma V_{ii}}U_{ii}}\right)^2.
\end{equation*}
\end{proof}


%\clearpage


\section{Reconstructions at SNR=3}
\begin{figure}[!ht]
    \centering
    \includegraphics[width=0.45\textwidth]{Appendices/ggplot/ggBlocksPenaltyLineSNR3.pdf}
    \includegraphics[width=0.45\textwidth]{Appendices/ggplot/ggBumpsPenaltyLineSNR3.pdf}
    \includegraphics[width=0.45\textwidth]{Appendices/ggplot/ggHeaviPenaltyLineSNR3.pdf}
    \includegraphics[width=0.45\textwidth]{Appendices/ggplot/ggDopplerPenaltyLineSNR3.pdf}
\caption{Reconstructions of generated \textit{Blocks}, \textit{Bumps}, \textit{HeaviSine} and \textit{Doppler} functions by V-spline at SNR=3. The penalty values $\lambda(t)$ in V-spline are projected into reconstructions. The blacks dots are the measurements. The bigger blacks dots indicate the larger penalty values.}\label{TractorsplineSNR3}
\end{figure}


\section{Residual Analysis of Simulations}
\begin{figure}[!ht]
    \centering
    \begin{subfigure}{0.45\textwidth}
    \centering
    \includegraphics[width=\textwidth]{Chapters/02TractorSplineTheory/plot/ggplot/ggacfBlocks7.pdf}
    \caption{ACF of residuals from \textit{Blocks}}
    \end{subfigure}%
    \begin{subfigure}{0.45\textwidth}
    \centering
    \includegraphics[width=\textwidth]{Chapters/02TractorSplineTheory/plot/ggplot/ggacfBumps7.pdf}
    \caption{ACF of residuals from \textit{Bumps}}
    \end{subfigure}
    \begin{subfigure}{0.45\textwidth}
    \centering
    \includegraphics[width=\textwidth]{Chapters/02TractorSplineTheory/plot/ggplot/ggacfHeavi7.pdf}
    \caption{ACF of residuals from \textit{HeaviSine} }
    \end{subfigure}
    \begin{subfigure}{0.45\textwidth}
    \centering
    \includegraphics[width=\textwidth]{Chapters/02TractorSplineTheory/plot/ggplot/ggacfDoppler7.pdf}
    \caption{ACF of residuals from \textit{Doppler}}
    \end{subfigure}
\caption{ACF of residuals at SNR level of 7.}\label{tractorsplineSNR7acf}
 \end{figure}
\begin{figure}[!ht]
    \centering
    \begin{subfigure}{0.45\textwidth}
    \centering
    \includegraphics[width=\textwidth]{Chapters/02TractorSplineTheory/plot/ggplot/ggacfBlocks3.pdf}
    \caption{ACF of residuals from \textit{Blocks}}
    \end{subfigure}%
    \begin{subfigure}{0.45\textwidth}
    \centering
    \includegraphics[width=\textwidth]{Chapters/02TractorSplineTheory/plot/ggplot/ggacfBumps3.pdf}
    \caption{ACF of residuals from \textit{Bumps}}
    \end{subfigure}
    \begin{subfigure}{0.45\textwidth}
    \centering
    \includegraphics[width=\textwidth]{Chapters/02TractorSplineTheory/plot/ggplot/ggacfHeavi3.pdf}
    \caption{ACF of residuals from \textit{HeaviSine}}
    \end{subfigure}
    \begin{subfigure}{0.45\textwidth}
    \centering
    \includegraphics[width=\textwidth]{Chapters/02TractorSplineTheory/plot/ggplot/ggacfDoppler3.pdf}
    \caption{ACF of residuals from \textit{Doppler}}
    \end{subfigure}
\caption{ACF of residuals at SNR level of 3.}\label{tractorsplineSNR3acf}
 \end{figure}

\begin{figure}[!ht]
    \centering
    \begin{subfigure}{\textwidth}
    \centering
    \includegraphics[width=0.45\linewidth]{Chapters/02TractorSplineTheory/plot/ggplot/ggRealdataXYResidualsXpoints.pdf}
    \includegraphics[width=0.45\linewidth]{Chapters/02TractorSplineTheory/plot/ggplot/ggRealdataXYResidualsXhist.pdf}
    \caption{residuals of $x$ }
    \end{subfigure}
    \begin{subfigure}{\textwidth}
    \centering
    \includegraphics[width=0.45\linewidth]{Chapters/02TractorSplineTheory/plot/ggplot/ggRealdataXYResidualsYpoints.pdf}
    \includegraphics[width=0.45\linewidth]{Chapters/02TractorSplineTheory/plot/ggplot/ggRealdataXYResidualsYhist.pdf}
    \caption{residuals of $y$ }
    \end{subfigure}
\caption{Residuals of 2-dimensional real data reconstruction }\label{tractorsplineResidualsRealdata}
 \end{figure}


 \chapter{Calculations and Figures of Adaptive Sequential MCMC}\label{appendMCMC}
  \input{Appendices/AppendixMCMC_Final}
 \chapter{A Spin-off Outcome: Data Simplification Method}\label{appendSimp}
  
\section{Introduction}

GPS devices are widely used in orchard planting and maintenance. This location-based system allows orchardist to check trajectory of tractors. The trajectory is a connection by a time series successive positions recorded by GPS devices. A classical GPS device records skeleton information, including time mark, latitude, longitude, number of available satellites, etc. Recently, researchers try to enrich trajectory (called Semantic Trajectory) by adding background geographic information to discover meaningful pattern \citep{ying2011semantic}. 

Normally GPS units record more data than necessary and cause more errors due to weak signals or shelter from branches. To obtain a more accurate observation data set and to save local storage space, several data simplification methods were proposed and are focusing on simplifying data set by making either a local or global decision. 

A local simplification algorithm focuses on a couple of particular consecutive points. By analyzing the relationship between these points, a decision is made that which point can be deleted or retained. Distance threshold algorithm is one of these algorithms. All points, whose distance to the preceding track point is less than a predetermined threshold, are deleted. Direction changing algorithm is another one. The point is retained if the change in direction is greater than a predetermined threshold  \citep{ivanov2012real}. 

Alternatively, global simplification algorithms have an overview of all tracked points. After analyzing the relationships among these points, a decision will be made about which one or more points to delete or to retain. The Douglas-Peucker algorithm is the most popular one  \citep{douglas1973algorithms}. A proposed simplification method, represented by \cite{chen2009trajectory}, consider both the skeleton information and semantic meanings of a trajectory when performing simplification. 


Intuitively, the global simplification algorithms can be applied on off-line data analyses and local simplification algorithms will perform better on on-line or real-time track simplification. However, a pertinent algorithm is required in our case. 


In our case, a GPS log is a sequence time series points $p_i \in P$, $P=\left\lbrace p_1,p_2, \ldots, p_n \right\rbrace$. Each GPS point $p_i$ contains information of time mark, latitude, longitude and semantic information of velocity, heading direction and boom status, which can be written in form of
\begin{equation}
T=\left\lbrace p_t=[x_t,y_t,v_t,\theta_t,b_t] \mid t \in \mathbb{R} \right\rbrace.
\end{equation}
Sequentially connect these points will give us a trajectory of a moving vehicle.
Particularly, a tractor working on an orchard generates two kinds of boom status information: operating and not-operating. This information is recorded by GPS units and is indicated by $b=1$ for operating and $b=0$ for not-operating.


To move further, here are two concepts that will be useful to understand the simplification scheme.
\begin{itemize}
\item $\mathbf{Segment}$ A segment is a part of the consecutive trajectory. Regarding the status of the boom, the trajectory can be simply divided into two kinds of the segment in our data set, one is boom-operating, the other is boom-not-operating. 
\item $\mathbf{Direction}$. Direction $\theta$ denotes the heading direction of a tractor at a specific point location. This parameter uses north direction as a basis, in which way $\ang{0} \leq \theta < \ang{360}$.
\end{itemize}



\section{Simplification Algorithm}

The first two steps are designed to reduce some errors caused by misoperation and GPS units bugs.
\begin{itemize}\itemsep0em
\item Merging Phase. If the length of a segment composed of consecutive boom operating or not-operating points is less than a threshold, merge this one into its backward segment. 
\item Removing Phase. If two or more data points have duplicated time mark, remove the latter ones. 
\end{itemize}
Now only two types of segment points are left in GPS log, boom operating, and not-operating and the length of each segment are greater than the predetermined threshold.

The following algorithm is based on the relationship between a candidate point $p_i$ and its neighboring points $p_{i-1}$ and $p_{i+1}$, and the importance of the $p_i$ in the segment where it belongs to, $i=2,\ldots,n-1$. 

\begin{itemize}\itemsep0em
\item Rule 1. The candidate point $p_i$ is retained if it is not linear predictable or cannot be used for linear predicting. With the velocity information $v_{i-1}, v_i$ at point $p_{i-1}, p_i$ and time differences $\Delta t_{i-1} = \lvert t_{i}-t_{i-1} \rvert,\Delta t_{i} = \lvert t_{i+1}-t_{i}\rvert$, an estimated position can be calculated by $\hat{p}_i=\Delta t_{i-1} p_{i-1}$, $\hat{p}_{i+1}=\Delta t_{i} p_{i}$. If the distance $\lvert \hat{p}_i-p_i\rvert$ or $\lvert \hat{p}_{i+1}-p_{i+1}\rvert$ is less than a threshold, then the point $p_i$ is not linear predictable or cannot be used for linear predicting.

\item Rule 2. Select a candidate point $p_i$. Retain this point if the distance between $p_i$ and $p_{i-1}$ is greater than the threshold $d$, where $d$ is the mean distances of these points $p_{i-1}, p_i, \ldots, p_{i+k}$ with same boom status $b_{i-1}=b_i=\cdots=b_{i+k}$. 

\item Rule 3. Neighbor Heading Changing. The candidate point $p_i$ belongs to the track if $\lvert \theta_i-\theta_{i-1}\rvert  + \lvert \theta_i-\theta_{i+1}\rvert >\theta$, where $\lvert \theta_i-\theta_{i-1}\rvert$ and $ \lvert \theta_i-\theta_{i+1}\rvert $ are the direction changes between points $p_i$ and $p_{i-1}$ and between points $p_i$ and $p_{i+1}$, $\theta$ is predefined threshold.

\item Rule 4. The candidate point $p_i$ belongs to the track if the boom status $b_i\neq b_{i-1}$.
\end{itemize}

Finally, the point $p_i$ belongs to the track if Rule 1 = TRUE or Rule 2 = TRUE or Rule 3 = TRUE or Rule 4 = TRUE.


\section{Evaluation}

Errors are measured by Synchronized Euclidean Distance \citep{lawson2011compression}. SED measures the distances between the original and compressed trace at the same time. As shown in Figure \ref{DataSimpSED}, the green points $P_{t1}, \ldots ,P_{t5}$ are original positions. After simplification, the points $P_{t2}, P_{t3}$ and $P_{t4}$ are removed. The black curve is the original trajectory and the gray dash-dot line is the simplified trajectory. The blue points $P'_{t2}$,$P'_{t3}$ and $P'_{t4}$ on simplified trajectory have the same time difference as the point $P_{t2}$, $P_{t3}$ and $P_{t4}$ on original trajectory did. For instance, the time difference between $P_{t2}$ and $P_{t3}$ is the same as that between $P'_{t2}$ and $P'_{t3}$. Further, the distances between $P_{t2}$ and $P'_{t2}$, $P_{t3}$ and $P'_{t3}$ and $P_{t4}$, $P'_{t4}$ can be calculated.


\begin{figure}
\centering
\begin{tikzpicture}
\begin{scope}[every node/.style={circle,draw}]
    \node[fill=green!30] (A) at (0,0) {$P_{t1}$};
    \node[fill=blue!30] (B) at (3,0) {$P'_{t2}$};
    \node[fill=blue!30] (C) at (7,0) {$P'_{t3}$};
    \node[fill=blue!30] (D) at (11,0) {$P'_{t4}$};
    \node[fill=green!30] (E) at (13,0) {$P_{t5}$};
    \node[fill=green!30] (F) at (5,4) {$P_{t2}$};
    \node[fill=green!30] (G) at (7,5) {$P_{t3}$};
    \node[fill=green!30] (H) at (9,4) {$P_{t4}$};
\end{scope}
\begin{scope}[>={Stealth[black]},
              every node/.style={fill=white,circle},
              every edge/.style={draw=black,thick}]
    \path [-] (A) edge[bend right= -1]  (F); 
    \path [-] (F) edge[bend right= -5]  (G); 
    \path [-] (G) edge[bend right=-5]  (H); 
    \path [-] (H) edge[bend right= -1] (E); 
\draw [line width=0.5mm,dash dot,gray] (A) -- (B);
\draw [line width=0.5mm,dash dot,gray] (B) -- (C);
\draw [line width=0.5mm,dash dot,gray] (C) -- (D);
\draw [line width=0.5mm,dash dot,gray] (D) -- (E);
\draw [line width=0.5mm,dotted] (F) -- (B);
\draw [line width=0.5mm,dotted] (G) -- (C);
\draw [line width=0.5mm,dotted] (D) -- (H);
\draw [line width=0.5mm,blue!30] (F) -- (5,0);
\draw [line width=0.5mm,blue!30] (H) -- (9,0);
\draw [right angle symbol={B}{A}{F}];
\draw [right angle symbol={A}{B}{H}];
\path [->] (0,1) edge node {\textit{fast}} (4,4);
\path [->] (10,4) edge node {\textit{fast}} (13,1);
\end{scope}
\end{tikzpicture}
\caption{Synchronized Euclidean Distance}\label{DataSimpSED}
\end{figure}


Another way to calculate the difference between a GPS trace and its compressed version is to measure the perpendicular distance. This algorithm ignores the temporal component and uses simple perpendicular distance \citep{meratnia2004spatiotemporal}. The Figure \ref{DataSimpAB} expresses these differences clearly. 

\begin{figure}
\centering
\centering
    \begin{subfigure}[b]{0.5\textwidth}
    \resizebox{\linewidth}{!}{
		\begin{tikzpicture}[dot/.style={circle,label={#1},name=#1}]
		\coordinate (a) at (0,0);
		\node[below of=a] {\LARGE $a(t_0)$};
		\fill (a) circle[radius=4pt];
		\coordinate (b) at (15,-1);
		\node[above of=b] {\LARGE $a(t_5)$};
		\fill (b) circle[radius=4pt];
		\coordinate (c) at (0,1);
		\node[above of=c] {\LARGE $p(t_0)$};
		\fill (c) circle[radius=4pt];
		\coordinate (d) at  (2,3);
		\node[above of=d] {\LARGE $p(t_1)$};
		\fill (d) circle[radius=4pt];
		\coordinate (e) at  (7,4);
		\node[above of=e] {\LARGE $p(t_2)$};
		\fill (e) circle[radius=4pt];
		\coordinate (f) at (8,3);
		\node[above of=f] {\LARGE $p(t_3)$};
		\fill (f) circle[radius=4pt];
		\coordinate (g) at (10,-3);
		\node[below of=g] {\LARGE $p(t_4)$};
		\fill (g) circle[radius=4pt];
		\coordinate (h) at (15,-2);
		\node[below of=h] {\LARGE $p(t_5)$};
		\fill (h) circle[radius=4pt];
		\draw [line width=0.5mm] (c) -- (d) -- (e)--(f)--(g)--(h);
		\draw [line width=0.5mm] (a) -- (b);
		 % %y=x+1
		\foreach \x in {0.0,0.2,...,2} {
		    \pgfmathsetmacro\y{\x +1}
		    	\draw (\x,\y) -- ($(0,0)!(\x,\y)!(15,-1)$);
		}
		% % y=0.2x+2.6
		\foreach \x in {2,2.5,...,7} {
		    \pgfmathsetmacro\y{0.2*\x +2.6}
		    	\draw (\x,\y) -- ($(0,0)!(\x,\y)!(15,-1)$);
		}
		% % y=-x+11
		\foreach \x in {7,7.2,...,8} {
		    \pgfmathsetmacro\y{-\x +11}
		    	\draw (\x,\y) -- ($(0,0)!(\x,\y)!(15,-1)$);
		}
		% % y=-3x+27
		\foreach \x in {8,8.2,...,10} {
		    \pgfmathsetmacro\y{-3*\x +27}
		    	\draw (\x,\y) -- ($(0,0)!(\x,\y)!(15,-1)$);
		}
		% % y=0.2x-5
		\foreach \x in {10,10.5,...,15} {
		    \pgfmathsetmacro\y{0.2*\x -5}
		    	\draw (\x,\y) -- ($(0,0)!(\x,\y)!(15,-1)$);
		}
		\end{tikzpicture}
	}
	\caption{perpendicular distance chords}\label{DataSimpA}
	\end{subfigure}%
	\begin{subfigure}[b]{0.5\textwidth}
	\centering
	\resizebox{\linewidth}{!}{
		\begin{tikzpicture}[dot/.style={circle,label={#1},name=#1}]
		\coordinate (a) at (0,0);
		\node[below of=a] {\LARGE $a(t_0)$};
		\fill (a) circle[radius=4pt];
		\coordinate (b) at (15,-1);
		\node[above of=b] {\LARGE $a(t_5)$};
		\fill (b) circle[radius=4pt];
		\coordinate (c) at (0,1);
		\node[above of=c] {\LARGE $p(t_0)$};
		\fill (c) circle[radius=4pt];
		\coordinate (d) at  (2,3);
		\node[above of=d] {\LARGE $p(t_1)$};
		\fill (d) circle[radius=4pt];
		\coordinate (e) at  (7,4);
		\node[above of=e] {\LARGE $p(t_2)$};
		\fill (e) circle[radius=4pt];
		\coordinate (f) at (8,3);
		\node[above of=f] {\LARGE $p(t_3)$};
		\fill (f) circle[radius=4pt];
		\coordinate (g) at (10,-3);
		\node[below of=g] {\LARGE $p(t_4)$};
		\fill (g) circle[radius=4pt];
		\coordinate (h) at (15,-2);
		\node[below of=h] {\LARGE $p(t_5)$};
		\fill (h) circle[radius=4pt];
		\draw [line width=0.5mm] (c) -- (d) -- (e)--(f)--(g)--(h);
		\draw [line width=0.5mm] (a) -- (b);
		 % %y=x+1	
		\foreach \x [count=\i] in {0.0,0.2,...,2} {
		    \pgfmathsetmacro\y{\x +1}
		    \pgfmathsetmacro\k{(\i-1)/3}
		    \pgfmathsetmacro\z{-1/15*\k}
		    	\draw (\x,\y) -- (\k,\z);
		}
		 % y=0.2x+2.6
		\foreach \x [count=\i] in {2,2.5,...,7} {
		    \pgfmathsetmacro\y{0.2*\x +2.6}
		    \pgfmathsetmacro\k{(\i+9)/3}
		    \pgfmathsetmacro\z{-1/15*\k }
		    	   \draw (\x,\y) -- (\k,\z);
		}
		 % y=-x+11
		\foreach \x [count=\i] in {7,7.2,...,8} {
		    \pgfmathsetmacro\y{-\x +11}
		    \pgfmathsetmacro\k{(\i+19)/3}
		    \pgfmathsetmacro\z{-1/15*\k }
		    	   \draw (\x,\y) -- (\k,\z);
		}
		% % y=-3x+27
		\foreach \x [count=\i] in {8,8.2,...,10} {
		    \pgfmathsetmacro\y{-3*\x +27}
		    \pgfmathsetmacro\k{(\i+24)/3}
		    \pgfmathsetmacro\z{-1/15*\k }
		    	   \draw (\x,\y) -- (\k,\z);
		}
		% % y=0.2x-5
		\foreach \x [count=\i] in {10,10.5,...,15} {
		    \pgfmathsetmacro\y{0.2*\x -5}
		    \pgfmathsetmacro\k{(\i+34)/3}
		    \pgfmathsetmacro\z{-1/15*\k }
		    	   \draw (\x,\y) -- (\k,\z);
		}
		\end{tikzpicture}
	}
	\caption{time-synchronous distance chords}\label{DataSimpB}
    \end{subfigure}
\caption{\ref{DataSimpA} indicates that the errors are measured at fixed sampling rate as sum of perpendicular distance chords. \ref{DataSimpB} indicates that the errors are measured at fixed sampling rates as sum of time-synchronous distance chords.}\label{DataSimpAB}
\end{figure}


\section{Numerical Study}

In the numerical simulation study, we use \textit{Kalman filter} (KF) to fit the trajectory after data simplification. The KF equations describe the prediction step in such a following way: 
\begin{align*}
\hat{x}_k^-&=A\hat{x}_{k-1}+Bu_k \\
P_k^-&=AP_{k-1}A^\top+Q
\end{align*}
where $\hat{x}_k^-$ is a priori state estimate, $\hat{x}_k$ is a posteriori state estimate, $A$ is status transition matrix, $P_k^-$ is a priori estimate for error covariance, $u_k$ is an input parameter and $Q$ is process noise covariance. When a new observation comes into the data stream, KF update and corrects its estimation by: 
\begin{align*}
K_k&=P_k^-H^\top \left(HP_k^-H^\top+R\right)^{-1} \\
\hat{x}_k&=\hat{x}_k^-+K_k\left(z_k-H\hat{x}_k^-\right) \\
P_k&=(I-K_kH)P_k^-
\end{align*}
where $K_k$ is the Kalman gain matrix, $z_k$ is the observed data.


The original data set contains 1\,021 rows, including latitude, longitude, velocity, bearing (heading direction) and boom status. Douglas-Peucker Algorithm, with distance threshold $0.205$m, retained 847 points. The proposed algorithm, given a predictable distance $5$m and heading direction changing threshold $\ang{30}$, returns the same amount of simplified points. Under the same circumstance, we calculated SED and other information. 

Table \ref{DataSimpCompTable} describes the results after being simplified by DP algorithm and the proposed algorithm. Figure \ref{DataSimpRawTra} demonstrates the simplified raw data and Figure \ref{DataSimpKFTra} is the fitted trajectories by KF. 

\begin{table}
\centering
\caption{Comparison between raw data and simplified data}
\label{DataSimpCompTable}\footnotesize
\begin{tabular}{|c|c|c|c|}
\hline 
  & \textbf{Original Data} & \textbf{DP Algorithm} & \textbf{Proposed Algorithm}  \\
\hline 
\textbf{Remaining Points} & 1021   & 847   & 847 \\
\hline 
\textbf{Tracked Distances}(m)  & 74041.31     & 74038.33    & 74012.56     \\
\hline 
\textbf{SED} ($m$)    & \textit{NA}    & 1316.715    & 607.9587   \\
\hline 
\end{tabular}
\end{table}
\normalsize

\begin{figure}
\centering
%\includegraphics[width=0.9\textwidth]{Chapters/06Spinoff/plot/3p1.pdf}
\begin{subfigure}[t]{0.47\textwidth}
\includegraphics[width=\linewidth]{Chapters/06Spinoff/plot/ggRawTrac.pdf}
\caption{Raw trajectory}\label{ggRawTrac}
\end{subfigure}
 \begin{subfigure}[t]{0.47\textwidth}
\includegraphics[width=\linewidth]{Chapters/06Spinoff/plot/ggDPTrac.pdf}
\caption{Simplified trajectory by DP}\label{ggDPTrac}
\end{subfigure}
 \begin{subfigure}[t]{0.47\textwidth}
%\includegraphics[width=\linewidth]{Chapters/06Spinoff/plot/ggSPTrac2.pdf}
\begin{tikzpicture}
    \node[anchor=south west,inner sep=0] at (0,0) {\includegraphics[width=\textwidth]{Chapters/06Spinoff/plot/ggSPTrac2.pdf}};
	\draw[red,ultra thick] (2.6,3.6) circle [x radius=1cm, y radius=4mm, rotate=140];
	\draw[red,ultra thick] (6,1.2) circle [x radius=1cm, y radius=5mm, rotate=140];
\end{tikzpicture}
\caption{Simplified trajectory by proposed algorithm}\label{ggSPTrac}
\end{subfigure}
\caption{A segment start from time $t=2\,000$ to $3\,000$, recorded by GPS units. $\blacktriangle$ indicates that the boom is not operating. $\bullet$ indicates that the boom is operating. Figure \ref{ggRawTrac}, the trajectory connected by raw data with 27 points. Figure \ref{ggDPTrac}, the trajectory connected by simplified data with Douglas-Peucker algorithm with 24 points. Figure \ref{ggSPTrac}, the trajectory connected by simplified data with proposed simplification algorithm with 23  points.}\label{DataSimpRawTra}
\end{figure}

\clearpage

\begin{figure}[h]
\centering
\begin{subfigure}[t]{0.47\textwidth}
\includegraphics[width=\linewidth]{Chapters/06Spinoff/plot/ggRawKF2.pdf}
\caption{Fitted Kalman filter with raw data}
\end{subfigure}
 \begin{subfigure}[t]{0.47\textwidth}
\includegraphics[width=\linewidth]{Chapters/06Spinoff/plot/ggDPKF2.pdf}
\caption{Fitted Kalman filter with simplified data by DP}
\end{subfigure}
 \begin{subfigure}[t]{0.47\textwidth}
\includegraphics[width=\linewidth]{Chapters/06Spinoff/plot/ggSPKF2.pdf}
\caption{Fitted Kalman filter with simplified data by proposed algorithm}
\end{subfigure}
\caption{Trajectory fitted by Kalman filter. The mean squared errors of raw data, DP and proposed algorithm are 26.8922, 23.9788 and 23.9710 respectively.}\label{DataSimpKFTra}
\end{figure}


\section{Conclusion}

The data simplification algorithm is originally proposed to solve the over-fitting and wiggle-construction problems. Duplicated and short-distance points cause reconstruction issues in spline fitting. The advantage of data simplification algorithm is that less data points potentially increase computation efficiency and save storage space without of losing information for reconstructing.  

\addtocontents{toc}{\endgroup}
\end{appendices}

% % % % % % % % bib  % % % % %
\bibliographystyle{otago}
\bibliography{thesis}
\clearemptydoublepage


\end{document}  