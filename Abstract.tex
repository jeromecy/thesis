
Inference and characterization of planar trajectories have long been the focus of scientific and commercial research. Efficient algorithms for both precise and efficient trajectory reconstruction remain in high demand in a wide variety of applications.

Given time series GPS data of a moving object, trajectory reconstruction is the process of inferring the path between successive observation points. However, widely separated points and measurement errors can give rise to trajectories with sharp angles, which are not typical of a moving object. Smoothing spline methods can efficiently build up a more smooth trajectory. In conventional smoothing splines, the objective function is augmented with a penalty term, which has a single parameter that controls the smoothness of reconstruction. Adaptive smoothing splines extend the single parameter to a function that can vary, hence the degree of smoothness can be different regions. A new method named the V-spline is proposed, which incorporates both location and velocity information but penalizes excessive accelerations. In the application of interest, the penalty term is also dependent on a known operational state of the object. The V-spline includes a parameter that controls the degree to which the velocity information is used in the reconstruction. In addition, the smoothing penalty adapts the observations are irregular in time. An extended cross-validation technique is used to find all spline parameters. 


It is known that a smoothing spline can be thought of as the posterior mean of a Gaussian process regression in a certain limit. By constructing a reproducing kernel Hilbert space with an appropriate inner product, the Bayesian form of the V-spline is derived when the penalty term is a fixed constant instead of a function. An extension to the usual generalized cross-validation formula is utilized to find the optimal V-spline parameters. 


In on-line trajectory reconstruction, smoothing methods give way to filtering methods. In most algorithms for combined state and parameter estimation in state-space models either estimate the states and parameters by incorporating the parameters in the state-space, or marginalize out the parameters through sufficient statistics. Instead of these approaches, an adaptive Markov chain Monte Carlo algorithm is proposed. In the case of a linear state-space model and starting with a joint distribution over states, observations, and parameters, an MCMC sampler is implemented with two phases. In the learning phase, a self-tuning sampler is used to learn the parameter mean and covariance structure. In the estimation phase, the parameter mean and covariance structure informs the proposed mechanism and is also used in a delayed-acceptance algorithm, which greatly improves sampling efficiency. Information on the resulting state of the system is indicated by a Gaussian mixture. In the on-line mode, the algorithm is adaptive and uses a sliding window approach by cutting off historical data to accelerate sampling speed and to maintain applicable acceptance rates. This algorithm is applied to the joint state and parameters estimation in the case of irregularly sampled GPS time series data. 
