

\numberwithin{equation}{section}
\numberwithin{lemma}{section}

\section{Penalty Matrix in \eqref{tractormse}}\label{PenaltyTermDetails}

The $i$-th $\Omega^{(i)}$ is a $2n \times 2n$ bandwidth four symmetric matrix and its non-zero elements on the upper triangular are 
\begin{align}
\Omega_{2i-1,2i-1}^{(i)} & =\int_{t_{i}}^{t_{i+1}} \frac{d^2 h_{00}^{(i)}(t)}{dt^2}  \frac{d^2 h_{00}^{(i)}(t)}{dt^2} dt=\frac{12}{\Delta_i^3}\\
\Omega_{2i-1,2i}^{(i)} &=\int_{t_{i}}^{t_{i+1}} \frac{d^2 h_{00}^{(i)}(t)}{dt^2}  \frac{d^2 h_{10}^{(i)}(t)}{dt^2} dt=\frac{6}{\Delta_i^2}\\
\Omega_{2i-1,2i+1}^{(i)} &=\int_{t_{i}}^{t_{i+1}} \frac{d^2 h_{00}^{(i)}(t)}{dt^2}  \frac{d^2 h_{01}^{(i)}(t)}{dt^2} dt=\frac{-12}{\Delta_i^3}\\
\Omega_{2i-1,2i+2}^{(i)} &=\int_{t_{i}}^{t_{i+1}} \frac{d^2 h_{00}^{(i)}(t)}{dt^2}  \frac{d^2 h_{11}^{(i)}(t)}{dt^2} dt=\frac{6}{\Delta_i^2}\\
\Omega_{2i,2i}^{(i)} &=\int_{t_{i}}^{t_{i+1}} \frac{d^2 h_{10}^{(i)}(t)}{dt^2}  \frac{d^2 h_{10}^{(i)}(t)}{dt^2} dt=\frac{4}{\Delta_i} \\
\Omega_{2i,2i+1}^{(i)} &=\int_{t_{i}}^{t_{i+1}} \frac{d^2 h_{10}^{(i)}(t)}{dt^2}  \frac{d^2 h_{01}^{(i)}(t)}{dt^2} dt=\frac{-6}{\Delta_i^2}\\
\Omega_{2i,2i+2}^{(i)} &=\int_{t_{i}}^{t_{i+1}} \frac{d^2 h_{10}^{(i)}(t)}{dt^2}  \frac{d^2 h_{11}^{(i)}(t)}{dt^2} dt=\frac{2}{\Delta_i}\\
\Omega_{2i+1,2i+1}^{(i)} &=\int_{t_{i}}^{t_{i+1}} \frac{d^2 h_{01}^{(i)}(t)}{dt^2}  \frac{d^2 h_{01}^{(i)}(t)}{dt^2} dt=\frac{12}{\Delta_i^3}\\
\Omega_{2i+1,2i+2}^{(i)} &=\int_{t_{i}}^{t_{i+1}} \frac{d^2 h_{01}^{(i)}(t)}{dt^2}  \frac{d^2 h_{11}^{(i)}(t)}{dt^2} dt=\frac{-6}{\Delta_i^2}\\
\Omega_{2i+2,2i+2}^{(i)} &=\int_{t_{i}}^{t_{i+1}} \frac{d^2 h_{11}^{(i)}(t)}{dt^2}  \frac{d^2 h_{11}^{(i)}(t)}{dt^2} dt=\frac{4}{\Delta_i}
\end{align}
where $\Delta_i=t_{i+1}-t_i$ and $i=1,2,\ldots,n-1$. Then 
\begin{equation*}
\mathbf{\Omega}=\sum_{i=1}^{n-1}\lambda_i\Omega^{(i)}
\end{equation*}


\section{Proof of Theorem \ref{TractorSplineTheorem}}\label{AppendixTractorSplineProof}

\begin{proof}
%The proof is based on Ex. 5.7 of \citep{esl2009}, and is essentially a proof by contradiction. If $g:[a,b]\to \mathbb{R}$ is a proposed minimizer, then we can construct a cubic spline $f(t)$ that agrees with $g(t)$ and its first derivatives at $t_1,\ldots,t_n$, and is component-wise linear on $[a, t_1]$ and $[t_n, b]$. This means the first two terms of the objective function will be the same for $f(t)$ and $g(t)$. We now show that the curvature penalty ...
If $g:[a,b]\mapsto \mathbb{R}$ is a proposed minimizer, construct a cubic spline $f(t)$ that agrees with $g(t)$ and its first derivatives at $t_1,\ldots,t_n$, and is component-wise linear on $[a, t_1]$ and $[t_n, b]$. Let $h(t) = g(t)-f(t)$. Then, for $i = 1,\dots ,n-1$, 
\begin{align*}
\int_{t_i}^{t_{i+1}}f''(t)h''(t)dt &=f''(t)h'(t) \bigg\rvert_{t_i}^{t_{i+1}}-\int_{t_i}^{t_{i+1}}f'''(t)h'(t)dt \\
&= 0-f'''\left(t_i^+\right)\int_{t_i}^{t_{i+1}}h'(t)dt \\
&= -f'''\left(t_i^+\right)\left( h(t_{i+1}) -h(t_i) \right)\\
&=0.
\end{align*}
Additionally, $\int_{a}^{t_1}f''(t)h''(t)dt=\int_{t_n}^{b}f''(t)h''(t)dt=0$, since $f(t)$ is assumed linear outside the knots. Thus, for $i=0,\ldots,n$, 
\begin{align*}
\int_{t_i}^{t_{i+1}}\lvert g''(t) \rvert^2dt &= \int_{t_i}^{t_{i+1}}\lvert f''(t)+h''(t)\rvert^2 dt\\
&= \int_{t_i}^{t_{i+1}}\lvert f''(t)\rvert^2dt+2\int_{t_i}^{t_{i+1}}f''(t)h''(t)dt+\int_{t_i}^{t_{i+1}}\lvert h''(t)\rvert^2dt\\
&=\int_{t_i}^{t_{i+1}}\lvert f''(t)\rvert^2dt+\int_{t_i}^{t_{i+1}}\lvert h''(t)\rvert^2dt\\
&\geq \int_{t_i}^{t_{i+1}}\lvert f''(t)\rvert^2dt.
\end{align*}
The result $J[f]\leq J[g]$ follows since $\lambda_i>0$. 

Furthermore, equality of the curvature penalty term only holds if $g(t) = f(t)$. On $[t_1, t_n]$, we require $h''(t) = 0$ but since $h(t_i) = h'(t_i) = 0$ for $i = 1, \ldots , n$, this means $h(t) = 0$. Meanwhile on $[a, t_1]$ and $[t_n, b]$, $f''(t) = 0$ so that equality requires $g''(t)=0$. Since $f(t)$ agrees with $g(t)$ and its first derivatives at $t_1$ and $t_n$, equality is forced on both intervals.
\end{proof}


\section{Proof of Corollary \ref{TractorsplineCorollary}}\label{proofofCorollary}
\begin{proof}
By setting $\gamma\to 0$, the velocity information $v$ is taken away. The degrees of freedom of parameters decreases from $2n$ to $n$. Hence, there exists an $n\times 2n$ matrix $Q_\lambda$ restricting $n$ degrees of freedom of $\hat{\theta}$ and satisfying $Q_\lambda\hat{\theta}=0$.

The matrices $B$ and $C$ have the following property:  
\begin{align*}
& BB^\top=CC^\top=I_n, \\
& C^\top CB^\top=B^\top BC^\top=0.
\end{align*}
Denoting $G=B^\top B+ \gamma C^\top C +n\Omega_\lambda$ and giving $\hat{\theta} =(B^\top B+ \gamma C^\top C +n\Omega_\lambda)^{-1}(B^\top y+\gamma C^\top v)$, we will have $G\hat{\theta}=B^\top y+\gamma C^\top v$ and 
\begin{align*}
& BG\hat{\theta}=y+\gamma BC^\top v\\
& CG\hat{\theta}=CB^\top y+\gamma v.
\end{align*}
Further, $C^\top CG\hat{\theta}=C^\top (CB^\top y+\gamma v) = \gamma C^\top v$. If by setting $\gamma=0$, one will get $Q_\lambda = C^\top CG$, which consists of the even rows of $\Omega_\lambda$.

%For $1<i<n$, 
%\begin{align*}
%&\lambda_i\frac{6}{\Delta_i^2}\theta_{2i-3}+\lambda_i\frac{2}{\Delta_i}\theta_{2i-2}+\left( \lambda_i\frac{-6}{\Delta_i^2}+ \lambda_{i+1}\frac{6}{\Delta_{i+1}^2} \right) \theta_{2i-1} \\
%+ & \left( \lambda_i\frac{4}{\Delta_i} + \lambda_{i+1}\frac{4}{\Delta_{i+1}}\right)\theta_{2i} 
%+ \lambda_{i+1}\frac{-16}{\Delta_{i+1}^2}\theta_{2i+1} +  \lambda_{i+1}\frac{2}{\Delta_{i+1}}\theta_{2i+2}
%\end{align*}

By integrating by parts and using properties of the basis functions at the knots, one can get the even rows of $\Omega^{(i)}$, which are 
\begin{align*}
\Omega^{(i)}_{2i,2i-1}&=N''_{2i-1}\left(t_i^-\right) - N''_{2i-1}\left(t_i^+\right)  \\
\Omega^{(i)}_{2i,2i}   &=N''_{2i}\left(t_i^-\right) - N''_{2i}\left(t_i^+\right)  \\
\Omega^{(i)}_{2i,2i+1}&=N''_{2i+1}\left(t_i^-\right) - N''_{2i+1}\left(t_i^+\right) \\
\Omega^{(i)}_{2i,2i+2}&=N''_{2i+2}\left(t_i^-\right) - N''_{2i+2}\left(t_i^+\right) \\
\Omega^{(i)}_{2i+2,2i-1}&=N''_{2i-1}\left(t_{i+1}^-\right) - N''_{2i-1}\left( t_{i+1}^+ \right) \\
\Omega^{(i)}_{2i+2,2i}&=N''_{2i}\left(t_{i+1}^-\right) - N''_{2i}\left(t_{i+1}^+\right) \\
\Omega^{(i)}_{2i+2,2i+1}&=N''_{2i+1}\left(t_{i+1}^-\right) - N''_{2i+1}\left(t_{i+1}^+\right) \\
\Omega^{(i)}_{2i+2,2i+2}&=N''_{2i+2}\left(t_{i+1}^-\right) - N''_{2i+2}\left(t_i^+\right) 
\end{align*}
Thus 
\begin{equation*}
Q_\lambda = nC^\top C\Omega_\lambda = nC^\top C\sum_i\lambda_i \Omega^{(i)}.
\end{equation*}
Consequently, if and only if $\lambda$ is constant, $Q_\lambda\hat{\theta}=-\lambda\left( f''\left(t_i^+\right)-f''\left(t_i^-\right) \right)=0$, for $i=1,\ldots,n$, otherwise $Q_\lambda\theta=0$ is true but does not represent $f''\left(t_i^+\right)-f''\left(t_i^-\right)$. 

As a result, $f''(t)$ is continuous at the knots $t_i$ if $\lambda(t)$ is constant and $\gamma=0$. 


\end{proof}




\section{Proof of Lemma \ref{cvlemma}}
\begin{proof}
For any smooth curve $f$ with $\mathbf{y}^*$, we have 
\begin{align*}
&\frac{1}{n} \sum_{j=1}^n\left(y_j^*-f(t_j)\right)^2+\frac{\gamma}{n} \sum_{j=1}^n\left(v_j^*-f'(t_j)\right)^2+\sum_{j=1}^{n} \lambda_j\int_{t_j}^{t_{j+1}} f''^2dt \\
\geq &\frac{1}{n} \sum_{j\neq i}\left(y_j^*-f(t_j)\right)^2+\frac{\gamma}{n} \sum_{j\neq i}\left(v_j^*-f'(t_j)\right)^2+\sum_{j=1}^{n} \lambda_j\int_{t_j}^{t_{j+1}} f''^2dt\\
\geq &\frac{1}{n}\sum_{j\neq i}\left(y_j^*-\hat{f}^{(-i)}(t_j)\right)^2+\frac{\gamma}{n} \sum_{j\neq i}\left(v_j^*-\hat{f}'^{(-i)}(t_j)\right)^2+\sum_{j=1}^{n} \lambda_j\int_{t_j}^{t_{j+1}}  \left(\hat{f}^{''(-i)}\right)^2dt\\
= &\frac{1}{n}\sum_{j=1}^{n}\left(y_j^*-\hat{f}^{(-i)}(t_j)\right)^2+\frac{\gamma}{n} \sum_{j=1}^{n}\left(v_j^*-\hat{f}'^{(-i)}(t_j)\right)^2+\sum_{j=1}^{n} \lambda_j\int_{t_j}^{t_{j+1}}  \left(\hat{f}^{''(-i)}\right)^2dt\\
\end{align*}
%For any spline $f$ and a given $\lambda(t)$, 
%\begin{equation}
%\begin{split}
%&\frac{1}{n}\sum_{j=1}^{n}\left(y_j^*-f(t_j)\right)^2+\frac{\gamma}{n} \sum_{j=1}^{n}\left(v_j^*-f'(t_j)\right)^2+\int\lambda(t) f''^2 \\
%\geq&\frac{1}{n}\sum_{j\neq i}^{n}\left(y_j^*-f(t_j)\right)^2+\frac{\gamma}{n} \sum_{j\neq i}^{n}\left(v_j^*-f'(t_j)\right)^2+\int\lambda(t) f''^2\\
%\geq&\frac{1}{n}\sum_{j\neq i}^{n}\left(y_j^*-\hat{f}^{(-i)}(t_j)\right)^2+\frac{\gamma}{n} \sum_{j\neq i}^{n}\left(v_j^*-\hat{f}'^{(-i)}(t_j)\right)^2+\int\lambda(t) \hat{f}^{(-i)''2}\\
%=&\frac{1}{n}\sum_{j=1}^{n}\left(y_j^*-\hat{f}^{(-i)}(t_j)\right)^2+\frac{\gamma}{n} \sum_{j=1}^{n}\left(v_j^*-\hat{f}'^{(-i)}(t_j)\right)^2+\int\lambda(t) \hat{f}^{(-i)''2}
%\end{split}
%\end{equation}
by the definition of $\mathbf{\hat{f}}^{(-i)}$, $\mathbf{\hat{f}}'^{(-i)}$ and the fact that $y_i^*=\hat{f}^{(-i)}(t_i)$, $v_i^*=\hat{f}'^{(-i)}(t_i)$. It follows that $\hat{f}^{(-i)}$ is the minimizer of the objective function \eqref{tractorsplineObjective}, so that
\begin{align*}
\mathbf{\hat{f}}^{(-i)}&=S\mathbf{y}^*+\gamma T\mathbf{v}^*\\
\mathbf{\hat{f}}'^{(-i)}&=U\mathbf{y}^*+\gamma V\mathbf{v}^*
\end{align*}
as required.
\end{proof}


\section{Proof of Theorem \ref{tractorsplinecvscore}}

\begin{proof}
\begin{equation}\label{th3proofeq1}
\begin{split}
\hat{f}^{(-i)}(t_i)-y_i=& \sum_{j=1}^{n}S_{ij}y_j^*+ \gamma \sum_{j=1}^{n}T_{ij}v_j^*-y_i^*\\
=&\sum_{j\neq i}^{n}S_{ij}y_j+ \gamma \sum_{j\neq i}^{n}T_{ij}v_j+S_{ii}\hat{f}^{(-i)}(t_i)+\gamma T_{ii}\hat{f}'^{(-i)}(t_i)-y_i\\
=&\sum_{j=1}^{n}S_{ij}y_j+ \gamma \sum_{j=1}^{n}T_{ij}v_j+S_{ii}\left(\hat{f}^{(-i)}(t_i)-y_i\right)+\gamma T_{ii}\left(\hat{f}'^{(-i)}(t_i)-v_i\right)-y_i\\
=&\left(\hat{f}(t_i)-y_i\right)+S_{ii}\left(\hat{f}^{(-i)}(t_i)-y_i\right)+\gamma T_{ii}\left(\hat{f}'^{(-i)}(t_i)-v_i\right).
\end{split}
\end{equation}
Additionally, 
\begin{equation}
\begin{split}
\hat{f}'^{(-i)}(t_i)-v_i=& \sum_{j=1}^{n}U_{ij}y_j^*+ \gamma \sum_{j=1}^{n}V_{ij}v_j^*-v_i^*\\
=&\sum_{j\neq i}^{n}U_{ij}y_j+ \gamma \sum_{j\neq i}^{n}V_{ij}v_j+U_{ii}\hat{f}^{(-i)}(t_i)+\gamma V_{ii}\hat{f}'^{(-i)}(t_i)-v_i\\
=&\sum_{j=1}^{n}U_{ij}y_j+ \gamma \sum_{j=1}^{n}V_{ij}v_j+U_{ii}\left(\hat{f}^{(-i)}(t_i)-y_i\right)+\gamma V_{ii}\left(\hat{f}'^{(-i)}(t_i)-v_i\right)-v_i\\
=&\left(\hat{f}'(t_i)-v_i\right)+U_{ii}\left(\hat{f}^{(-i)}(t_i)-y_i\right)+\gamma V_{ii}\left(\hat{f}'^{(-i)}(t_i)-v_i\right).
\end{split}
\end{equation}
Thus 
\begin{equation}\label{th3proofeq2}
\hat{f}'^{(-i)}(t_i)-v_i = \frac{\hat{f}'(t_i)-v_i}{1-\gamma V_{ii}}+ \frac{U_{ii}\left(\hat{f}^{(-i)}(t_i)-y_i\right)}{1-\gamma V_{ii}}.
\end{equation}
By substituting equation \eqref{th3proofeq2} to \eqref{th3proofeq1}, we get
\begin{equation*}
\hat{f}^{(-i)}(t_i)-y_i=\frac{\hat{f}(t_i)-y_i+\gamma \frac{T_{ii}}{1-\gamma V_{ii}}\left(\hat{f}'(t_i)-v_i\right)}{1-S_{ii}-\gamma\frac{T_{ii}}{1-\gamma V_{ii}}U_{ii}}.
\end{equation*}
Consequently, 
\begin{equation*}
CV(\lambda,\gamma)=\frac{1}{n}\sum_{i=1}^{n}\left( \frac{\hat{f}(t_i)-y_i+\gamma \frac{T_{ii}}{1-\gamma V_{ii}}\left(\hat{f}'(t_i)-v_i\right)}{1-S_{ii}-\gamma\frac{T_{ii}}{1-\gamma V_{ii}}U_{ii}}\right)^2.
\end{equation*}
\end{proof}


%\clearpage


\section{Reconstructions at SNR=3}
\begin{figure}[!ht]
    \centering
    \includegraphics[width=0.45\textwidth]{Appendices/ggplot/ggBlocksPenaltyLineSNR3.pdf}
    \includegraphics[width=0.45\textwidth]{Appendices/ggplot/ggBumpsPenaltyLineSNR3.pdf}
    \includegraphics[width=0.45\textwidth]{Appendices/ggplot/ggHeaviPenaltyLineSNR3.pdf}
    \includegraphics[width=0.45\textwidth]{Appendices/ggplot/ggDopplerPenaltyLineSNR3.pdf}
\caption{Reconstructions of generated \textit{Blocks}, \textit{Bumps}, \textit{HeaviSine} and \textit{Doppler} functions by V-spline at SNR=3. The penalty values $\lambda(t)$ in V-spline are projected into reconstructions. The blacks dots are the measurements. The bigger blacks dots indicate the larger penalty values.}\label{TractorsplineSNR3}
\end{figure}


\section{Residual Analysis of Simulations}
\begin{figure}[!ht]
    \centering
    \begin{subfigure}{0.45\textwidth}
    \centering
    \includegraphics[width=\textwidth]{Chapters/02TractorSplineTheory/plot/ggplot/ggacfBlocks7.pdf}
    \caption{ACF of residuals from \textit{Blocks}}
    \end{subfigure}%
    \begin{subfigure}{0.45\textwidth}
    \centering
    \includegraphics[width=\textwidth]{Chapters/02TractorSplineTheory/plot/ggplot/ggacfBumps7.pdf}
    \caption{ACF of residuals from \textit{Bumps}}
    \end{subfigure}
    \begin{subfigure}{0.45\textwidth}
    \centering
    \includegraphics[width=\textwidth]{Chapters/02TractorSplineTheory/plot/ggplot/ggacfHeavi7.pdf}
    \caption{ACF of residuals from \textit{HeaviSine} }
    \end{subfigure}
    \begin{subfigure}{0.45\textwidth}
    \centering
    \includegraphics[width=\textwidth]{Chapters/02TractorSplineTheory/plot/ggplot/ggacfDoppler7.pdf}
    \caption{ACF of residuals from \textit{Doppler}}
    \end{subfigure}
\caption{ACF of residuals at SNR level of 7.}\label{tractorsplineSNR7acf}
 \end{figure}
\begin{figure}[!ht]
    \centering
    \begin{subfigure}{0.45\textwidth}
    \centering
    \includegraphics[width=\textwidth]{Chapters/02TractorSplineTheory/plot/ggplot/ggacfBlocks3.pdf}
    \caption{ACF of residuals from \textit{Blocks}}
    \end{subfigure}%
    \begin{subfigure}{0.45\textwidth}
    \centering
    \includegraphics[width=\textwidth]{Chapters/02TractorSplineTheory/plot/ggplot/ggacfBumps3.pdf}
    \caption{ACF of residuals from \textit{Bumps}}
    \end{subfigure}
    \begin{subfigure}{0.45\textwidth}
    \centering
    \includegraphics[width=\textwidth]{Chapters/02TractorSplineTheory/plot/ggplot/ggacfHeavi3.pdf}
    \caption{ACF of residuals from \textit{HeaviSine}}
    \end{subfigure}
    \begin{subfigure}{0.45\textwidth}
    \centering
    \includegraphics[width=\textwidth]{Chapters/02TractorSplineTheory/plot/ggplot/ggacfDoppler3.pdf}
    \caption{ACF of residuals from \textit{Doppler}}
    \end{subfigure}
\caption{ACF of residuals at SNR level of 3.}\label{tractorsplineSNR3acf}
 \end{figure}

\begin{figure}[!ht]
    \centering
    \begin{subfigure}{\textwidth}
    \centering
    \includegraphics[width=0.45\linewidth]{Chapters/02TractorSplineTheory/plot/ggplot/ggRealdataXYResidualsXpoints.pdf}
    \includegraphics[width=0.45\linewidth]{Chapters/02TractorSplineTheory/plot/ggplot/ggRealdataXYResidualsXhist.pdf}
    \caption{residuals of $x$ }
    \end{subfigure}
    \begin{subfigure}{\textwidth}
    \centering
    \includegraphics[width=0.45\linewidth]{Chapters/02TractorSplineTheory/plot/ggplot/ggRealdataXYResidualsYpoints.pdf}
    \includegraphics[width=0.45\linewidth]{Chapters/02TractorSplineTheory/plot/ggplot/ggRealdataXYResidualsYhist.pdf}
    \caption{residuals of $y$ }
    \end{subfigure}
\caption{Residuals of 2-dimensional real data reconstruction }\label{tractorsplineResidualsRealdata}
 \end{figure}

